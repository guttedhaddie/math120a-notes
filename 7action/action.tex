\graphicspath{{4perm/asy/},{7action/asy/}}
\setcounter{section}{6}

\section{Group Actions}\label{chap:action}


\subsection{Group Actions, Fixed Sets and Isotropy Subgroups}\label{sec:action1}

In this final chapter, we revisit a central idea: groups are interesting and useful often because of how they \emph{transform sets.} Recall how the symmetric group $S_n$ was defined in terms of what its elements do to the set $\{1,\ldots,n\}$. This is an example of a general situation.

\begin{defn}{}{}
A group $G$ \emph{acts\footnotemark} on a set $X$ via a map $\cdot:G\times X\to X$ if,
\begin{enumeratea}
  \item $\forall x\in X,\ e\cdot x=x$,\quad and,
  \item $\forall x\in X,\ g,h\in G,\  g\cdot (h\cdot x)=(gh)\cdot x$.
\end{enumeratea}
\end{defn}

\footnotetext{This is really a \emph{left} action. There is an analogous definition of a \emph{right action.} In this course, all actions will be left.}

Part (b) says $g\mapsto g\cdot$ is a homomorphism of \emph{binary structures} (the functions $X\to X$ needn't form a group).

\begin{examples}{}{actioneasy}
\exstart The symmetric $S_n$ group acts on $X=\{1,2,\ldots,n\}$. As a sanity check:\vspace{-3pt}
\begin{enumerate}\setcounter{enumi}{1}
  \item[]\begin{enumerate}
		\item $e(x)=x$ for all $x\in \{1,\ldots,n\}$.
		\item $\sigma\bigl(\tau(x)\bigr)=(\sigma\tau)(x)$ is composition of functions!
	\end{enumerate}
  \item Any group $G$ acts on itself by left multiplication. This is essentially Cayley's Theorem (\ref{thm:cayley}). It also acts on itself by conjugation ($c_g\circ c_h=c_{gh}$ is Theorem \ref{thm:aut}).
  \item If $X$ is the set of orientations of a regular $n$-gon such that one vertex is at $(1,0)$ and the center is at $(0,0)$, then $D_n$ acts on $X$ by rotations and reflections. Note that $X$ has cardinality $2n$.
  \item Matrix groups act on vector spaces by matrix multiplication. For example the orthogonal group $\rO_2(\R)$ can be seen to transform vectors via rotations and reflections.
	\[
		\rO_2(\R)\times\R^2\to\R^2:(A,\vv)\mapsto A\vv
	\]
  
  \item\label{ex:orthmultaction} A group can act on many different sets. Here are three further actions of the orthogonal group:
	\begin{enumerate}
  	\item[i.] $\rO_2(\R)$ acts on the set $X=\{1,-1\}$ via $A\cdot x:=(\det A)x$.
		\item[ii.] $\rO_2(\R)$ acts on the set $X=\R^3$ via $A\cdot\vv:=A(v_1\vi+v_2\vj)+v_3\vk$.
		\item[iii.] $\rO_2(\R)$ acts on the unit circle $X=S^1\subseteq\R^2$ via matrix multiplication $A\cdot\vv:=A\vv$.
	\end{enumerate}
\end{enumerate}
\end{examples}

We often use an action to visualize a group; in this context, some actions are better than others. Consider the three actions of $\rO_2(\R)$ in part \ref*{ex:orthmultaction} above:
\begin{enumerate}\itemsep0pt
  \item[i.] The set $X$ is very small. Many matrices act in exactly the same way so the action is an unhelpful means of visualizing the group.
	\item[ii.] The set $X$ feels too large. The action leaves any vertical vector untouched.
	\item[iii.] The circle $X=S^1$ is large enough so that the action of distinct matrices can be distinguished without being inefficiently large.\footnote{A \emph{Goldilocks} action, perhaps?}
\end{enumerate}

\goodbreak

These notions can be formalized.



\begin{defn}{}{}
Let $G\times X\to X$ be an action.
\begin{enumerate}
  \item The \emph{fixed set} of $g\in G$ is the set
  \[
  	\Fix(g):=\{x\in X:g\cdot x=x\} \tag{also written $X_g$, though we won't do this}
  \]
  \item The \emph{isotropy subgroup} or \emph{stabilizer} of $x\in X$ is the set
  \[
  	\Stab(x):=\{g\in G:g\cdot x=x\} \tag{also written $G_x$}
  \] 
  \item The action is \emph{faithful} if the only element of $G$ which fixes everything is the identity. This can be stated in two equivalent ways:
  \begin{enumerate}
    \item $\Fix(g)=X\iff g=e$\qquad\qquad\qquad (b)\lstsp$\bigcap\limits_{x\in X}\Stab(x)=\{e\}$
  \end{enumerate}
  \item The action is \emph{transitive} if any element of $X$ may be transformed to any other:
	\[
		\forall x,y\in X,\ \exists g\in G\ \text{ such that }\  y=g\cdot x
	\]
\end{enumerate}
\end{defn}

\begin{examples*}{\ref{ex:actioneasy} cont}{}
	\exstart The action of $S_n$ on $\{1,2,\ldots,n\}$ is both faithful and transitive:
	\begin{enumerate}\setcounter{enumi}{1}
	  \item[]\begin{description}
	  	\item[\normalfont\emph{Faithful}:] if $\sigma(x)=x$ for all $x\in\{1,2,\ldots,n\}$, then $\sigma=e$.
	  	\item[\normalfont\emph{Transitive}:] if $x\neq y$, then the 2-cycle $(x\,y)$ maps $x\mapsto y$.
	  \end{description}
	  \item The action of a group on itself by left multiplication is both faithful and transitive. Conjugation is more complex: in most situations it is neither.
	  \item $D_n$ acts faithfully and transitively on the orientations of the $n$-gon.
	  \item The action of $\rO_2(\R)$ on $\R^2$ is faithful but not transitive: for instance the zero vector cannot be transformed into any other vector so $\Stab(\V0)=\rO_2(\R)$.
	  \item We leave these as exercises.
	%   \begin{enumerate}
	%     \item[i.] The action $A\cdot x=\det(A)x$ is transitive, but not faithful: indeed if $A$ is any orthogonal matrix with determinant 1, then $\Fix(A)=X$.
	%     \item[ii.] The action of $\rO_2(\R)$ on $\R^3$ is faithful but not transitive. It is impossible to transform, say, the zero vector into anything else.
	%     \item[iii.] The action of $\rO_2(\R)$ is both faithful and transitive.
	% 	\end{enumerate}
	\end{enumerate}
\end{examples*}


\begin{lemm}{}{}
	For each $x\in X$, the stabilizer $\Stab(x)$ is indeed a subgroup of $G$.
\end{lemm}

\begin{proof}
	$\Stab(x)$ is a non-empty subset of $G$ since $e\in \Stab(x)$. It sufficient to show that it is closed under multiplication and inverses. Let $g,h\in\Stab(x)$, then
	\[
		(gh)\cdot x=g\cdot(h\cdot x)
		=g\cdot x=x\implies gh\in\Stab(x)
	\]
	Moreover
	\[
		x=g\cdot x
		\implies g^{-1}\cdot x=g^{-1}\cdot (g\cdot x)
		=(g^{-1}g)\cdot x=e\cdot x=x
		\tag*{\qedhere}
	\]
\end{proof}

\begin{example}{}{actiond3}
	The dihedral group $D_3=\{e,\rho_1,\rho_2,\mu_1,\mu_2,\mu_3\}$ acts on the set $X$ of vertices of an equilateral triangle.\footnotemark\ The fixed sets and stabilizers for this action are as follows:\par
	\begin{minipage}[t]{0.74\linewidth}\vspace{-10pt}
		\[
			\begin{array}[t]{c|c}
				\text{Element $g$}&\Fix(g)\\\hline
				e&\{1,2,3\}\\
				\rho_1&\emptyset\\
				\rho_2&\emptyset\\
				\mu_1&\{1\}\\
				\mu_2&\{2\}\\
				\mu_3&\{3\}
			\end{array}
			\qquad
			\begin{array}[t]{c|c}
				\text{Vertex $x$}&\Stab(x)\\\hline
				1&\{e,\mu_1\}\\
				2&\{e,\mu_2\}\\
				3&\{e,\mu_3\}
			\end{array}
		\]
	\end{minipage}
	\hfill
	\begin{minipage}[t]{0.25\linewidth}\vspace{-20pt}
		\flushright\includegraphics{perm-d3}
	\end{minipage}\smallbreak
	$D_3$ also acts on the set of \emph{edges} of the triangle $Y=\bigl\{\{1,2\},\{1,3\},\{2,3\}\bigr\}$. You needn't write all these out since, by the symmetry of the triangle, stabilizing an edge is equivalent to stabilizing its opposite vertex. Still, here is the data:
	\[
		\begin{array}[t]{c|c}
			\text{Element $g$}&\Fix(g)\\\hline
			e&\bigl\{1,2,3\bigr\}\\
			\rho_1&\emptyset\\
			\rho_2&\emptyset\\
			\mu_1&\bigl\{\{2,3\}\bigr\}\\
			\mu_2&\bigl\{\{1,3\}\bigr\}\\
			\mu_3&\bigl\{\{1,2\}\bigr\}
		\end{array}
		\qquad
		\begin{array}[t]{c|c}
			\text{Edge $\{x,y\}$}&\Stab(\{x,y\})\\\hline
			\{1,2\}&\{e,\mu_3\}\\
			\{1,3\}&\{e,\mu_2\}\\
			\{2,3\}&\{e,\mu_1\}
		\end{array}
	\]
\end{example}

\footnotetext{%
	Recall that $\rho_1$ rotates \ang{120} counter-clockwise, that $\rho_2=\rho_1^2$ and that $\mu_i$ reflects across the altitude through vertex $i$.
}


\begin{exercises}{}{}
	Key concepts:
	\begin{quote}
		\emph{(left) action\quad $\Fix(g)$\quad $\Stab(x)\le G$\quad faithful/transitive actions}
	\end{quote}
	
	\begin{enumerate}
	  \item For part \ref*{ex:orthmultaction} of Example \ref{ex:actioneasy}, determine whether each action is faithful and/or transitive.
	
		
		\item\label{exs:orbitsigma1} Let $G=\ip{\sigma}\le S_6$ where $\sigma=(1\,2\,3\,4\,5\,6)$. $G$ acts on the set $X=\{1,2,3,4,5,6\}$ in a natural way.
		\begin{enumerate}
		  \item State the fixed sets and stabilizers for this action.
		  \item Is the action of $G$ faithful? Transitive?
		\end{enumerate}
		
		
		\item\label{exs:orbitsigma2} Repeat the previous question when $\sigma=(1\,3)(2\,4\,6)$.
		
	
		\item Mimic Example \ref{ex:actiond3} for the actions of $D_4$ on $X=\{$vertices$\}$ and $Y=\{$edges$\}$ of the square.\\
		(\emph{Use whatever notation you like; $\rho,\mu,\delta$ or cycle notation})
	
	
		\item Suppose $G$ acts on $X$.
		\begin{enumerate}
		  \item Let $Y\subseteq X$ and define $\Stab Y=\{g\in G:\forall y\in Y, g\cdot y=y\}$. Prove that $\Stab Y$ is a subgroup of $G$.
			\item Let $G$ act on itself by conjugation ($X=G$!). What is another name for the subgroup $\Stab G$?
		\end{enumerate}
		
	
		\item Suppose $G$ has a left action on $X$. Prove that $G$ acts faithfully on $X$ if and only if no two distinct elements of $G$ have the same action on every element.
	
	\end{enumerate}
\end{exercises}

\clearpage



\subsection{Orbits \& Burnside's Formula}

We first encountered orbits in the context of the symmetric groups $S_n$. The same idea applies to any action.

\begin{defn}{}{}
	Let $G\times X\to X$ be an action. The \emph{orbit} of $x\in X$ under $G$ is the set of elements into which $x$ may be transformed:
	\[
		Gx=\{g\cdot x:g\in G\}\subseteq X
	\]
\end{defn}


\begin{examples}{}{}
	\exstart If $X=\{1,2,\ldots n\}$ and $G=\ip{\sigma}\le S_n$, then
	\[
		Gx=\{\sigma^k(x):k\in\Z\}=\orb_x(\sigma)
	\]
	The definition of orbits therefore coincides with that seen earlier in the course.
	\begin{enumerate}\setcounter{enumi}{1}
	  \item A transitive \emph{action\footnotemark} has only one orbit.
	  
	  \item If $\rO_2(\R)$ acts on $\R^2$ by matrix multiplication, then the orbits are circles centered at the origin!
	\end{enumerate}
\end{examples}

\footnotetext{%
	Unhelpfully, we now have two distinct meanings of \emph{transitive}; one for equivalence relations and one for actions.%
}


\begin{lemm}{}{actionorbit}
	The orbits of an action partition $X$.
\end{lemm}

Since this is almost identical to the corresponding result for orbits in $S_n$ (Lemma \ref{lemm:orbit}), we leave the proof as an exercise.\smallbreak

Our next result is analogous to Lemma \ref{lemm:homobij}, where we counted the number of (left) cosets of $\ker\phi$.

% \begin{proof}
% Define $x\sim y\iff y\in Gx\iff \exists g\in G$ such that $g\cdot x=y$. We claim that $\sim$ is an equivalence relation.
% \begin{description}
% 	\item[\normalfont\emph{Reflexivity}:] $x=e\cdot x\implies x\sim x$.
% 	\item[\normalfont\emph{Symmetry}:] $g\cdot x=y\implies x=g^{-1}\cdot y$, thus $x\sim y\implies y\sim x$.
% 	\item[\normalfont\emph{Transitivity}:] If $x\sim y$ and $y\sim z$, then $y=g\cdot x$ and $z=h\cdot y$ for some $g,h\in G$. But then $z=(hg)\cdot x$, whence $x\sim z$.\qedhere
% \end{description}
% \end{proof}



\begin{lemm}{}{actionorbits2}
	The cardinality of the orbit $Gx$ is the index of the isotropy subgroup $\Stab(x)$:
	\[
		\nm{Gx}=\bigl(G:\Stab(x)\bigr)
	\]
\end{lemm}


\begin{proof}
	Observe that
	\[
		g\cdot x=h\cdot x
		\iff h^{-1}g\cdot x=x
		\iff h^{-1}g\in \Stab(x)
		\iff g\Stab(x)=h\Stab(x)
	\]
	The contrapositive says that distinct elements of the orbit $Gx$ correspond to distinct left cosets.
\end{proof}


\begin{example}{}{burnsimple}
	Let $\sigma=(1\,4)(2\,7\,3)\in S_7$. Consider $X=\{1,2,3,4,5,6,7\}$ under the action of the cyclic group $G=\ip{\sigma}$. The orbits are precisely the disjoint cycles: $\{1,4\}$, $\{2,3,7\}$, $\{5\}$, $\{6\}$. Observe that $G$ has six elements:
	\[
		e,\quad 
		\sigma=(1\,4)(2\,7\,3),\quad 
		\sigma^2=(2\,3\,7),\quad 
		\sigma^3=(1\,4),\quad 
		\sigma^4=(2\,7\,3),\quad 
		\sigma^5=(1\,4)(2\,3\,7)
		\]
	The Lemma is easily verifiable: for instance if $x=3$,
	\begin{gather*}
		\Stab(x)=\{\tau\in G:\tau(3)=3\}=\{\sigma^k:\sigma^k(3)=3\}=\{e,\sigma^3\}\\
		\implies (G:\Stab(x))=\frac 62=3=\nm{\{2,3,7\}}=\nm{Gx}
	\end{gather*}
\end{example}



It is often useful to count the \emph{number} of orbits of an action. For \emph{finite} actions, this turns out to be possible in two different ways.

\begin{thm}{Burnside's formula}{}
	Let $G$ be a finite group acting on a finite set $X$. Then the number of orbits in $X$ under $G$ satisfies
	\[
		\text{\# orbits}
		=\frac{1}{\nm G}\sum_{x\in X}\nm{\Stab(x)}
		=\frac{1}{\nm G}\sum_{g\in G}\nm{\Fix(g)}
	\]
\end{thm}

\begin{proof}
	By Lemma \ref{lemm:actionorbits2},
	It follows that
	\[
		\frac 1{\nm G}\sum_{x\in X}\nm{\Stab(x)} 
		=\sum_{x\in X}\frac{\nm{\Stab(x)}}{\nm G} 
		=\sum_{x\in X}\frac 1{(G:\Stab(x))} 
		=\sum_{x\in X}\frac{1}{\nm{Gx}}
		\tag*{($\ast$)}
	\]
	Consider a fixed orbit $Gy$. Since $\nm{Gx}=\nm{Gy}$ for each $x\in Gy$, we see that
	\[
		\sum_{x\in Gy}\frac{1}{\nm{Gx}}
		=\frac{\nm{Gy}}{\nm{Gy}}=1
	\]
	The sum ($\ast$) therefore counts 1 for each distinct orbit in $X$ and therefore returns the number of orbits.\smallbreak
	For the second equality, observe that
	\[
		S=\{(g,x)\in G\times X:g\cdot x=x\}
	\]
	has cardinality
	\[
		\nm S=\sum\limits_{x\in X}\nm{\Stab(x)}
		=\sum\limits_{g\in G}\nm{\Fix(g)}
		\tag*{\qedhere}
	\]
\end{proof}



\begin{example*}{\ref{ex:burnsimple} cont}{}
	When $G=\ip\sigma=\ip{(1\,4)(2\,7\,3)}$ acts on $X=\{1,2,3,4,5,6,7\}$, the stabilizers and fixed sets are as follows:
	\[
		\begin{array}{c|c}
			x\in X&\Stab(x)\\\hline
			1&\{e,\sigma^2,\sigma^4\}\\
			2&\{e,\sigma^3\}\\
			3&\{e,\sigma^3\}\\
			4&\{e,\sigma^2,\sigma^4\}\\
			5&G=\{e,\sigma,\sigma^2,\sigma^3,\sigma^4,\sigma^5\}\\
			6&G\\
			7&\{e,\sigma^3\}
		\end{array}
		\qquad\qquad
		\begin{array}{c|c}
			g\in G&\Fix(g)\\\hline
			e&X=\{1,2,3,4,5,6,7\}\\
			\sigma&\{5,6\}\\
			\sigma^2&\{1,4,5,6\}\\
			\sigma^3&\{2,3,5,6,7\}\\
			\sigma^4&\{1,4,5,6\}\\
			\sigma^5&\{5,6\}\\
			\multicolumn{2}{c}{}
		\end{array}
	\]
	Burnside's formula just sums the number of elements in all of the subsets in the right column of each table:
	\begin{align*}
		4=\text{\# orbits}
		&=\frac{1}{\nm G}\sum_{x\in X}\nm{\Stab(x)}
			=\frac{1}{6}(3+2+2+3+6+6+2)\\
		&=\frac{1}{\nm G}\sum_{g\in G}\nm{\Fix(g)}
			=\frac{1}{6}(7+2+4+5+4+2)
	\end{align*}
\end{example*}

\goodbreak

One reason to count the number of orbits of an action is that we often want to consider objects as equivalent if they differ by the action of some simple group.

\begin{example}{}{toy}
	A child's toy consists of a wooden equilateral triangle where the edges are to be painted using any choice of colors from the rainbow. How many distinct toys could we create?\smallbreak
	There are two problems: we need to describe the variety of possible toys, and we need to know what \emph{distinct} means!\smallbreak
	We use group actions to address both problems:
	\begin{itemize}
	  \item A toy may be considered as a subset of $X=\{$painted triangles$\}=\{$ordered color triples$\}$. Since there are 7 choices for the color of each edge, we see that $\nm X=7^3=343$ is a large set!
		\begin{minipage}[t]{0.55\linewidth}\vspace{0pt}
		  \item Two toys are equivalent if they differ by a rotation in 3-dimensions. This amount to the natural action of $D_3$ on $X$: for instance
		  \[
		  	\rho_1\cdot (\text{red,green,violet})
		  	=(\text{violet,red,green})
		  \]
		\end{minipage}
		\hfill
		\begin{minipage}[t]{0.4\linewidth}\vspace{0pt}
			\hfill\includegraphics{actions-triangles}
		\end{minipage}
	\end{itemize}
	The number of orbits is the number of distinct toys, which we may compute using Burnside. Since it would be time consuming to compute the stabilizer of each element of $X$, we use the fixed set approach.
	\begin{itemize}
	  \item Identity $e$: \ Plainly $\Fix(e)=X$, since $e$ leaves every coloring unchanged.
	  \item Rotations $\rho_1,\rho_2$: \ If a color-scheme is fixed by $\rho_j$, then all pairs of adjacent edges must be the same color. The only color-schemes fixed by $\rho_j$ are those where all sides have the same color, whence $\nm{\Fix(\rho_i)}=7$.
	  \item Reflections $\mu_1,\mu_2,\mu_3$: \ Since $\mu_j$ swaps two edges, anything in its fixed set must have these edges the same. We have 7 choices for the color of the switched edges, and an independent choice of 7 colors for the other edge, whence $\nm{\Fix(\mu_j)}=7^2=49$.
	\end{itemize}
	The number of distinct toys is therefore
	\begin{align*}
		\text{\# orbits}
		&=\frac 1{\nm{D_3}}\sum_{\sigma\in D_3}\nm{\Fix(\sigma)}
			=\frac 16(7^3+7+7+7^2+7^2+7^2)\\
		&=\frac 76(49+1+1+7+7+7)=84
	\end{align*}
	The question was a little tricky because we are allowed multiple sides to have the same color. A simpler version would restrict to the situation where all sides had to be different colors. In this case $D_3$ acts on a set of color schemes with cardinality $\nm Y=7\cdot 6\cdot 5=210$. Moreover, only the identity element has a non-empty fixed set; in this situation the number of distinct toys would be
	\[
		\text{\# orbits}
		=\frac 1{\nm{D_3}}\sum_{\sigma\in D_3}\nm{\Fix(g)}
		=\frac 16(210+0+\cdots+0)
		=\frac{210}6=35
	\]
	Of course you could answer these questions by pure combinatorics without any resort to group theory!
\end{example}

\goodbreak

\boldsubsubsection{Dice-rolling for Geeks!}

\begin{minipage}[t]{0.69\linewidth}\vspace{-10pt}
	Games like Dungeons \& Dragons make use of several differently shaped dice: rather than simply using the standard 6-sided cubic die, situations might require rolling, say, a 4-sided tetrahedral die or a 20-sided icosahedral die.\smallbreak
	Since dice are designed for rolling, we consider two dice to be the same if one can be rotated into the other. Play with the two tetrahedral dice on the right; you should be convinced that you cannot rotate one to make the other so these dice are distinct.\smallbreak
	It is not difficult to see that, up to rotations, these are the \emph{only} tetrahedral dice just by counting!
	\begin{itemize}
	  \item Place face 4 on the table.
	  \item When looking from above, the remaining faces are numbered 1, 2, 3 either clockwise or counter-clockwise. 
	\end{itemize}
	For larger dice, this approach is not practical! However, with a little thinking about symmetry groups, Burnside's formula will ride to the rescue.\smallbreak
	Suppose a regular polyhedron has $f$ faces, each with $n$ sides.
	\end{minipage}\hfill\begin{minipage}[t]{0.3\linewidth}\vspace{-20pt}
	\flushright
		\href{https://www.math.uci.edu/~ndonalds/math120a/perm-a4.html}{\includegraphics[scale=0.15]{perm-a4.png}}\bigbreak
	  \href{https://www.math.uci.edu/~ndonalds/math120a/actions-a4-2.html}{\includegraphics[scale=0.15]{actions-a4-2.png}}
\end{minipage}\par

\begin{itemize}
  \item The faces may be labelled 1 thorough $f$ in $f!$ distinct ways: the set of distinct labellings is $X$.
  \item We may rotate the polyhedron so that any face is mapped to any other, \emph{in any orientation.} It follows that the rotation group $G$ has $fn$ elements.
	\item Each non-identity element of the rotation group moves at least one face, whence
	\[
		\nm{\Fix(g)}=
		\begin{cases}
			X&\text{if }g=e\\
			\emptyset&\text{if }g\neq e
		\end{cases}
	\]
	\item The number of distinct dice for a regular polyhedron is therefore
	\[
		\text{\# orbits}
		=\frac 1{\nm G}\nm{\Fix(e)}
		=\frac{\nm X}{\nm G}
		=\frac{f!}{fn}
		=\frac{(f-1)!}{n}
	\]
	We don't need to know what the rotation group is, only its \emph{order}. For completeness, here are all the possibilities for the regular platonic solids.
	\begin{quote}
		\begin{tabular}{l|c|c|c|l}
			Polyhedron & $f$ & $n$ & Rotation Group & $\#$ distinct dice\\\hline
			Tetrahedron & 4 & 3 & $A_4$ & 2\\
			Cube & 6 & 4 & $S_4$ & 30\\
			Octahedron & 8 & 3 &  $S_4$ & 1,680\\
			Dodecahedron & 12 & 5 & $A_5$ & 7,983,360\\
			Icosahedron & 20 & 3 & $A_5$ & 40,548,366,802,944,000
		\end{tabular}
	\end{quote}
\end{itemize} 


\goodbreak

\boldsubsubsection{Subgroups of Prime Order \& the Class Equation}

We finish with a taste of where group theory traditionally goes next.\smallbreak

Suppose $G$ acts on a finite set $X$ and denote by $X_G=\{x\in X:\forall g\in G, g\cdot x=x\}$ the subset of $X$ that is fixed by the action of $G$ (the set of 1-element orbits). Suppose also that $x_1,\ldots,x_r$ are representatives of the distinct remaining (larger) orbits. Then, by counting elements,
\[
	\nm X=\nm{X_G}+\sum_{j=1}^r\nm{Gx_j} 
	=\nm{X_G}+\sum_{j=1}^r\bigl(G:\Stab(x_j)\bigr)
	=\nm{X_G}+ \sum_{j=1}^r \frac{\nm G}{\nm{\Stab(x_j)}}
\]
When $G$ acts on itself by conjugation, the 1-element orbits together comprise the center of $G$ and we obtain the \emph{class equation}:
\[
	\nm{G} =\nm{Z(G)}+\sum_{j=1}^r \frac{\nm G}{\nm{\Stab(x_j)}}
\]

\begin{example}{}{}
	Since the conjugacy classes in $S_4$ are the cycle types, the class equation reads
	\[
		24=\nm{\{e\}} +\nm{\text{2-cycles}} +\nm{\text{3-cycles}} +\nm{\text{4-cycles}} +\nm{\text{2,2-cycles}}
		=1+6+8+6+3
	\]
\end{example}

Here is an example of how the class equation may be applied.

\begin{lemm}{}{}
	Suppose $G$ is a non-abelian group whose order is divisible by a prime $p$. Then $G$ has a \emph{proper} subgroup whose order is divisible by $p$.
\end{lemm}

\begin{proof}
	Since $G$ is non-abelian, $Z(G)$ is a proper subgroup. Let $x$ be any element \emph{not} in the center. Then
	\[
		2\le \nm{Gx}=\frac{\nm G}{\nm{\Stab(x)}} 
		\implies \Stab(x)\text{ is a proper subgroup of }G
	\]
	If $p$ divides $\nm{\Stab(x)}$, then we're done. If not, then $p$ divides $\nm{Gx}=\bigl(G:\Stab(x)\bigr)$. If this holds for all non-trivial orbits, the class equation says that $\nm{Z(G)}$ is divisible by $p$.
\end{proof}


\begin{thm}{Cauchy}{}
	If a prime $p$ divides $\nm G$, then $G$ contains a subgroup/element of order $p$. 
\end{thm}

% 
% 
% \begin{example}{}{}
% 	$S_4\times \Z_{10}\times D_6$ has order $24\cdot 10\cdot 12=2^6\cdot 3^2\cdot 5$ and thus has subgroups of orders 2, 3 \& 5 by Cauchy's Theorem. It of course has subgroups of many other orders.
% \end{example}


\begin{proof}
	\begin{enumerate}
	  \item A proof when $G$ is abelian is in Exercise \ref{exs:cauchysthmabelian}.
	  \item If $G$ is non-abelian, apply the Lemma. If the resulting subgroup is abelian, part 1 finishes things off. Otherwise repeat. If we never reached an abelian subgroup, then we'd have an infinite sequence of proper subgroups and thus a decreasing sequence of positive integers; contradiction.\qedhere
	\end{enumerate}
\end{proof}


Haven't we done this already?! Exercise \ref*{sec:direct}.\ref{exs:abeliansubgroup} appears to cover abelian groups, but this depends on the Fundamental Theorem (\ref{thm:fund}) whose proof requires Cauchy for abelian groups\ldots! Indeed Cauchy's Theorem may be extended to prove that if $p^k$ divides $G$, then $G$ has a subgroup of order $p^k$. This is the beginning of the Sylow theory of $p$-subgroups which has applications to group classification and the existence of sequences of normal subgroups. 


\goodbreak

\begin{exercises}{}{}
	Key concepts:
	\begin{quote}
		\emph{Orbits of $G$ partition $X$\qquad Cardinality of orbit $\nm{Gx}=(G:\Stab(x))$ divides $\nm G$\\
	Burnside's formula for counting number of orbits}
	\end{quote}
	
	\begin{enumerate}
	  \item Determine the orbits of $G=\ip\sigma$ on $X=\{1,2,3,4,5,6\}$ for each of Exercises \ref*{sec:action1}.\ref{exs:orbitsigma1} and \ref{exs:orbitsigma2}. In both cases verify Burnside's formula.
	  
	  
		\item Revisit Example \ref{ex:toy}. How may distinct toys may be created if:
		\begin{enumerate}
		  \item A maximum of two colors can be used?
		  \item Exactly two colors must be used?
		\end{enumerate}
		
	  
	  \item Prove Lemma \ref{lemm:actionorbit}: the orbits of a left action partition $X$.
	  
		
		\item A 10-sided die is shaped so that all faces are congruent \emph{kites}: five faces are arranged around the north pole and five around the south, so that each face is adjacent to four others.
		\begin{enumerate}
		  \item Argue that the group of rotational symmetries of such a die has ten elements.\par
		  (\emph{In fact it is non-abelian and is therefore isomorphic to $D_5$}).
		  \item Use Burnside's formula to determine how many distinct 10-sided dice may be produced.
		\end{enumerate}
		
	
	
	
	
		\begin{minipage}[t]{0.74\linewidth}\vspace{-6pt}
			\item A soccer ball is constructed from 20 regular hexagons and 12 regular pentagons as in the picture.\par
			Suppose the 20 hexagonal patches are all to have different colors, as are the 12 pentagonal patches. How many distinct balls may be produced?
			
			\item The faces of a cuboid measuring $1\times 1\times 2$\,in is to be painted using (at most) two colors. Up to equivalence by rotations, how many ways can this be done? 
		\end{minipage}\hfill\begin{minipage}[t]{0.25\linewidth}\vspace{-25pt}
			\flushright\includegraphics[scale=0.15]{ball}
		\end{minipage}
		
		
			
	
		%is to be made into an extremely unfair die by painting the numbers 1 to 6, one on each face. If we assume that the orientation of the numbers on each face is irrelevant, how many distinct dice can we make?\\
	% To apply Burnside we need a set $X$ and group $G$ acting on $X$ such that the orbits of the action consist of indistinct dice. Thus let $X$ be the set of all possible labelings of the faces and $G$ be the group of rotations of the prism. There are clearly $6!$ possible ways to label the faces. The group of rotations is a little trickier. Consider one of the square faces. We can leave this face where it is by performing the identity of one of three rotations. Alternatively, we can swap this face with the other square face before rotating. The result is a rotation group\footnote{We don't need to know $G$ in terms of the standard lists, just its order. However, it isn't hard to see that the rotation group is isomorphic to $D_4$.} of order 8. learly the cyclic group of order 4, generated by a $90^\circ$ rotation. Every face of a labelled cuboid has a different number, thus any element of the rotation group changes every labelling, with the exception of the identity which changes nothing. Therefore
	% \[\Fix(g)=\begin{cases}
	%       X,&\text{if }g=e,\\ \emptyset,&\text{if }g\neq e.
	%       \end{cases}\]
	% It follows that the number of distinct dice is
	% \[\frac{1}{\nm G}\sum_{g\in G}\nm{\Fix(g)}=\frac{6!}{8}=90.\]
	% This question can also be answered very simply using combinatorics. There are $\binom{6}{2}$ choices for the pair of numbers to be painted on the square ends. Once these are chosen, there are 6 configurations of the remaining 4 numbers on the longer sides (up to rotation of the end squares). We therefore obtain $\binom{6}{2}\cdot 6=90$ distinct dice.
	
	
		\item Repeat the previous question for a regular tetrahedron.
		
		
		
		\item Suppose $G$ is a finite group with order $p^n$ where $p$ is a prime. If $x\in G$ lies in a conjugacy class with at least 2 elements, prove that the order of $\Stab(x)$ divides $p^{n-1}$. Now use the class equation to prove that $p$ divides the order of the center $Z(G)$.
	% 	If $Gx$ is an orbit with at least 2 elements, then $\nm{Gx}=\frac{\nm G}{\Stab(x)}=\frac{p^n}{\Stab (x)}\ge 2\implies \nm{\Stab(x)}\le \frac 12p^n<p^n$. Since $\Stab(x)$ is a subgroup of $G$ it must have order dividing $p^n$.
	% 	
	% 	To finish, observe that every index $(G:\Stab(x))$ in the class equation is divisible by $p$, as is $\nm G$. We conclude that $p$ divides $\nm{Z(G)}$.
	
		
		\item\label{exs:cauchysthmabelian} We prove the abelian part of Cauchy's Theorem by induction on the order $n=\nm G$.
		\begin{enumerate}
		  \item Explain why the base case $n=\nm G=2$ is true.
		  \item Fix $n\ge 3$ and assume $p$ divides the order $n$ of some abelian group $G$. For the induction hypothesis, assume that if $\nm H<n$ and $p\bigm|\nm{H}$, then $H$ has a subgroup of order $p$. %divides $\nm G\ge 3$ and assume the result holds for all abelian groups of order $<\nm G$.
		  \begin{itemize}
				\item Let $x\in G$ be a non-identity element with order $m=\nm{\ip x}$ (necessarily $m\ge 2$).
				\item Choose any prime $q$ dividing $m$, define $y:=x^{m/q}$ and let $H:=\ip y$.
			\end{itemize}
			\begin{enumerate}
			  \item What is the order of $H$? Explain why are we done if $q=p$.
		  	\item If $q\neq p$, use the induction hypothesis to explain why there exists a coset $zH\in \quotient GH$ of order $p$. Now prove that $z^q$ has order $p$ in $G$.
		  \end{enumerate}
		\end{enumerate}
		
		
	\end{enumerate}
\end{exercises}

\clearpage

\iffalse

\subsection{The Sylow Theorems (optional and non-examinable)}

These go some way toward establishing the existence and number of certain subgroups.

\begin{defn}{}{}
	Let $p$ be prime. $G$ is a \emph{$p$-group} if every element of $G$ has order a power of $p$.\\
A subgroup $H$ of a group $G$ is a \emph{$p$-subgroup of $G$} if it is a $p$-group in its own right.
\end{defn}

\begin{examples}{}{}
	\exstart $G=\Z_8$ is a $2$-group.
	\begin{enumerate}\setcounter{enumi}{1}
		\item $\Z_9$ is a 3-subgroup of $\Z_{18}$ (note that $\Z_{18}$ is not a $p$-group).
		\item $\Z_3\times\Z_3$ is a 3-subgroup of $\Z_3\times\Z_6$.
		\item $V$ is a 2-subgroup of $S_4$.
		\item $\Z_4\times\Z_4$ is a 2-subgroup of $\Z_4\times\Z_8$ (this last is \emph{also} a 2-group).
	\end{enumerate}
\end{examples}

Note that $G$ need not be a $p$-group in the second part of the definition, although certainly all subgroups of $p$-groups are themselves $p$-subgroups.

% For the next theorem we consider equation $(\ast)$ for the case when $X=G$ acts on itself by conjugation. In this case we have $X_G=\{g\in G:gxg^{-1}=x,\forall x\in G\}=Z(G)$, the \emph{center} of $G$ (Section \ref{sec:center}).
% 
% \begin{thm}{Cauchy}{}
% Suppose that $p$ is a prime that divides the order of a finite group $G$. Then $ G$ contains an element $x$ of order $p$ and thus a subgroup $\ip x\le G$ of order $p$ (indeed $\ip x\cong\Z_p$).
% \end{thm}
% 
% 
% \begin{example}{}{}
% $S_4\times \Z_{10}\times D_6$ has order $24\cdot 10\cdot 12=2^6\cdot 3^2\cdot 5$ and thus has subgroups of orders 2, 3 \& 5 by Cauchy's Theorem. It of course has subgroups of many other orders.
% \end{example}


\begin{defn}{}{normalizer}
	The stabilizer of $H\le G$ under conjugation is the \emph{normalizer} of $H$:
	\[
		G_H:=\Stab(H)=\{g\in G:gHg^{-1}=H\}
		=\{g\in G:gh g^{-1}\in H,\forall h\in H\}
	\]
\end{defn}


The normalizer of $H$ merely keeps $H$ together while perhaps permuting its elements. The following should be clear:
\begin{itemize}
  \item $H\triangleleft G\iff G_H=G$
  \item $H\triangleleft G_H$ and $G_H$ is the largest subgroup of $G$ having $H$ as a normal subgroup.
\end{itemize}


\begin{lemm}{}{pmod}
	Let $\nm G=p^k$ where $p$ is prime, and suppose $X$ is a finite set acted on by $G$. Then $\nm X\equiv\nm{X_G}\pmod p$.
\end{lemm}

\begin{proof}
	In the class equation, each $\frac{\nm G}{\nm{\Stab(x_j)}}$ is divisible by $p$.
\end{proof}


\begin{lemm}{}{pgroup}
	If $H$ is a $p$-subgroup of $G$, then $(G_H:H)\equiv (G:H)\pmod p$.
\end{lemm}

\begin{proof}
	Let $H$ act on the set of left cosets $\mathcal L$ of $H$ in $G$ by left multiplication: $h\cdot(gH)=(hg)H$. By definition, $\nm{\mathcal L}=(G:H)$.\par
	Let $\mathcal L_H=\{$left cosets of $H$ fixed by $H$\}: $gH\in\mathcal L_H\Longleftrightarrow \forall h\in H, \ hgH=gH$. Observe:
	\[
		gH\in\mathcal L_H\iff \forall h\in H,\ g^{-1}hg\in H\iff g\in G_H
	\]
	Thus $\nm{\mathcal L_H}=(G_H:H)$, the number of (left) cosets of $H$ in $G_H$.\smallbreak
	Since $H$ is a $p$-group, Lemma \ref{lemm:pmod} gives $\nm{\mathcal L}\equiv\nm{\mathcal L_H}\pmod p$ which is the result.
\end{proof}


\goodbreak


\begin{thm}{1st Sylow theorem}{}
	Let $\nm G=p^nm$ where $p$ is prime, $n\ge 1$ and $p\nmid m$. Then:
	\begin{enumerate}
		\item $G$ contains a subgroup of order $p^i$ for each $i=1\ldots,n$.
		\item Any such subgroup with $i<n$ is a normal subgroup of some subgroup of order $p^{i+1}$.
	\end{enumerate}
\end{thm}

\begin{proof}
	We argue by induction. Cauchy's Theorem is the base case: $G$ has a subgroup of order $p$.\smallbreak
	For the induction hypothesis, fix $i<n$ and suppose $G$ has a subgroup $H$ of order $p^i$. Plainly 
	\[
		(G:H)=p^{n-i}m
	\]
	is divisible by $p$. Consider the factor group $\quotient{G_H}H$ (order $(G_H:H)$). Since $H$ is a $p$-group, Lemma \ref{lemm:pgroup} says that $p$ divides the order of $\quotient{G_H}H$; Cauchy says there is a subgroup $K\le \quotient{G_H}H$ of order $p$.\par
	Let $\gamma:G_H\to\quotient{G_H}H$ be the canonical homomorphism, then the inverse image
	\[
		\gamma^{-1}(K)=\bigl\{g\in G_H:\gamma(g)\in K\bigr\}
	\]
	is a subgroup of $G_H$ and thus of $G$. Clearly $H$ is normal in $\gamma^{-1}(K)\le G_H$; by the first Isomorphism Theorem,
	\[
		\quotient{\gamma^{-1}(K)}H\cong\gamma\bigl(\gamma^{-1}(K)\bigr)=K
	\]
	Since $K$ has order $p$, and $H$ has order $p^i$, it is clear that $\gamma^{-1}(K)$ has order $p^{i+1}$. We are done.
\end{proof}

The first Sylow theorem says that given $G$ of order $p^nm$ there exists a chain of subgroups $H_i$ of order $p^i$ which are normal inside each other:
\[
	\{e\}\triangleleft H_1\triangleleft H_2\triangleleft\cdots\triangleleft H_n\le G
\]
Only the final subgroup inclusion need not be normal. The third Sylow theorem gives a method whereby you can often tell if the final inclusion is normal.

\begin{defn}{}{}
	A \emph{Sylow $p$-subgroup} of a group $G$ is a $p$-subgroup of maximal size (not contained in any larger $p$-subgroup).
\end{defn}

When $\nm G=p^nm$ as in the first theorem, the Sylow $p$-subgroups of $G$ constitute all subgroups of order $p^n$. Moreover, any $p$-subgroup of $G$ is necessarily contained in a Sylow $p$-subgroup of $G$ by the second part of the first theorem.

\begin{examples}{}{}
	\exstart Let $G$ be a group of order $100=2^2\cdot 5^2$. Then $G$ has at least one Sylow 5-subgroup of order 25, and at least one Sylow 2-subgroup of order 4.
	\begin{enumerate}\setcounter{enumi}{1}
		\item Let $G$ be a group of order $1800=2^3\cdot 3^2\cdot 5^2$. Thus $G$ has at least one Sylow 2-subgroup of order 8, at least one Sylow 3-subgroup of order 9, and at least one Sylow 5-subgroup of order 25.
	\end{enumerate}
\end{examples}

The second and third Sylow theorems give us information about how Sylow $p$-subgroups are related, and how many there might be in a given group.

\begin{thm}{Second Sylow theorem}{}
	If $G$ is a finite group, then any two Sylow $p$-subgroups are conjugate.
\end{thm}

\begin{proof}
	Suppose $P_1$ and $P_2$ are two Sylow $p$-subgroups of $G$. Since all such have the same order, we see that the groups are conjugate (by some $x\in G$), if and only if
	\begin{align*}
		\forall y\in P_1,\ x^{-1}yx\in P_2 &\iff \forall y\in P_1,\ yxP_2=xP_2\\
		&\iff xP_2\in\mathcal L_{P_1}
	\end{align*}
	That is, the left coset $xP_2$ is fixed by the left action of $P_1$. It suffices to show that $\mathcal L_{P_1}$ is non-empty.\par
	By Lemma \ref{lemm:pgroup}, $\nm{\mathcal L_{P_1}}\equiv\nm{\mathcal L}=(G:P_2)\pmod p$, where $\mathcal L$ is the set of left cosets of $P_2$ in $G$.\par
	Since $(G:P_2)$ is not divisible by $p$, we have $\nm{\mathcal L_{P_1}}\neq 0$, as required.
\end{proof}

\begin{thm}{Third Sylow theorem}{}
	The number of Sylow $p$-subgroups of $G$ is congruent to 1 modulo $p$ and divides $\nm G$.
\end{thm}

\begin{proof}
	\def\SSS{\mathcal{S}}
	Let $P$ be a Sylow $p$-subgroup of $G$, and $\SSS$ the set of all such. Consider two actions on $\SSS$.
	\begin{enumerate}
	  \item $G$ acts on $\SSS$ by conjugation: If $g\in G$ and $Q\in\SSS$, then $xQx^{-1}$ is another Sylow $p$-subgroup. By the second Sylow Theorem, there is only one orbit of $\SSS$ under $G$. But then
		\[
		\nm{\SSS}=\nm{\text{orbit $GP$ of $P$}}=(G:G_P)=\frac{\nm{G}}{\nm{G_P}}
		\]
		This is clearly a divisor of $\nm G$ and so the number of Sylow $p$-subgroups divides the order $G$.
		\item	The Sylow $p$-subgroup $P$ also acts on $\SSS$ by conjugation.\footnotemark{} Since $P$ is a $p$-group acting on a set $\SSS$, Lemma \ref{lemm:pgroup} implies that
		\[
			\nm{\SSS}\equiv\nm{\SSS_P}\pmod p
		\]
		That is, the number of $p$-subgroups is congruent modulo $p$ to the cardinality of the set of $p$-subgroups fixed by the action of $P$.\par
		Suppose $Q\in \SSS_P$. Then, for any $x\in P$ we have $xQx^{-1}=Q$, from which 
		\[
			P\le G_Q \tag{$P$ is a subgroup of the \emph{normalizer} of $Q$}
		\]
		However, we also have $Q\le G_Q\le G$. Since $P,Q$ are Sylow $p$-subgroups of $G$, it follows that they are also Sylow $p$-subgroups of $G_Q$. By the second Sylow theorem, $P,Q$ are conjugate in $G_Q$: that is, $Q=yPy^{-1}$ for some $y\in G_Q$. But $Q$ is a normal subgroup of $G_Q$, whence $Q=P$. We conclude that $P$ is the only Sylow $p$-subgroup fixed by the action of $P$ (i.e., $\SSS_P=\{P\}$), whence
	\[
		\nm{\SSS}\equiv 1\pmod p\tag*{\qedhere}
	\]
	\end{enumerate}
\end{proof}

\footnotetext{Since $P$ is a subgroup of $G$, there need not be only one orbit. Indeed, as the proof shows, the only way there can be one orbit is if $P$ is the unique Sylow $p$-subgroup of $G$.}

\goodbreak

\begin{examples}{}{}
	\exstart Suppose $\nm G=100$, which has divisors 1, 2, 4, 5, 10, 20, 25, 50 \& 100.
	\begin{enumerate}\setcounter{enumi}{1}
		\item[]\begin{itemize}
			\item The only divisor congruent to 1 modulo 5 is 1 itself. By the third theorem, there is precisely one Sylow 5-subgroup of $G$. Since this is the only subgroup of order 25, it must be self-conjugate and thus normal.
			\item	The divisors 1, 5 and 25 are congruent to 1 modulo 2, hence there might be 1, 5 or 25 distinct Sylow 2-subgroups of $G$.
		\end{itemize}
		We now give two sub-examples where the Sylow $p$-subgroups can be seen explicitly.
		\begin{enumerate}
			\item $\Z_{100}$ has one Sylow 5-subgroup $\ip{4}\cong\Z_{25}$, and one Sylow 2-subgroup $\ip{25}\cong\Z_4$.
			\item $D_{50}$ has one Sylow 5-subgroup of order 25 consisting of half of the 50 rotations:
			\[
				\ip{\rho_2}=\{\rho_0,\rho_2,\ldots,\rho_{48}\}\cong\Z_{25}
			\]
			There are 25 distinct Sylow-2 subgroups, each of which is isomorphic to the Klein 4-group $V$. These may be described explicitly if we label the reflections $\mu_1,\ldots,\mu_{50}$:
			\[
				V_i=\{\rho_0,\rho_{25},\mu_i,\mu_{i+25}\},\quad i=1,\ldots,25
			\]
			Note that $D_{50}$ has no subgroups isomorphic to $\Z_4$ (reflections have order 2 and rotations have orders 1, 5 or 25). By the second Sylow theorem all Sylow 2-subgroups are conjugate and thus isomorphic, so we need not look for $\Z_4$ subgroups regardless.
		\end{enumerate}
	
		\item Suppose $\nm G=20=2^2\cdot 5$. The divisors of $\nm G$ are 1, 2, 4, 5, 10 and 20. The must be either one or five Sylow 2-subgroups; and one Sylow 5-subgroup of $G$, which is necessarily normal. Thus $G$ has a normal subgroup isomorphic to $\Z_5$. A similar analysis to the above will show what these are when $G$ is either $\Z_{20}$ or $D_{10}$.
	
		\item Let $\nm G=225=3^2\cdot 5^2$. The divisors of $\nm G$ are 1, 3, 5, 9, 15, 25, 45, 75, 225. Thus $G$ has either one or 25 distinct Sylow 3-subgroups, and only one Sylow 5-subgroup (which is thus normal).
	
		\item Let $G$ have order $pq$, where $p<q$ are distinct primes $\ge 3$. Then all proper subgroups of $G$ are cyclic (isomorphic to $\Z_p$ or $\Z_q$). Moreover, $\Z_q$ is a Sylow $q$-subgroup of $G$. The third Sylow theorem states that there can only be one, $p$ or $q$ of these, and that this number is congruent to 1 modulo $q$. If there were $p$ such subgroups, then
		\[
			p\equiv 1\pmod q\implies p=1+\lambda q \tag{for some $\lambda\in\N$}
		\]
		which contradicts $p<q$. We conclude that there is only one Sylow $q$-subgroup of $G$.\smallbreak
		The other argument is false. For example if $p=3$ and $q=7$, then $q\equiv 1\pmod p$, so there may be one or seven Sylow 3-subgroups of $G$. The former situation in fact yields $\Z_{21}$, while the latter results in a new (non-abelian) group, not seen in these notes.\smallbreak
		In general it can be shown that if $G$ has order $pq$, where $p<q$ are any primes such that $q-1$ is not divisible by $p$, then $G\cong\Z_{pq}$. For example, the groups $\Z_{15}$, $\Z_{33}$, $\Z_{35}$, etc., are the only groups up to isomorphism of their given order.
	\end{enumerate}
\end{examples}

\section*{Homogeneous spaces --- for interest only}

Homogeneous spaces are an application of group actions to geometry and Physics and have extremely important applications. We now know enough to be able to give a few basic examples.

A homogeneous space is, loosely, a continuous set which looks the same around every point. For example an ant sitting on the surface of a sphere cannot distinguish between directions: every direction looks the same to him. Here is a more precise definition in terms of group actions.

\begin{defn}{}{}
Let $G\times X\to X$ be a transitive action of a continuous group (for example a matrix group) on a set $X$. Then we call $X$ a \emph{homogeneous space} (or sometimes a homogeneous $G$-space) and we write $X=G/H$ where $H$ is isomorphic to the stabilizer $G_x$ for some (indeed any) $x\in X$.
\end{defn}

$G/H$ is \emph{not} a factor group: it is the set of left cosets of $H$ in $G$. Observe that $\#(G/H)=(G:G_x)=\# Gx=\nm X$ since the action is transitive. The definition doesn't depend on the choice of $x$ because all isotropy subgroups are isomorphic in $G$ when the action is transitive:
\[y=g\cdot x\implies G_y=gG_xg^{-1}\cong G_x.\]

Homogeneous spaces are extremely useful in geometry as they allow us to describe and calculate outside of vector spaces.

\begin{examples}{}{}
\exstart The orthogonal group $\rO_3(\R)$ acts transitively on the sphere $S^2\subset\R^3$. To see this, suppose that $\V v\neq\V w$ are unit vectors, then there exists a rotation about the vector $\V v\times\V w$ taking $\V v$ to $\V w$: but any rotation is an element of $\rO_3(\R)$. Moreover $\rO_3(\R)$ preserves the lengths of vectors, so it certainly preserves the sphere.

Now consider the isotropy group of the north pole $(0,0,1)\in S^2$. If $A\in\rO_3(\R)$ fixes the north pole then, by orthogonality, it must map the perpendicular (equatorial) plane to itself: i.e.\ $A\ip{\V i,\V j}=\ip{\V i,\V j}$ preserves the span of $\V i,\V j$. Since $A$ is orthogonal, it must act as an element of the orthogonal group of 1 dimension lower on the plane $\ip{\V i,\V j}$. Thus the isotropy group at the north pole is isomorphic to $\rO_2(\R)$. This is similar to an example constructed earlier.

In particular we may write
\[S^2=\rO_3(\R)/\rO_2(\R).\]
One may therefore think of maps into the sphere in terms of maps into the coset space $\rO_3(\R)/\rO_2(\R)$. This can be extremely useful when it comes to differentiating maps into the sphere, for one can use Lie theory to do calculus on the sphere without having to rely on the surrounding vector space structure.\\

The discussion can be generalized to the $n$-sphere: $S^n=\rO_{n+1}(\R)/\rO_n(\R)$.
\begin{enumerate}\setcounter{enumi}{1}
\item Projective space is an example of a homogeneous space where there is no sensible vector space in which to work. Geometry in projective space is the geometry of perspective: very important for engineers and designers of 3d-computer graphics, amongst other things. Define
\[\pr(\R^3)=\{\text{1-dimensional vector subspaces of }\R^3\}.\]
Just like for the sphere, the orthogonal group $\rO_3(\R)$ acts transitively on the set of lines through the origin. Similarly the stabilizer of a line $\ell$ contains the orthogonal group $\rO_2(\R)$ acting on the plane perpendicular to $\ell$. Moreover it also contains the orthogonal maps sending $\ell$ to itself, namely $\pm I$ on $\ell$. This is a copy of $\rO_1(\R)$. The isotropy subgroup of a line $\ell$ is therefore isomorphic to $\rO_2(\R)\times\rO_1(\R)$, and so
\[\pr(\R^3)=\rO_3(\R)/(\rO_2(\R)\times\rO_1(\R)).\]

Just as for the sphere, the above description can be generalized to cover the set of $k$-dimensional subspaces of $\R^n$ --- the so-called \emph{Grassmannian} $G_k(\R^n)$. We have
\[G_k(\R^n)=\rO_n(\R)/(\rO_k(\R)\times\rO_{n-k}(\R)).\]

Once again this description leads to the ability to do calculus in a set that is not a vector space --- there are many, many applications.
\end{enumerate}
\end{examples}

\fi