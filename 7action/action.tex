\graphicspath{{4perm/asy/},{7action/asy/}}
\setcounter{section}{6}

\section{Group Actions}\label{chap:action}

\subsection{Group Actions, Fixed Sets and Isotropy Subgroups}\label{sec:action1}

You may feel by now that groups are worthy of study purely in their own right---if so great! For many, however, the fundamental reason to care about groups is because of how they \emph{transform sets.} Recall how the symmetric group $S_n$ (Section \ref{chap:perm}) was defined in terms of what its elements do to the set $\{1,\ldots,n\}$. This is an example of a general situation.

\begin{defn}{}{action}
	An \emph{action\footnotemark} of a group $G$ on a set $X$ is a function $\cdot:G\times X\to X$ for which,
	\begin{enumeratea}
	  \item $\forall x\in X,\ e\cdot x=x$,\quad and,
	  \item $\forall x\in X,\ g,h\in G,\  g\cdot (h\cdot x)=(gh)\cdot x$.
	\end{enumeratea}
\end{defn}

\footnotetext{%
	This is strictly a \emph{left} action. There is an analogous definition of a \emph{right action.} In these notes, all actions will be left.%
}

Part (b) says $g\mapsto g\cdot$ is a homomorphism of \emph{binary structures} $(G,\cdot)\to\big(\{f:X\to X\},\circ\big)$.

\begin{examples}{}{actioneasy}
	\exstart The symmetric $S_n$ group acts on $X=\{1,2,\ldots,n\}$. As a sanity check:\vspace{-3pt}
	\begin{enumerate}\setcounter{enumi}{1}
	  \item[]\begin{enumerate}
			\item $e(x)=x$ for all $x\in \{1,\ldots,n\}$.
			\item $\sigma\bigl(\tau(x)\bigr)=(\sigma\tau)(x)$ is simply composition of functions!
		\end{enumerate}
	  
	  \item If $X$ is the set of orientations of a regular $n$-gon centered at the origin and with one vertex at $(1,0)$, then $D_n$ acts on $X$ by rotations and reflections. This is essentially our definition of $D_n$!
		
	  \item Any group $G$ acts on itself by left multiplication ($X=G$). This is essentially the proof of Cayley's Theorem (\ref{thm:cayley}). $G$ also acts on itself by conjugation ($c_g\circ c_h=c_{gh}$ is Theorem \ref{thm:aut}).
	  
	  \item Matrix groups act on vector spaces by matrix multiplication. For example the orthogonal group $\rO_2(\R)$ transforms vectors via rotations and reflections:
		\[
			\rO_2(\R)\times\R^2\to\R^2:(A,\vv)\mapsto A\vv
		\]
	  
	  \item\label{ex:orthmultaction} A group can act on many different sets. Here are three further actions of the orthogonal group:
		\begin{enumerate}
	  	\item[i.] $\rO_2(\R)$ acts on $X=\{1,-1\}$ via $A\cdot x:=(\det A)x$.
			\item[ii.] $\rO_2(\R)$ acts on $X=\R^3$ via $A\cdot\vv:=A(v_1\vi+v_2\vj)+v_3\vk$.
			\item[iii.] $\rO_2(\R)$ acts on the unit circle $X=S^1\subseteq\R^2$ via matrix multiplication $A\cdot\vv:=A\vv$.
		\end{enumerate}
	\end{enumerate}
\end{examples}

We often use actions to help visualize a group (or even define it!); in this context, some actions are better than others. Consider the three actions of $\rO_2(\R)$ in part \ref*{ex:orthmultaction} above.
\begin{enumerate}\itemsep2pt
  \item[i.] $X=\{1,-1\}$ is too small to provide a detailed picture of the group since many matrices act in the same way: e.g.{} $\begin{smatrix}1&0\\0&1\end{smatrix}$ and $\begin{smatrix}-1&0\\0&-1\end{smatrix}$ both ``multiply by 1 ($=\det A$)''.
	\item[ii.] $X=\R^3$ is unnecessarily large. The action leaves any vertical vector untouched.
	\item[iii.] $X=S^1$ is large enough that the action of distinct matrices can be distinguished without being inefficiently large (a \emph{Goldilocks} action, perhaps?).
\end{enumerate}


\pagebreak[3]


These notions can be formalized.

\begin{defn}{}{}
	Let $G\times X\to X$ be an action.
	\begin{enumerate}
	  \item The \emph{fixed set} of $g\in G$ is the set (subset of $X$)
	  \[
	  	\Fix(g):=\{x\in X:g\cdot x=x\} \tag{also written $X_g$}
	  \]
	  \item The \emph{isotropy subgroup} or \emph{stabilizer} of $x\in X$ is the set (subset of $G$)
	  \[
	  	\Stab(x):=\{g\in G:g\cdot x=x\} \tag{also written $G_x$}
	  \] 
	  \item The action is \emph{faithful} (or \emph{effective}) if the only element of $G$ fixing everything is $e$. Equivalently:
	  \begin{enumerate}
	    \item $\Fix(g)=X\iff g=e$\qquad\qquad\qquad (b)\lstsp$\bigcap\limits_{x\in X}\Stab(x)=\{e\}$
	  \end{enumerate}
	  \item The action is \emph{transitive} if any element of $X$ may be transformed to any other:
		\[
			\forall x,y\in X,\ \exists g\in G\ \text{ such that }\  y=g\cdot x
		\]
	\end{enumerate}
\end{defn}

Very loosely, an action that is both faithful and transitive is likely reasonable for visualizing a group.

\begin{examples*}{\ref{ex:actioneasy} cont}{}
	\exstart The action of $S_n$ on $\{1,2,\ldots,n\}$ is both faithful and transitive:
	\begin{enumerate}\setcounter{enumi}{1}
	  \item[]\begin{description}
	  	\item[\normalfont\emph{Faithful}:] if $\sigma(x)=x$ for all $x\in\{1,2,\ldots,n\}$, then $\sigma=e$.
	  	\item[\normalfont\emph{Transitive}:] if $x\neq y$, then the 2-cycle $(x\,y)$ maps $x\mapsto y$.
	  \end{description}
	  
	  \item $D_n$ acts faithfully and transitively on the orientations of the $n$-gon.
	  
	  \item The action of a group on itself by left multiplication is both faithful and transitive. Conjugation is more complex: in most situations it is neither.
	  
	  \item The action of $\rO_2(\R)$ on $\R^2$ is faithful but not transitive: for instance the zero vector cannot be transformed into any other vector so $\Stab(\V0)=\rO_2(\R)$.
	  
	  \item We leave these as exercises.
	%   \begin{enumerate}
	%     \item[i.] The action $A\cdot x=\det(A)x$ is transitive, but not faithful: indeed if $A$ is any orthogonal matrix with determinant 1, then $\Fix(A)=X$.
	%     \item[ii.] The action of $\rO_2(\R)$ on $\R^3$ is faithful but not transitive. It is impossible to transform, say, the zero vector into anything else.
	%     \item[iii.] The action of $\rO_2(\R)$ is both faithful and transitive.
	% 	\end{enumerate}
	\end{enumerate}
\end{examples*}


\begin{lemm}{}{}
	For each $x\in X$, the stabilizer $\Stab(x)$ is indeed a subgroup of $G$.
\end{lemm}

\begin{proof}
	We use the subgroup criterion:
	\begin{description}
		\item[\normalfont\emph{Non-emptiness}] Part (a) of Definition \ref{defn:action} says that $e\in \Stab(x)$.
		\item[\normalfont\emph{Closure}] This is part (b) of the Definition. Let $g,h\in\Stab(x)$, then
		\[
			(gh)\cdot x=g\cdot(h\cdot x)
			=g\cdot x=x\implies gh\in\Stab(x)
		\]
		\item[\normalfont\emph{Closure}] This relies on both parts of the Definition. If $g\in\Stab(x)$, then
		\begin{align*}
			x=g\cdot x
			&\implies g^{-1}\cdot x=g^{-1}\cdot (g\cdot x)
			=(g^{-1}g)\cdot x=e\cdot x=x\\
			&\implies g^{-1}\in \Stab(x)
			\tag*{\qedhere}
		\end{align*}
	\end{description}
\end{proof}


\pagebreak[3]


\begin{example}{}{actiond3}
	The dihedral group $D_3=\{e,\rho_1,\rho_2,\mu_1,\mu_2,\mu_3\}$ acts on the set $X$ of vertices of an equilateral triangle.\footnotemark\ The fixed sets and stabilizers for this action are as follows:\par
	\begin{minipage}[t]{0.74\linewidth}\vspace{-17pt}
		\[
			\begin{array}[t]{c|c}
				\text{Element $g$}&\Fix(g)\\\hline
				e&\{1,2,3\}\\
				\rho_1&\emptyset\\
				\rho_2&\emptyset\\
				\mu_1&\{1\}\\
				\mu_2&\{2\}\\
				\mu_3&\{3\}
			\end{array}
			\qquad
			\begin{array}[t]{c|c}
				\text{Vertex $x$}&\Stab(x)\\\hline
				1&\{e,\mu_1\}\\
				2&\{e,\mu_2\}\\
				3&\{e,\mu_3\}
			\end{array}
		\]
	\end{minipage}
	\hfill
	\begin{minipage}[t]{0.25\linewidth}\vspace{-17pt}
		\flushright\includegraphics[scale=0.95]{perm-d3}
	\end{minipage}\medbreak
	$D_3$ also acts on the set of \emph{edges} of the triangle $E =\big\{\{1,2\},\{1,3\},\{2,3\}\big\}$. You needn't write all these out since stabilizing an edge is equivalent to stabilizing its opposite vertex. Still, here is the data:
	\[
		\begin{array}[t]{c|c}
			\text{Element $g$}&\Fix(g)\\\hline
			e&E\\
			\rho_1&\emptyset\\
			\rho_2&\emptyset\\
			\mu_1&\big\{\{2,3\}\big\}\\
			\mu_2&\big\{\{1,3\}\big\}\\
			\mu_3&\big\{\{1,2\}\big\}
		\end{array}
		\qquad
		\begin{array}[t]{c|c}
			\text{Edge $\{x,y\}$}&\Stab(\{x,y\})\\\hline
			\{1,2\}&\{e,\mu_3\}\\
			\{1,3\}&\{e,\mu_2\}\\
			\{2,3\}&\{e,\mu_1\}
		\end{array}
	\]
\end{example}

\footnotetext{%
	Recall that $\rho_1$ rotates \ang{120} counter-clockwise, that $\rho_2=\rho_1^2$ and that $\mu_i$ reflects across the altitude through vertex $i$.
}

\begin{exercises}{}{}
	Key concepts:\quad \emph{(left) action\quad $\Fix(g)$\quad $\Stab(x)\le G$\quad faithful \& transitive actions}
	
	\begin{enumerate}
	  \item For part \ref*{ex:orthmultaction} of Example \ref{ex:actioneasy}, determine whether each action is faithful and/or transitive.
	
		
		\item\label{exs:orbitsigma1} Consider the cyclic subgroup $G=\ip\sigma$ of $S_6$ generated by the 6-cycle $\sigma=(1\,2\,3\,4\,5\,6)$.
		\begin{enumerate}
		  \item State the fixed sets and stabilizers for the natural action of $G$ on the set $X=\{1,2,3,4,5,6\}$.
		  \item Is the action of $G$ faithful? Transitive?
		\end{enumerate}
		
		
		\item\label{exs:orbitsigma2} Repeat the previous question when $\sigma=(1\,3)(2\,4\,6)$.
		
	
		\item Mimic Example \ref{ex:actiond3} for the actions of $D_4$ on $X=\{$vertices$\}$ and $E=\{$edges$\}$ of the square.\\
		(\emph{Use whatever notation you like; $\rho,\mu,\delta$ or cycle notation})
	
	
		\item Prove that \emph{left multiplication} of $G$ on itself $g\cdot x=gx$ is:
		\begin{enumerate}
		  \item \emph{Free}: $\textcolor{red}{\exists} x\in G,\ g\cdot x=x\Longrightarrow g=e$ \ (this is stronger than faithfulness [$\forall$ versus $\textcolor{red}{\exists}$]).
		  \item \emph{Transitive}.
		\end{enumerate}
			
		\item Suppose that $G$ acts on itself ($X=G$) by \emph{conjugation} $g\cdot x= gx g^{-1}$.
		\begin{enumerate}
		  \item Prove that conjugation is faithful if and only if the \emph{center} is trivial: $Z(G)=\{e\}$.
		  \item If $G$ is abelian, is conjugation faithful? Transitive? Explain.
		  \item If $G=S_n$, is conjugation faithful? Transitive? Explain.
		\end{enumerate}
	
	
		\item Suppose $G$ has a left action on $X$. Prove that $G$ acts faithfully on $X$ if and only if no two distinct elements of $G$ have the same action on every element.
	
	\end{enumerate}
\end{exercises}


\clearpage



\subsection{Orbits \& Burnside's Formula}

We first met orbits in the context of the symmetric groups $S_n$. The same idea applies to any action.

\begin{defn}{}{}
	Let $G\times X\to X$ be an action. The \emph{orbit} of $x\in X$ under $G$ is the set of elements into which $x$ may be transformed:
	\[
		Gx=\{g\cdot x:g\in G\}\subseteq X \tag{also written $G\cdot x$}
	\]
\end{defn}


\begin{examples}{}{}
	\exstart If $X=\{1,2,\ldots n\}$ and $G=\ip{\sigma}\le S_n$, then
	\[
		Gx=\{\sigma^k(x):k\in\Z\}=\orb_x(\sigma)
	\]
	The definition of orbits therefore coincides with that in Section \ref{sec:orbits}.
	\begin{enumerate}\setcounter{enumi}{1}
	  \item A transitive action\footnotemark{} has only one orbit.
	  
	  \item If $\rO_2(\R)$ acts on $\R^2$ by matrix multiplication, then the orbits are circles centered at the origin!
	\end{enumerate}
\end{examples}

\footnotetext{%
	We now have two meanings of \emph{transitive}; one for equivalence relations and one for actions. Be careful!%
}


\begin{lemm}{}{actionorbit}
	The orbits of an action partition $X$.
\end{lemm}

We omit the proof: compare the special case where $S_n$ acts on $X=\{1,\ldots,n\}$ (Lemma \ref{lemm:orbit}).\smallbreak

Our next result is analogous to Lemma \ref{lemm:homobij} where we counted the number of (left) cosets of $\ker\phi$.


\begin{lemm}{}{actionorbits2}
	The cardinality of the orbit $Gx$ is the index of the isotropy subgroup $\Stab(x)$:
	\[
		\nm{Gx}=\bigl(G:\Stab(x)\bigr)
	\]
	For a finite group $\nm G$, the size of the orbit necessarily divides the order of the group $\nm G$. 
\end{lemm}

\begin{proof}
	Observe that
	\begin{align*}
		g\cdot x=h\cdot x
		&\iff h^{-1}g\cdot x=x
		\iff h^{-1}g\in \Stab(x)\\
		&\iff g\Stab(x)=h\Stab(x)
	\end{align*}
	Otherwise said (contrapositive) distinct elements of $Gx$ correspond to distinct left cosets.
\end{proof}


\begin{example}{}{burnsimple}
	Let $\sigma=(1\,4)(2\,7\,3)\in S_7$. Consider $X=\{1,2,3,4,5,6,7\}$ under the action of the cyclic group $G=\ip{\sigma}$. The orbits are precisely the disjoint cycles: $\{1,4\}$, $\{2,3,7\}$, $\{5\}$, $\{6\}$. Observe that $G$ has six elements:
	\[
		e,\quad 
		\sigma=(1\,4)(2\,7\,3),\quad 
		\sigma^2=(2\,3\,7),\quad 
		\sigma^3=(1\,4),\quad 
		\sigma^4=(2\,7\,3),\quad 
		\sigma^5=(1\,4)(2\,3\,7)
		\]
	The Lemma is easily verifiable: for instance,
	\begin{gather*}
		\Stab(3)=\{\tau\in G:\tau(3)=3\}=\{\sigma^k:\sigma^k(3)=3\}=\{e,\sigma^3\}\\
		\implies \big(G:\Stab(3)\big)=\frac 62=3=\nm{\{2,3,7\}}=\nm{G3}
	\end{gather*}
\end{example}

\boldsubsubsection{Burnside's Formula}

It can be useful to count the \emph{number} of orbits of an action. For \emph{finite} actions, this can be done in two different ways, which leads to an interesting formula. We start by observing that
% \[
% 	S=\big\{(g,x)\in G\times X:g\cdot x=x\big\}
% 	\quad\text{has cardinality}\quad
% 	\nm S=\sum\limits_{x\in X}^{\phantom{a}}\nm{\Stab(x)}
% 	=\sum\limits_{g\in G}\nm{\Fix(g)}
% \]
\[
	S=\big\{(g,x)\in G\times X:g\cdot x=x\big\}
\]
has cardinality
\[
	\nm S=\sum\limits_{x\in X}\nm{\Stab(x)}
	=\sum\limits_{g\in G}\nm{\Fix(g)}
\]
We now count the elements of $S$ in a different way. Since $(G:\Stab(x))=\frac{\nm G}{\nm{\Stab(x)}}$, we see that
\[
	\frac{\nm S}{\nm G}=\sum_{x\in X}\frac{\nm{\Stab(x)}}{\nm G}
		=\sum_{x\in X}\frac 1{(G:\Stab(x))} 
		=\sum_{x\in X}\frac{1}{\nm{Gx}}
		\tag*{($\ast$)}
\]
where we used Lemma \ref{lemm:actionorbits2} for the last equality. Consider a fixed orbit $Gy$. Since $\nm{Gx}=\nm{Gy}$ is \emph{constant} for each $x\in Gy$, we conclude that
\[
	\sum_{x\in Gy}\frac{1}{\nm{Gx}}
	=\frac{\nm{Gy}}{\nm{Gy}}=1
\]
The summation ($\ast$) therefore counts 1 for \emph{each distinct orbit}. In concludsion:


\begin{thm}{Burnside's formula}{}
	Let $G$ be a finite group acting on a finite set $X$. Then
	\[
		\text{\# orbits}
		=\frac{1}{\nm G}\sum_{x\in X}\nm{\Stab(x)}
		=\frac{1}{\nm G}\sum_{g\in G}\nm{\Fix(g)}
	\]
\end{thm}

% \begin{proof}
% 	By Lemma \ref{lemm:actionorbits2},
% 	It follows that
% 	\[
% 		\frac 1{\nm G}\sum_{x\in X}\nm{\Stab(x)} 
% 		=\sum_{x\in X}\frac{\nm{\Stab(x)}}{\nm G} 
% 		=\sum_{x\in X}\frac 1{(G:\Stab(x))} 
% 		=\sum_{x\in X}\frac{1}{\nm{Gx}}
% 		\tag*{($\ast$)}
% 	\]
% 	Consider a fixed orbit $Gy$. Since $\nm{Gx}=\nm{Gy}$ for each $x\in Gy$, we see that
% 	\[
% 		\sum_{x\in Gy}\frac{1}{\nm{Gx}}
% 		=\frac{\nm{Gy}}{\nm{Gy}}=1
% 	\]
% 	The sum ($\ast$) therefore counts 1 for each distinct orbit in $X$ and therefore returns the number of orbits.\smallbreak
% 	For the second equality, observe that
% 	\[
% 		S=\{(g,x)\in G\times X:g\cdot x=x\}
% 	\]
% 	has cardinality
% 	\[
% 		\nm S=\sum\limits_{x\in X}\nm{\Stab(x)}
% 		=\sum\limits_{g\in G}\nm{\Fix(g)}
% 		\tag*{\qedhere}
% 	\]
% \end{proof}



\begin{example*}{\ref{ex:burnsimple} cont}{}
	Recall that $G=\ip\sigma=\ip{(1\,4)(2\,7\,3)}$ acting on $X=\{1,2,3,4,5,6,7\}$ has \emph{four} orbits. The stabilizers and fixed sets are as follows:
	\[
		\begin{array}{c|c}
			x\in X&\Stab(x)\\\hline
			1&\{e,\sigma^2,\sigma^4\}\\
			2&\{e,\sigma^3\}\\
			3&\{e,\sigma^3\}\\
			4&\{e,\sigma^2,\sigma^4\}\\
			5&G=\{e,\sigma,\sigma^2,\sigma^3,\sigma^4,\sigma^5\}\\
			6&G\\
			7&\{e,\sigma^3\}
		\end{array}
		\qquad\qquad
		\begin{array}{c|c}
			g\in G&\Fix(g)\\\hline
			e&X=\{1,2,3,4,5,6,7\}\\
			\sigma&\{5,6\}\\
			\sigma^2&\{1,4,5,6\}\\
			\sigma^3&\{2,3,5,6,7\}\\
			\sigma^4&\{1,4,5,6\}\\
			\sigma^5&\{5,6\}\\
			\multicolumn{2}{c}{}
		\end{array}
	\]
	Burnside's formula merely sums the cardinalities of all the subsets in the right column of each table:
	\begin{align*}
		4=\text{\# orbits}
		&=\frac{1}{\nm G}\sum_{x\in X}\nm{\Stab(x)}
			=\frac{1}{6}(3+2+2+3+6+6+2)\\
		&=\frac{1}{\nm G}\sum_{g\in G}\nm{\Fix(g)}
			=\frac{1}{6}(7+2+4+5+4+2)
	\end{align*}
\end{example*}



One reason to count the number of orbits of an action is that we often want to consider objects as equivalent if they differ by the action of some group.

\begin{example}{}{toy}
	A child's toy consists of a wooden equilateral triangle whose edges are painted using any choice of colors of the rainbow. How many distinct toys could we create?\smallbreak
	This might feel like an imprecisely-posed problem, but think for a moment like a child: wouldn't they likely consider the two pictured triangles below to be the same? The language of group actions can help make this precise.
	\begin{itemize}
	  \item If we orient each triangle the same way, then a single toy may be considered as an element of $X=\{$ordered color triples$\}$. Since there are 7 choices for the color of each edge, we see that $\nm X=7^3=343$ (a large set!).\par
		\begin{minipage}[t]{0.55\linewidth}\vspace{0pt}
		  \item Two toys are equivalent if they differ only by a rotation in 3-dimensions. This amounts to the natural action of $D_3$ on $X$: for instance, taking $\rho_1$ to rotate \ang{120} counter-clockwise,
		  \[
		  	\rho_1\cdot (\text{red,green,violet})
		  	=(\text{violet,red,green})
		  \]
		\end{minipage}
		\hfill
		\begin{minipage}[t]{0.4\linewidth}\vspace{0pt}
			\hfill\includegraphics{actions-triangles}
		\end{minipage}
	\end{itemize}
	The number of distinct toys is therefore the number of orbits of $D_3$ on $X$, which we may compute using Burnside. Since it would be time consuming to find the stabilizer of each element of $X$, we use the fixed set approach.
	\begin{description}
	  \item[\normalfont\emph{Identity} $e$:] For any action $\Fix(e)=X$.
	  \item[\normalfont\emph{Rotations} $\rho_1,\rho_2$:] If a color-scheme is fixed by $\rho_j$, then all pairs of adjacent edges must be the same color. Since there are seven colors in the rainbow, we see that $\nm{\Fix(\rho_i)}=7$.
	  \item[\normalfont\emph{Reflections} $\mu_1,\mu_2,\mu_3$:] Since $\mu_j$ swaps two edges, any color-scheme in its fixed set must have these edges the same color. We have 7 choices for the color of the switched edges, and an independent choice of 7 colors for the other edge. It follows that $\nm{\Fix(\mu_j)}=7^2=49$.
	\end{description}
	The number of distinct toys is therefore
	\begin{align*}
		\text{\# orbits}
		&=\frac 1{\nm{D_3}}\sum_{\sigma\in D_3}\nm{\Fix(\sigma)}
			=\frac 16(7^3 + \!\underbrace{7+7}_{\text{rotations}}\! + \,\underbrace{7^2+7^2+7^2}_{\text{reflections}})\\
		&=\frac 76(49+1+1+7+7+7)=84
	\end{align*}\smallskip
	
	For simpler version of the problem, consider the situation where all sides must be different colors. In this case $D_3$ acts on a set of color-schemes with cardinality $\nm Y=7\cdot 6\cdot 5=210$. Moreover, only the identity element has a non-empty fixed set. The number of distinct three-colored toys is therefore
	\[
		\text{\# orbits}
		=\frac 1{\nm{D_3}}\sum_{\sigma\in D_3}\nm{\Fix(g)}
		=\frac 16(210+0+\cdots+0)
		=\frac{210}6=35
	\]
	Of course you are welcome to answer questions like these using pure combinatorics without any resort to group theory!
\end{example}

\pagebreak[3]

\begin{example}{Dice-rolling for Geeks}{}
	Various games (such as Dungeons \& Dragons) make use of polyhedral dice beyond merely the cube (see, for instance, Exercise \ref*{sec:geomgroups}.\ref{exs:dnddice}).\medbreak
	Since dice are for rolling, we consider two such to be equivalent if one can be rotated into the other. It shouldn't be hard to convince yourself that the pictured tetrahedral dice are distinct (non-equivalent): the first \emph{cannot} be rotated to make the second.\par
	\begin{minipage}[t]{0.6\linewidth}\vspace{-5pt}
		Indeed, it is not difficult to see that these are the \emph{only} tetrahedral dice: if you place 4 on the table, then the remaining faces must be numbered 1, 2, 3 either \emph{counter-clockwise} or \emph{clockwise}.\medbreak
		For larger dice, such a counting approach is impractical. However, with a little thinking about symmetry groups, Burnside's formula will ride to the rescue.
		\end{minipage}
		\hfill
		\begin{minipage}[t]{0.39\linewidth}\vspace{-10pt}
		\flushright
			\href{https://www.math.uci.edu/~ndonalds/math120a/perm-a4.html}{\includegraphics[scale=0.1]{perm-a4.png}}
			\quad
		  \href{https://www.math.uci.edu/~ndonalds/math120a/actions-a4-2.html}{\includegraphics[scale=0.1]{actions-a4-2.png}}
	\end{minipage}
	\bigbreak

	Suppose a regular polyhedron has $f$ faces, each with $n$ sides.
	\begin{itemize}
	  \item The faces may be labelled 1 thorough $f$ in $f!$ distinct ways. Denote the set of distinct labellings by $X$.
	  \item We may rotate the polyhedron so that any face is mapped to any other, \emph{in any orientation.} For instance, the face labelled 1 may be rotated to any of the $f$ faces, before rotating the polyhedron around that face in any of $n$ orientations. The rotation group $G$ of the polyhedron therefore has $fn$ elements.
		\item Each non-identity element of the rotation group $G$ moves at least one face, whence
		\[
			\nm{\Fix(g)}=
			\begin{cases}
				X&\text{if }g=e\\
				\emptyset&\text{if }g\neq e
			\end{cases}
		\]
		\item By Burnside's formula, the number of distinct dice for a regular polyhedron is therefore
		\[
			\text{\# orbits}
			=\frac 1{\nm G}\nm{\Fix(e)}
			=\frac{\nm X}{\nm G}
			=\frac{f!}{fn}
			=\frac{(f-1)!}{n}
		\]
	\end{itemize}
	
	Here is the complete data for all the regular platonic solids.
	\begin{quote}
		\begin{tabular}{l||c|c|c|l}
			Polyhedron & $f$ & $n$ & Rotation Group & $\#$ distinct dice (orbits)\\\hline\hline
			Tetrahedron & 4 & 3 & $A_4$ & 2\\
			Cube & 6 & 4 & $S_4$ & 30\\
			Octahedron & 8 & 3 &  $S_4$ & 1,680\\
			Dodecahedron & 12 & 5 & $A_5$ & 7,983,360\\
			Icosahedron & 20 & 3 & $A_5$ & 40,548,366,802,944,000
		\end{tabular}
	\end{quote}
	Note that we don't need an explicit description of the rotation group, only its \emph{order} $\nm G$. We saw that the rotation group of regular tetrahedron was $A_4$ in Example \ref{ex:tetra}.\ref{ex:tetra2}. Proving that the remaining groups are as claimed is somewhat trickier\ldots
\end{example}


\pagebreak[3]


\begin{exercises}{}{}
	Key concepts:\qquad \emph{Orbits partition $X$\qquad $\nm{Gx}=(G:\Stab(x))$\qquad	Burnside's formula}
	
	\begin{enumerate}
	  \item Determine the orbits of $G=\ip\sigma$ on $X=\{1,2,3,4,5,6\}$ for each of Exercises \ref*{sec:action1}.\ref{exs:orbitsigma1} and \ref{exs:orbitsigma2}. In both cases verify Burnside's formula.
	  
	  
		\item Revisit Example \ref{ex:toy}. How may distinct toys may be created if:
		\begin{enumerate}
		  \item A maximum of two colors can be used?
		  \item Exactly two colors must be used?
		\end{enumerate}
		
	  
	  \item Prove Lemma \ref{lemm:actionorbit}: the orbits of a (left) action partition $X$.
	  
		
		\item A 10-sided die (see Exercise \ref*{sec:geomgroups}.\ref{exs:dnddice}) is shaped so that all faces are congruent \emph{kites}: five faces are arranged around the north pole and five around the south.
		\begin{enumerate}
		  \item Argue that the group of rotational symmetries of such a die has ten elements, and that it is isomorphic to $D_5$.
		  \item Use Burnside's formula to determine how many distinct 10-sided dice (faces numbered 1 to 10) may be produced.
		\end{enumerate}
		
	
	
	
	
		\begin{minipage}[t]{0.74\linewidth}\vspace{-6pt}
			\item A soccer ball is constructed from 20 regular hexagons and 12 regular pentagons as in the picture.\par
			Suppose the 20 hexagonal patches are all to have different colors, as are the 12 pentagonal patches. How many distinct balls may be produced?
			
			\item The faces of a cuboid measuring $1\times 1\times 2$\,in are to be painted using (at most) two colors. Up to equivalence by rotations, how many ways can this be done? 
		\end{minipage}\hfill\begin{minipage}[t]{0.25\linewidth}\vspace{-25pt}
			\flushright\includegraphics[scale=0.15]{ball}
		\end{minipage}
		
		
			
	
		%is to be made into an extremely unfair die by painting the numbers 1 to 6, one on each face. If we assume that the orientation of the numbers on each face is irrelevant, how many distinct dice can we make?\\
	% To apply Burnside we need a set $X$ and group $G$ acting on $X$ such that the orbits of the action consist of indistinct dice. Thus let $X$ be the set of all possible labelings of the faces and $G$ be the group of rotations of the prism. There are clearly $6!$ possible ways to label the faces. The group of rotations is a little trickier. Consider one of the square faces. We can leave this face where it is by performing the identity of one of three rotations. Alternatively, we can swap this face with the other square face before rotating. The result is a rotation group\footnote{We don't need to know $G$ in terms of the standard lists, just its order. However, it isn't hard to see that the rotation group is isomorphic to $D_4$.} of order 8. learly the cyclic group of order 4, generated by a $90^\circ$ rotation. Every face of a labelled cuboid has a different number, thus any element of the rotation group changes every labelling, with the exception of the identity which changes nothing. Therefore
	% \[\Fix(g)=\begin{cases}
	%       X,&\text{if }g=e,\\ \emptyset,&\text{if }g\neq e.
	%       \end{cases}\]
	% It follows that the number of distinct dice is
	% \[\frac{1}{\nm G}\sum_{g\in G}\nm{\Fix(g)}=\frac{6!}{8}=90.\]
	% This question can also be answered very simply using combinatorics. There are $\binom{6}{2}$ choices for the pair of numbers to be painted on the square ends. Once these are chosen, there are 6 configurations of the remaining 4 numbers on the longer sides (up to rotation of the end squares). We therefore obtain $\binom{6}{2}\cdot 6=90$ distinct dice.
	
	
		\item Repeat the previous question for a regular tetrahedron.
		
		
		\def\ca#1{\textcolor{red}{#1}}
		\def\cb#1{\textcolor{blue}{#1}}
		\def\cc#1{\textcolor{Green}{#1}}
		\def\cd#1{\textcolor{Magenta}{#1}}
		\begin{minipage}[t]{0.69\linewidth}\vspace{0pt}
			\item (Hard) A cube has four \emph{diagonals}: $\ca{a},\cb{b},\cc{c},\cd{d}$, which are permuted by its rotation group. For instance, we might rotate by \ang{180} around an axis through the midpoints of the sides $\cl{\ca{a_1}\cb{b_1}}$ and $\cl{\ca{a_2}\cb{b_2}}$. This switches diagonals $\ca{a}$ and $\cb{b}$, but leaves $\cc{c}$ and $\cd{d}$ alone and therefore acts as the 2-cycle $(\ca{a}\,\cb{b})$ on $\{\ca{a},\cb{b},\cc{c},\cd{d}\}$. As a function on vertices,
			\[
				(\ca{a}\,\cb{b}):
				\begin{pmatrix}
					\ca{a_1}&\cb{b_1}&\cc{c_1}&\cd{d_1}\\
					\ca{a_2}&\cb{b_2}&\cc{c_2}&\cd{d_2}
				\end{pmatrix}
				\mapsto
				\begin{pmatrix}
					\cb{b_1}&\ca{a_1}&\cc{c_2}&\cd{d_2}\\
					\cb{b_2}&\ca{a_2}&\cc{c_1}&\cd{d_1}
				\end{pmatrix}
			\]
		\end{minipage}
		\hfill
		\begin{minipage}[t]{0.3\linewidth}\vspace{0pt}
			\hfill\href{https://www.math.uci.edu/~ndonalds/math120a/actions-cube.html}{\includegraphics[scale=0.9]{actions-cube}}
		\end{minipage}
		
		
		\begin{enumerate}
		  \item Describe as best you can how to rotate the cube in a way that corresponds to each of the following rotations (written in cycle notation):
		  \[
		  	(\ca{a}\,\cb{b}\,\cc{c}),\quad (\ca{a}\,\cb{b}\,\cc{c}\,\cd{d}),\quad (\ca{a}\,\cb{b})(\cc{c}\,\cd{d})
		  \]
		  Hence argue that the rotation group of the cube is isomorphic to $S_4$.
		  \item A cube can operate as a ``3-sided'' die by labelling two each of its faces using the numbers 1, 2, 3. In how many distinct ways can this be done?
		\end{enumerate}
	\end{enumerate}
\end{exercises}

\clearpage


\subsection{The Class Equation, {\itshape p}-groups, and the Theorems of Cauchy and Sylow}\label{sec:pgroup}

The toolkit provided by group actions is central to an enormous topic: the classification of groups and their internal structure. This final section offers a small taste of this discussion.\medbreak

Suppose $G$ acts on a finite set $X$. Denote the set of 1-element orbits by
\[
 	X_G=\bigcap_{g\in G}\Fix(g)=\{x\in X:\forall g\in G, g\cdot x=x\}
\]
Let $x_1,\ldots,x_r$ be representatives of the remaining (larger) orbits. Since the orbits partition $X$,
\[
	\nm X=\nm{X_G}+\sum_{j=1}^r\nm{Gx_j} 
	=\nm{X_G}+\sum_{j=1}^r\bigl(G:\Stab(x_j)\bigr)
	=\nm{X_G}+ \sum_{j=1}^r \frac{\nm G}{\nm{\Stab(x_j)}} \tag{$\ast$}
\]
We focus on the special case when $G$ acts on itself ($X=G$) by \emph{conjugation} $g\cdot x=gx g^{-1}$.
\begin{itemize}\itemsep0pt
  \item The 1-element orbits comprise the group \emph{center}: $X_G=Z(G)$.
  \item In this context the stabilizer of an element $x\in G$ is known as its \emph{centralizer}:
	\[
		C(x)=\Stab(x)=\{g\in G:gx g^{-1}=x\} =\{g\in G:gx=xg\}
	\]
	\item Equation ($\ast$) is called the \emph{class equation}:
\[
	\tcbhighmath{\nm{G} =\nm{Z(G)}+\sum_{j=1}^r\nm{\text{conjugacy class of }x_j} =\nm{Z(G)}+\sum_{j=1}^r \frac{\nm G}{\nm{C(x_j)}}}
\]
\end{itemize}

\begin{example}{}{}
	The conjugacy classes in $S_4$ are the cycle-types, so the class equation is easily verified:
	\[
		24=\nm{\{e\}} +\#\big(\text{2-cycles}\big) +\#\big(\text{3-cycles}\big) +\#\big(\text{4-cycles}\big) +\#\big(\text{2,2-cycles}\big)
		=1+6+8+6+3
	\]
\end{example}

Our next result applies the class equation to obtain a partial converse to Lagrange's Theorem.

\begin{thm}{Cauchy}{}
	If a prime $p$ divides $\nm G$, then $G$ contains a subgroup/element of order $p$. 
\end{thm}


\begin{proof}
	\begin{enumerate}%\itemsep2pt
	  \item Exercise \ref{exs:cauchysthmabelian} supplies an inductive proof that abelian $G$ have such subgroups. 
	  \item If $G$ is non-abelian, then the center $Z(G)$ is a proper (abelian) subgroup.\par
	  Let $x\not\in Z(G)$. Plainly $C(x)$ is a proper subgroup of $G$. If $p$ does not divide $\nm{C(x)}$, then $p$ divides $\nm{Gx}=\bigl(G:C(x)\bigr)=\frac{\nm G}{\nm{C(x)}}$. If this holds for all non-trivial orbits, the class equation says that $\nm{Z(G)}$ is divisible by $p$. Either way, $G$ has a \textbf{proper subgroup} $H$ whose order is divisible by $p$.\par
	  If $H$ is abelian, apply case 1. Otherwise, repeat starting with $H$ to find an even smaller subgroup whose order is divisible by $p$. If this process never reached an abelian subgroup, then we'd have an infinite descending sequence of proper subgroups: contradiction.%\qedhere
	\end{enumerate}\medskip
	The resulting subgroup is cyclic (Corollary \ref{cor:orderpcyclic}), whence an element of order $p$ also exists.
\end{proof}

\begin{example}{}{}
	If $G$ has order $60=2^2\cdot 3\cdot 5$, then it has subgroups of orders 2, 3 and 5. It also has subgroups of other orders (at the very least 1, 60 and 4 --- this last by the 1\st{} Sylow Theorem below).
\end{example}



Haven't we done this already? Exercise \ref*{sec:direct}.\ref{exs:abeliansubgroup} appears to cover abelian groups, but this depends on the Fundamental Theorem (\ref{thm:fund}), the standard proof of which in fact relies on Cauchy (Exercise \ref{exs:fundthmproof})!


\begin{defn}{}{}
	Let $p$ be a prime. A group $G$ is a \emph{$p$-group} if all its elements have order a power of $p$.
\end{defn}

\begin{examples}{}{}
	\exstart $G=\Z_8$ is a 2-group.
	\begin{enumerate}\setcounter{enumi}{1}\itemsep2pt
		\item $\Z_9$ is a 3-subgroup of $\Z_{18}$ (this last is not a 3-group as it contains an element of order 2).
		\item $\Z_3\times\Z_3$ is a 3-subgroup of $\Z_3\times\Z_6$.
		\item $V=\big\{e,(1\,2)(3\,4),(1\,3)(2\,4),(1\,4)(2\,3)\big\}$ is a 2-subgroup of $S_4$.
		%\item $\Z_4\times\Z_4$ is a 2-subgroup of $\Z_4\times\Z_8$ (this last is also a 2-group).
	\end{enumerate}
\end{examples}


The proofs of the next result are exercises.

\begin{thm}{}{pgroupdefn}
	\exstart A finite group $G$ is a $p$-group if and only if $\nm G=p^n$ for some $n$.
	\begin{enumerate}\setcounter{enumi}{1}
	  \item If $G$ is a $p$-group, then its center $Z(G)$ is non-trivial.
	\end{enumerate}
\end{thm}

\begin{cor}{}{psquaredabelian}
	Let $p$ be prime. If $G$ has order $p^2$, then it is abelian.
\end{cor}

\begin{proof}
	By the Theorem, we know that $\nm{Z(G)}=p$ or $p^2$. In the second case we are done: $Z(G)=G$ says that $G$ is abelian. In the first case the factor group is cyclic:
	\[
		\nm{\quotient G{Z(G)}}=p\implies \quotient G{Z(G)}\cong \Z_p
	\]
	Exercise \ref*{sec:conj}.\ref{exs:centerquotientcyclic} says $G$ is abelian, though really it's a contradiction: $Z(G)=G\Longrightarrow \quotient G{Z(G)}\cong\Z_1$.
\end{proof}


Our final results go some way towards establishing the existence and number of certain subgroups. We omit the proofs: a typical approach uses induction with Cauchy's Theorem as the base case, and the application of various ingenious group actions. If you are interested, look them up!

\begin{thm}{Sylow}{}
	Let $G$ be finite and write $\nm G=p^nm$, where $p$ is prime and $\gcd(p,m)=1$.
	\begin{enumerate}\itemsep2pt
	  \item $G$ contains a subgroup $H_{p^i}$ of order $p^i$ for each $i=1,\ldots,n$. Moreover, any such subgroup with $i<n$ is a normal subgroup of some subgroup of order $p^{i+1}$:
	  \[
	  	\{e\}\triangleleft H_p\triangleleft H_{p^2}\triangleleft\cdots\triangleleft H_{p^n}\le G \tag{$\dag$}
	  \]
	  \item Any two maximal $p$-subgroups $H_{p^n}$ (called \emph{Sylow $p$-subgroups}) are conjugate.
	  \item The number of Sylow $p$-subgroups divides $\nm G$ and is congruent to 1 modulo $p$.
	\end{enumerate}
\end{thm}

Only the last inclusion in $(\dag)$ can be non-normal. For a given example, the 3\rd{} Theorem might show that there is a \emph{unique} Sylow $p$-subgroup $H_{p^n}$: by Exercise \ref*{sec:conj}.\ref{exs:conjsubgroup}, such a subgroup is necessarily normal.  

\pagebreak[3]

\begin{examples}{}{sylow}
	\exstart Suppose $\nm G=15=3\cdot 5$. By the 1\st{} Sylow Theorem, $G$ has at least one Sylow 3-subgroup (isomorphic to $\Z_3$) and at least one Sylow 5-subgroup (isomorphic to $\Z_5$). 
	\begin{enumerate}\setcounter{enumi}{1}
	  \begin{minipage}[t]{0.65\linewidth}\vspace{-5pt}
	  	\item[] The divisors of 15 are listed, along with whether each is congruent to 1 modulo 3 or 5. By the 2\nd{} and 3\rd{} Theorems, $G$ has exactly one subgroup isomorphic to $\Z_3$ and one to $\Z_5$: being self-conjugate, both subgroups are normal.
	  \end{minipage}
	  \hfill
	  \begin{minipage}[t]{0.32\linewidth}\vspace{-5pt}
	  	\hfill\begin{tabular}{@{}l||c@{\ \ }c@{\ \ }c@{\ \ }c@{}}
	  		Divisor $d$&1&3&5&15\\\hline
	  		$d\equiv 1\pmod 2$?&\cmark\\
	  		$d\equiv 1\pmod 3$?&\cmark\\
	  	\end{tabular}
	  \end{minipage}\smallbreak
	  By Lagrange, the orders of elements in $G$ can only be 1, 3, 5 or 15. Since an order 3 element generates a subgroup isomorphic to $\Z_3$ (the \emph{unique} Sylow 3-subgroup), only two elements in $G$ have order 3. Similarly only four elements in $G$ have order 5 (each being a generator of the unique Sylow 5-subgroup). Since only the identity has order 1, the remaining $8=15-1-2-4$ elements must have order 15, and thus generate $G$.\smallbreak
	  In conclusion: if $\nm G=15$, then $G\cong\Z_{15}$ is cyclic. Exercise \ref{exs:zpq} extends this analysis to when $\nm G=pq$, where $p<q$ are primes for which $q-1$ is not divisible by $p$: in such a case $G\cong\Z_{pq}$.

		\item\label{ex:sylow100} Suppose $\nm{G}=100=2^2\cdot 5^2$. By the 1\st{} Sylow Theorem, $G$ has at least one Sylow 5-subgroup of order $5^2=25$, and at least one Sylow 2-subgroup of order $4=2^2$. We can be more precise by applying the other results.\par
		The full list of divisors of $\nm G=100$ is: 1, 2, 4, 5, 10, 20, 25, 50, 100.
		\begin{itemize}
			\item The only divisor congruent to 1 modulo 5 is 1 itself! By the 3\rd{} Theorem, there is precisely one Sylow 5-subgroup $H_{25}$ of $G$. Since this is the only subgroup of order 25, it must be normal: $H_{25}\triangleleft G$. By Corollary \ref{cor:psquaredabelian}, $H_{25}$ is abelian. The Fundamental Theorem says that either $H_{25}\cong\Z_{25}$ or $H_{25}\cong\Z_5\times\Z_5$.
			\item	The divisors 1, 5 and 25 are congruent to 1 modulo 2, so there might be 1, 5 or 25 distinct Sylow 2-subgroups of $G$.
		\end{itemize}
		We give several sub-examples where the Sylow $p$-subgroups can be seen explicitly.
		\begin{enumerate}
			\item $G=\Z_{100}$ has one Sylow 5-subgroup $\ip{4}\cong\Z_{25}$, and one Sylow 2-subgroup $\ip{25}\cong\Z_4$.
			\item $G=\Z_{50}\times\Z_2$ has one Sylow 5-subgroup $\ip{(2,0)}\cong\Z_{25}$, and one Sylow 2-subgroup $\ip{25}\times\Z_2\cong\Z_2\times\Z_2$.
			\item $G=\Z_5\times\Z_{20}$ has one Sylow 5-subgroup $\Z_5\times\ip{4}\cong\Z_5\times\Z_5$, and one Sylow 2-subgroup $\ip{(0,5)}\cong\Z_4$.
			\item $G=D_{50}$ has one Sylow 5-subgroup consisting of half the 50 rotations:
			\[
				\ip{\rho_2}=\{\rho_0,\rho_2,\ldots,\rho_{48}\}\cong\Z_{25}
			\]
			There are 25 distinct Sylow 2-subgroups, each of which is isomorphic to the Klein 4-group $V$. These may be described explicitly if we label the reflections $\mu_1,\ldots,\mu_{50}$:
			\[
				V_i=\{\rho_0,\rho_{25},\mu_i,\mu_{i+25}\},\quad i=1,\ldots,25
			\]
			Note that $D_{50}$ has no subgroups isomorphic to $\Z_4$ (reflections have order 2 and rotations have orders 1, 5 or 25). This is also clear from the 2\nd{} Theorem: all Sylow 2-subgroups are conjugate and thus isomorphic (to $V$).
		\end{enumerate}
	\end{enumerate}
\end{examples}


\iffalse

\boldsubsubsection{Proofs of the Sylow Theorems}

\begin{lemm}{}{pgroup}
	If $H$ is a $p$-subgroup of $G$, then $(G_H:H)\equiv (G:H)\pmod p$.
\end{lemm}

\begin{proof}
	Let $H$ act on the set of its left cosets $\mathcal L$ by left multiplication: $h\cdot(gH)=(hg)H$. By definition, $\nm{\mathcal L}=(G:H)$.	Let $\mathcal L_H=\{$left cosets of $H$ fixed by $H$\}: $gH\in\mathcal L_H\Longleftrightarrow \forall h\in H, \ hgH=gH$. Observe:
	\[
		gH\in\mathcal L_H\iff \forall h\in H,\ g^{-1}hg\in H\iff g\in G_H
	\]
	Thus $\nm{\mathcal L_H}=(G_H:H)$, the number of (left) cosets of $H$ in $G_H$.\smallbreak
	Since $H$ is a $p$-group, Exercise \ref{exs:pgroupfinite} gives $\nm{\mathcal L}\equiv\nm{\mathcal L_H}\pmod p$ which is the result.
\end{proof}

\begin{defn}{}{normalizer}
	The stabilizer of $H\le G$ under conjugation is the \emph{normalizer} of $H$:
	\[
		G_H:=\Stab(H)=\{g\in G:gHg^{-1}=H\}
		=\{g\in G:gh g^{-1}\in H,\forall h\in H\}
	\]
\end{defn}

The normalizer of $H$ keeps $H$ together while perhaps permuting its elements. The following are clear:
\begin{itemize}
  \item $H\triangleleft G\iff G_H=G$
  \item $H\triangleleft G_H$ and $G_H$ is the largest subgroup of $G$ having $H$ as a normal subgroup.
\end{itemize}


\begin{proof}[Proof: 1\st{} Sylow Theorem]
	We argue by induction. Cauchy's Theorem is the base case.\smallbreak
	For the induction hypothesis, fix $i<n$ and suppose $G$ has a subgroup $H$ of order $p^i$. Plainly 
	\[
		(G:H)=p^{n-i}m
	\]
	is divisible by $p$. Consider the factor group $\quotient{G_H}H$ (order $(G_H:H)$). Since $H$ is a $p$-group, Lemma \ref{lemm:pgroup} says that $p$ divides the order of $\quotient{G_H}H$; Cauchy says it has a subgroup $K$ of order $p$.\par
	Let $\gamma:G_H\to\quotient{G_H}H$ be the canonical homomorphism. The inverse image
	\[
		\gamma^{-1}(K)=\bigl\{g\in G_H:\gamma(g)\in K\bigr\}
	\]
	is a subgroup of $G_H$ and thus of $G$. Clearly $H$ is normal in $\gamma^{-1}(K)\le G_H$. By the first Isomorphism Theorem,
	\[
		\quotient{\gamma^{-1}(K)}H\cong\gamma\bigl(\gamma^{-1}(K)\bigr)=K
	\]
	Since $K$ has order $p$, and $H$ has order $p^i$, it is clear that $\gamma^{-1}(K)$ has order $p^{i+1}$.
\end{proof}


\begin{proof}[Proof: 2\nd{} Sylow Theorem]
	Suppose $P_1$ and $P_2$ are two Sylow $p$-subgroups of $G$. Since all such have the same order, we see that the groups are conjugate (by some $x\in G$), if and only if
	\begin{align*}
		\forall y\in P_1,\ x^{-1}yx\in P_2 &\iff \forall y\in P_1,\ yxP_2=xP_2 \iff xP_2\in\mathcal L_{P_1}
	\end{align*}
	That is, the left coset $xP_2$ is fixed by the left action of $P_1$. It suffices to show that $\mathcal L_{P_1}$ is non-empty.\par
	By Lemma \ref{lemm:pgroup}, $\nm{\mathcal L_{P_1}}\equiv\nm{\mathcal L}=(G:P_2)\pmod p$, where $\mathcal L$ is the set of left cosets of $P_2$ in $G$.\par
	Since $(G:P_2)$ is not divisible by $p$, we have $\nm{\mathcal L_{P_1}}\neq 0$, as required.
\end{proof}


\begin{proof}[Proof: 3\rd{} Sylow Theorem]
	\def\SSS{\mathcal{S}}
	Let $P$ be a Sylow $p$-subgroup of $G$, and $\SSS$ the set of all such. Consider two actions on $\SSS$.
	\begin{itemize}
	  \item[I.] $G$ acts on $\SSS$ by conjugation: If $g\in G$ and $Q\in\SSS$, then $xQx^{-1}$ is another Sylow $p$-subgroup. By the second Sylow Theorem, there is only one orbit of $\SSS$ under $G$. But then
		\[
		\nm{\SSS}=\nm{\text{orbit $GP$ of $P$}}=(G:G_P)=\frac{\nm{G}}{\nm{G_P}}
		\]
		This is clearly a divisor of $\nm G$ and so the number of Sylow $p$-subgroups divides the order $G$.
		\item[II.]	The Sylow $p$-subgroup $P$ also acts on $\SSS$ by conjugation.\footnotemark{} Since $P$ is a $p$-group acting on a set $\SSS$, Lemma \ref{lemm:pgroup} implies that
		\[
			\nm{\SSS}\equiv\nm{\SSS_P}\pmod p
		\]
		That is, the number of $p$-subgroups is congruent modulo $p$ to the cardinality of the set of $p$-subgroups fixed by the action of $P$.\par
		Suppose $Q\in \SSS_P$. Then, for any $x\in P$ we have $xQx^{-1}=Q$, from which 
		\[
			P\le G_Q \tag{$P$ is a subgroup of the \emph{normalizer} of $Q$}
		\]
		However, we also have $Q\le G_Q\le G$. Since $P,Q$ are Sylow $p$-subgroups of $G$, it follows that they are also Sylow $p$-subgroups of $G_Q$. By the second Sylow theorem, $P,Q$ are conjugate in $G_Q$: that is, $Q=yPy^{-1}$ for some $y\in G_Q$. But $Q$ is a normal subgroup of $G_Q$, whence $Q=P$. We conclude that $P$ is the only Sylow $p$-subgroup fixed by the action of $P$ (i.e., $\SSS_P=\{P\}$), whence
	\[
		\nm{\SSS}\equiv 1\pmod p\tag*{\qedhere}
	\]
	\end{itemize}
\end{proof}

\footnotetext{Since $P$ is a subgroup of $G$, there need not be only one orbit. Indeed, as the proof shows, the only way there can be one orbit is if $P$ is the unique Sylow $p$-subgroup of $G$.}

\fi


\begin{exercises}{}
	\exstart State all the 3-subgroups of $\Z_{36}$.
	\begin{enumerate}\setcounter{enumi}{1}
	  \item Let $p$ be a prime of your choice. In contrast to Corollary \ref{cor:psquaredabelian}, give an example of a non-abelian group of order $p^3$.
	
	
		\item Suppose $\nm G=225=3^2\cdot 5^2$. Prove that $G$ has a normal subgroup of order 25.
	 
	 	
		\item Suppose $\nm G=20=2^2\cdot 5$.
		\begin{enumerate}
		  \item Determine how many distinct Sylow 2- and Sylow 5-subgroups $G$ can have. Must they be normal subgroups?
		  \item Similarly to Example \ref*{ex:sylow}.\ref{ex:sylow100}, state the Sylow subgroups when
		  \begin{enumerate}
		    \item $G=\Z_{20}$,\qquad\qquad ii. \ $G=D_{10}$
		  \end{enumerate}
		\end{enumerate}

	
		\item\label{exs:zpq} Let $G$ have order $pq$, where $p<q$ are distinct primes ($\ge 3$). Then all proper subgroups of $G$ are cyclic (isomorphic to $\Z_p$, $\Z_q$ or $\Z_1$).
		\begin{enumerate}
		  \item If $pq=2\cdot 3=6$, then $G\cong \Z_6$ or $\cong S_3$. State all the Sylow 2- and 3-subgroups in each case.
		  \item Returning to the general case, prove that there is a unique Sylow $q$-subgroup $H\cong\Z_q$ (which is therefore normal) by showing that $p\equiv 1\pmod q$ is a contradiction.
		  \item Suppose moreover that $q-1$ is not divisible by $p$. Prove that there is a unique Sylow $p$-subgroup $K\cong \Z_p$ (which is also normal). Extending the analysis of Example \ref{ex:sylow}.1, prove that $G\cong\Z_{pq}$.
		 	\item If $\nm G=21$ ($p=3,q=7$), show that there are either one or seven Sylow 3-subgroups. In the former case, argue that $G\cong\Z_{21}$.\par
		  (\textsf{The latter case [seven 3-subgroups] results in a new non-abelian group not seen in these notes.})
		\end{enumerate}

	  
	  \item Prove Theorem \ref{thm:pgroupdefn}, part 1.\par
	  (\emph{Hints: one direction uses Lagrange's Theorem, the other Cauchy})
	  
		\item\label{exs:pgroupfinite} Suppose $G$ is a $p$-group which acts on a finite set $X$.
		\begin{enumerate}
		  \item Prove that $\nm X\equiv \nm{X_G}\pmod p$.\par
		  (\emph{Hint: what can you say about $\nm{Gx}$ if $x$ lies in a non-trivial orbit?})
		  \item Prove part 2 or Theorem \ref{thm:pgroupdefn}, that $p$ divides the order of the center $Z(G)$.
		\end{enumerate}
		  
		  
		\item The alternating group $A_5$ has order $60=2^2\cdot 3\cdot 5$ and comprises the identity, all 3-cycles, 5-cycles, and 2,2-cycles in $S_5$.\par
		Describe all the Sylow 2-, 3-, and 5-subgroups of $A_5$ and verify that the number of each is in line with the 3\rd{} Theorem.\par
		(\textsf{You should find that there is more than one of each type of subgroup, whence none are normal. Indeed it can be shown that the only normal subgroups of $A_5$ are itself and $\{e\}$: we call this a \emph{simple group} [if you want a challenge try to prove this\ldots]. Simple groups are very important and have applications throughout mathematics. Their full classification was completed in 2004---an enormous undertaking: look it up\ldots})
		
		
		\pagebreak[3]


		\item\label{exs:cauchysthmabelian} We prove the abelian part of Cauchy's Theorem by induction on the order of $G$.
		\begin{enumerate}
		  \item Explain why the base case $\nm G=2$ is true.
		\end{enumerate}
		Fix $n\ge 3$, let $G$ be abelian of order $n$ and assume $p$ divides $n$. For the induction hypothesis, assume that if $K$ is \emph{any} abelian group of order $\nm K<n$ where $p$ divides $\nm K$, then $K$ has a subgroup of order $p$.
		 \begin{itemize}
				\item Let $x\in G$ be a non-identity element with order $m=\nm{\ip x}$ (necessarily $m\ge 2$).
				\item Choose a prime $q$ dividing $m$, define $y:=x^{m/q}$ and let $H:=\ip y$.
			\end{itemize}
			\begin{enumerate}\setcounter{enumii}{1}
			  \item What is the order of $H$? Explain why are we done if $q=p$.
		  	\item If $q\neq p$, use the induction hypothesis to explain why there exists a coset $zH\in \quotient GH$ of order $p$. Now prove that $z^q$ has order $p$ in $G$.
		  \end{enumerate}
		  
		  
		
% 		\item\label{exs:directprodabelian} Let $G$ be abelian. Suppose it has subgroups $H,K$ which satisfy
% 		\[
% 			G=HK=\{hk:h\in H,k\in K\}\quad\text{and}\quad H\cap K=\{e\}
% 		\]
% 		Prove that $H\times K\cong G$ via the isomorphism $\psi:(h,k)\mapsto hk$.
		
% 		\item We first check the $\psi$ is a homomorphism:
% 			\begin{align*}
% 				\psi\big((h_1,k_1)(h_2,k_2)\bigr) &=\psi(h_1h_2,k_1k_2)=h_1h_2k_1k_2 =h_1k_1h_2k_2 \tag{$G$ abelian}\\
% 				&=\psi(h_1,k_1)\psi(h_2,k_2)
% 			\end{align*}
% 			Now observe injectivity:
% 			\[
% 				\psi(h_1,k_1)=\psi(h_2,k_2)\implies h_1k_1=h_2k_2\implies h_2^{-1}h_1=k_1k_2^{-1}
% 			\]
% 			This last lies in $H\cap K$, and so must be the identity. We conclude that $(h_1,k_1)=(h_2,k_2)$.\par
% 			Finally, $\psi$ is surjective by definition.
		
		\item\label{exs:fundthmproof} We sketch part of the proof of the Fundamental Theorem of Finite Abelian Groups (\ref{thm:fund}).\smallbreak
		Suppose $G$ is abelian and that $\nm G=p^nm$ where $p$ does not divide $m$. Define
		\[
		 	H=\{x\in G:x^{p^n}=e\},\qquad K=\{x\in G:x^m=e\}
		\]
		\begin{enumerate}
		  \item Show that $H$ is a $p$-group.
		  \item Since $\gcd(m,p^n)=1$, we may write $1=\kappa m+\lambda p^n$ for some $\kappa,\lambda\in\Z$.
		  \begin{enumerate}
		    \item For any $x\in G$, prove that $x^{\kappa m}\in H$ and $x^{\lambda p^n}\in K$.
		    \item Prove that the following function is an isomorphism:
			  \[
			  	\psi:G\to H\times K:x\mapsto \big(x^{\kappa m},x^{\lambda p^n}\big)
			  \]
			  (\emph{Hint: try evaluating $\psi(hk)$\ldots}).
			\end{enumerate}
		  \item Prove that $\nm H=p^n$ by applying Cauchy's Theorem to show that $p$ does not divide $\nm K$.
		\end{enumerate}
		(\textsf{Inducting on this shows that $G\cong H_1\times\cdots \times H_k$ where each $H_j$ is a $p_j$-group of maximal order. A little more work is needed to show that each $H_j$ is itself a direct product of cyclic groups and thus complete the proof.})
	\end{enumerate}
\end{exercises}



\iffalse

\section*{Homogeneous spaces --- for interest only}

Homogeneous spaces are an application of group actions to geometry and Physics and have extremely important applications. We now know enough to be able to give a few basic examples.

A homogeneous space is, loosely, a continuous set which looks the same around every point. For example an ant sitting on the surface of a sphere cannot distinguish between directions: every direction looks the same to him. Here is a more precise definition in terms of group actions.

\begin{defn}{}{}
Let $G\times X\to X$ be a transitive action of a continuous group (for example a matrix group) on a set $X$. Then we call $X$ a \emph{homogeneous space} (or sometimes a homogeneous $G$-space) and we write $X=G/H$ where $H$ is isomorphic to the stabilizer $G_x$ for some (indeed any) $x\in X$.
\end{defn}

$G/H$ is \emph{not} a factor group: it is the set of left cosets of $H$ in $G$. Observe that $\#(G/H)=(G:G_x)=\# Gx=\nm X$ since the action is transitive. The definition doesn't depend on the choice of $x$ because all isotropy subgroups are isomorphic in $G$ when the action is transitive:
\[y=g\cdot x\implies G_y=gG_xg^{-1}\cong G_x.\]

Homogeneous spaces are extremely useful in geometry as they allow us to describe and calculate outside of vector spaces.

\begin{examples}{}{}
\exstart The orthogonal group $\rO_3(\R)$ acts transitively on the sphere $S^2\subset\R^3$. To see this, suppose that $\V v\neq\V w$ are unit vectors, then there exists a rotation about the vector $\V v\times\V w$ taking $\V v$ to $\V w$: but any rotation is an element of $\rO_3(\R)$. Moreover $\rO_3(\R)$ preserves the lengths of vectors, so it certainly preserves the sphere.

Now consider the isotropy group of the north pole $(0,0,1)\in S^2$. If $A\in\rO_3(\R)$ fixes the north pole then, by orthogonality, it must map the perpendicular (equatorial) plane to itself: i.e.\ $A\ip{\V i,\V j}=\ip{\V i,\V j}$ preserves the span of $\V i,\V j$. Since $A$ is orthogonal, it must act as an element of the orthogonal group of 1 dimension lower on the plane $\ip{\V i,\V j}$. Thus the isotropy group at the north pole is isomorphic to $\rO_2(\R)$. This is similar to an example constructed earlier.

In particular we may write
\[S^2=\rO_3(\R)/\rO_2(\R).\]
One may therefore think of maps into the sphere in terms of maps into the coset space $\rO_3(\R)/\rO_2(\R)$. This can be extremely useful when it comes to differentiating maps into the sphere, for one can use Lie theory to do calculus on the sphere without having to rely on the surrounding vector space structure.\\

The discussion can be generalized to the $n$-sphere: $S^n=\rO_{n+1}(\R)/\rO_n(\R)$.
\begin{enumerate}\setcounter{enumi}{1}
\item Projective space is an example of a homogeneous space where there is no sensible vector space in which to work. Geometry in projective space is the geometry of perspective: very important for engineers and designers of 3d-computer graphics, amongst other things. Define
\[\pr(\R^3)=\{\text{1-dimensional vector subspaces of }\R^3\}.\]
Just like for the sphere, the orthogonal group $\rO_3(\R)$ acts transitively on the set of lines through the origin. Similarly the stabilizer of a line $\ell$ contains the orthogonal group $\rO_2(\R)$ acting on the plane perpendicular to $\ell$. Moreover it also contains the orthogonal maps sending $\ell$ to itself, namely $\pm I$ on $\ell$. This is a copy of $\rO_1(\R)$. The isotropy subgroup of a line $\ell$ is therefore isomorphic to $\rO_2(\R)\times\rO_1(\R)$, and so
\[\pr(\R^3)=\rO_3(\R)/(\rO_2(\R)\times\rO_1(\R)).\]

Just as for the sphere, the above description can be generalized to cover the set of $k$-dimensional subspaces of $\R^n$ --- the so-called \emph{Grassmannian} $G_k(\R^n)$. We have
\[G_k(\R^n)=\rO_n(\R)/(\rO_k(\R)\times\rO_{n-k}(\R)).\]

Once again this description leads to the ability to do calculus in a set that is not a vector space --- there are many, many applications.
\end{enumerate}
\end{examples}

\fi