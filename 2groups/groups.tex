\graphicspath{{2groups/asy/}}

\section{Groups: Axioms and Basic Examples}\label{chap:groups}

%briefly cover some of the standard examples of group theory,  (e.g.\ subgroup, order, homomorphism, abstract) that dominates the rest of the course. Many of the examples and structures will be considered more rigorously in future chapters. The main purpose is at present is to gain familiarity and comfort by playing with examples.

In this chapter we define our main objects of study and introduce some of the common language that will be used throughout the course. Most of the examples are very simple and many should be familiar. We start by individually considering the axioms of a group.

\subsection{The Axioms of a Group}\label{sec:groupaxioms}

\begin{defn}{Closure}{}
A \emph{binary operation} $*$ on a set $G$ is a function $*:G\times G\to G$. Equivalently,
\[
	\forall x,y\in G,\text{ we have }x\opast y\in G\tag{$\dag$}
\]
We say that $G$ is \emph{closed} under $*$, and that $(G,*)$ is a \emph{binary structure.}
\end{defn}


In the abstract, including most theorems, we typically drop the symbol and use \emph{juxtaposition} ($x\opast y=xy$). In explicit \emph{examples} this might be a bad idea, say if $*$ is addition\ldots

\begin{examples}{}{}
\exstart Addition ($+$) is a binary operation on the set of \emph{integers} $\Z$: explicitly,\vspace{-2pt}
\begin{enumerate}\setcounter{enumi}{1}\itemsep2pt
\item[]\begin{quote}
Given $x,y\in\Z$, we know that $x+y\in\Z$
\end{quote}
\smallskip
This isn't a claim you can \emph{prove} since it is really part of the definition of addition on the integers.

  \item  Subtraction ($-$) is \emph{not} a binary operation on the positive integers $\N=\{1,2,3,4,\ldots\}$. This you can prove; to show that ($\dag$) is \emph{false,} simply exhibit a \emph{counter-example}
  \[1-7=-6\not\in\N\tag{$\exists x,y\in\N$ such that $x-y\not\in\N$}\]
  On the integers, however, subtraction is a binary operation.
  
  %\item Division ($\div$) is \emph{not} a binary operation on $\Z$: for instance $2\div 3=\frac 23\not\in\Z$. Division is a binary operation on the set $\R^\times$ of \emph{non-zero} real numbers.
  
 	\begin{minipage}[t]{0.79\linewidth}\vspace{0pt}
  \item\label{ex:table1} It can be convenient to use a table to represent a binary operation on a \emph{small} set; for instance the example describes an operation on a set of three elements $\{e,a,b\}$. Read the  \emph{left} column first, then the \emph{top} row; thus
  \[\textcolor{red}{a}\textcolor{blue}{b}=\textcolor{Green}{e}\]
	\end{minipage}\begin{minipage}[t]{0.2\linewidth}\vspace{0pt}
	\flushright $\begin{array}{c||c|c|c}
			* & e & a & \textcolor{blue}{b}\\\hline\hline
			e & e & a & b\\\hline
			\textcolor{red}{a} & a & e & \textcolor{Green}{e}\\\hline
			b & b & e & a
		\end{array}$
	\end{minipage}
\end{enumerate}
\end{examples}

We'll continue checking these examples for each of the group axioms.


\begin{defn}{Associativity}{assoc}
A binary structure $(G,\ast)$ is \emph{associative} if
\[\forall x,y,z\in G,\quad x(yz)=(xy)z\]
Associativity means that the expression $x y z$ has unambiguous meaning, as does the usual \emph{exponential/power} notation shorthand, e.g.\ $x^n=x\cdots x$.
\end{defn}

\begin{examples*}{ver.\,II}{}
\exstart Addition is associative: $x+(y+z)=(x+y)+z$ for any integers.\vspace{-2pt}
\begin{enumerate}\setcounter{enumi}{1}\itemsep2pt
  \item $(\Z,-)$ is non-associative: e.g.\ $(1-1)-2=-2\neq 2=1-(1-2)$.
  %\item $(\R^\times,\div)$ is non-associative: e.g.\ $2\div(4\div 2)=\frac 2{4/2}=1\neq \frac 18=\frac{2/4}2=(2\div 4)\div 2$.
  \item $\bigl(\{e,a,b\},*\bigr)$ is non-associative: e.g.\ $a(b^2)=a^2=e\neq b=eb=(ab)b$.
\end{enumerate}
\end{examples*}

\goodbreak

\begin{defn}{Identity}{}
A binary structure $(G,\ast)$ has an \emph{identity element} $e\in G$ if
\[\forall x\in G,\quad ex=xe=x\]
\end{defn}

\begin{examples*}{ver.\,III}{}
\exstart Addition has identity 0: that is $0+x=x+0=x$ for any integer $x$.
\begin{enumerate}\setcounter{enumi}{1}\itemsep2pt
  \item $(\Z,-)$ does not have an identity: e.g.\ if $e-x=x$, then $e=-2x$ depends on $x$!
  %\item $(\R^\times,\div)$ does not have an identity: $e\div x=\frac ex=x\implies e=x^2$ again depends on $x$.
  \item $\bigl(\{e,a,b\},*\bigr)$ has identity $e$; observe the first row and column of the table.
\end{enumerate}
\end{examples*}

By convention, if $G$ is finite and has an identity (e.g.\ Example \ref{ex:table1},) we list it first. Indeed, we can always list \emph{it} first, since\ldots

\begin{lemm}{Uniqueness of identity}{}
If a binary structure $(G,\ast)$ has an identity, then it is unique.
\end{lemm}

It is now legitimate to refer to \emph{the} identity $e$ using the \emph{definite article.} Uniqueness proofs in mathematics typically follow a pattern: suppose there are two such objects and show that they are identical.

\begin{proof}
Suppose $e,f\in G$ are identities. Then
\[ef=\begin{cases}
	f&\text{since $e$ is an identity}\\
	e&\text{since $f$ is an identity}
\end{cases}\]
We conclude that $f=e$.
\end{proof}

We used almost nothing about $(G,*)$; in particular it need not be associative (e.g.\ example \ref{ex:table1}).

\begin{defn}{Inverse}{}
Let $(G,*)$ have identity $e$. An element $x\in G$ has an \emph{inverse} $y\in G$ if
\[xy=yx=e\]
\end{defn}

\begin{examples*}{ver.\,IV}{}
\exstart Every integer $x$ has an inverse under addition: $x+(-x)=(-x)+x=0$.\vspace{0pt}
\begin{enumerate}\setcounter{enumi}{1}\itemsep2pt
  \item Since $(\Z,-)$ has no identity, the question of inverses makes no sense.
  
  \begin{minipage}[t]{0.7\linewidth}\vspace{-2pt}
  \item Since $e^2=a^2=ab=ba=e$, we see that every element has an inverse; indeed $a$ has \emph{two} inverses!
  \end{minipage}\begin{minipage}[t]{0.3\linewidth}\vspace{-2pt}
  \flushright $\begin{array}{l||c|c|c}
  \text{Element}&e&a&b\\\hline
  \text{Inverse(s)}&e&a,b&a
  \end{array}$
  \end{minipage}
\end{enumerate}
\end{examples*}



\begin{lemm}{Uniqueness of inverses}{uniqueinv}
Suppose $(G,\ast)$ is associative and has an identity. If $x\in G$ has an inverse, then it is unique.
\end{lemm}

\begin{proof}
Suppose $x$ has inverses $y,z\in G$. Then,
\[
	z(\textcolor{red}{xy})=(\textcolor{blue}{zx})y \implies z\textcolor{red}{e}=\textcolor{blue}{e}y \implies z=y
	\tag*{\qedhere}
\]
\end{proof}
Note where associativity was used in the proof. Example \ref{ex:table1} shows that this condition is \emph{necessary}: a non-associative structure can have non-unique inverses.

\goodbreak

\begin{defn}{Commutativity}{}
Let $(G,*)$ be a binary structure. Elements $x,y\in G$ \emph{commute} if $xy=yx$. We say that $*$ is \emph{commutative} if all elements commute:\vspace{-3pt}
\[\forall x,y\in G,\ xy=yx\]
\end{defn}


\begin{examples*}{ver.V}{}
\exstart Addition of integers is commutative: $\forall x,y\in\Z$, $x+y=y+x$.\vspace{-2pt}
\begin{enumerate}\setcounter{enumi}{1}\itemsep0pt
  \item Subtraction is \emph{non-commutative}: e.g.\ $2-3\neq 3-2$.
  %\item Division is non-commutative: e.g.\ $\frac 23\neq 3\frac 32$.
  \item The relation is commutative since its table is \emph{symmetric} across its main $\searrow$ diagonal.
\end{enumerate}
\end{examples*}

We simply assemble the pieces to obtain our main definition. 

\begin{defn}{Group axioms}{group}
A \emph{group} is a binary structure $(G,*)$ satisfying the \emph{associativity} and \emph{identity} axioms, and for which all elements have \emph{inverses.} This is summarized by the mnemonic\vspace{-2pt}
\begin{quote}
\emph{Closure, Associativity, Identity, Inverse}\vspace{-2pt}
\end{quote}
The \emph{order} of $G$ is its cardinality $\nm G$.
Moreover, $G$ \emph{abelian} if $*$ is \emph{commutative.}
\end{defn}


Of our examples, only $(\Z,+)$ is a group; indeed an \emph{abelian, infinite} (order), \emph{additive\footnote{\label{fn:additive}The operation is addition; a \emph{multiplicative} group follows the multiplication/juxtaposition convention. These are distinctions only of notation: e.g.\ $x+x+x=3x$ in an additive group corresponds to $xxx=x^3$ in a multiplicative group.}} group (the operation is addition). The same observations show that $(\Q,+)$, $(\R,+)$ and $(\C,+)$ are abelian groups.


\begin{examples}{}{rx}
\exstart The non-zero real numbers $\R^\times$ forms an abelian group under multiplication.\vspace{-3pt}
\begin{enumerate}\setcounter{enumi}{1}\itemsep0pt
  \item[]\begin{quote}\renewcommand{\arraystretch}{1.1}
		\begin{tabular}{@{}ll}
			\emph{Closure}&If $x,y\neq 0$, then $xy\neq 0$\\
			\emph{Associativity}&$\forall x,y,z,\ x(yz)=(xy)z$\\
			\emph{Identity}&If $x\neq 0$, then $1\cdot x=x\cdot 1=x$, so $1\in\R^\times$ is an identity\\
			\emph{Inverse}&Given $x\neq 0$, observe that $x^{-1}=\frac 1x$ is an inverse: $x\cdot \frac 1x=\frac 1x\cdot x=1$\\
			\emph{Commutativity}&If $x,y\neq 0$, then $xy=yx$
		\end{tabular}
	\end{quote}
	%As before, we cannot realistically prove these claims. 
	Similarly, $(\Q^\times,\cdot)$ and $(\C^\times,\cdot)$ are abelian groups.
	
	  
  \item The \emph{even} integers $2\Z=\{2z:z\in\Z\}$ form an abelian group under addition.
%   \begin{quote}
% 	\renewcommand{\arraystretch}{1.2}
% 	\begin{tabular}{@{}ll}
% 		\emph{Closure}&If $2x,2y\in 2\Z$, then $2x+2y=2(x+y)\in 2\Z$\\
% 		\emph{Associativity}&If $2x,2y,2z\in 2\Z$, then $2x(2y\cdot 2z)=8xyz=(2x\cdot 2y)\cdot 2z$\\
% 		\emph{Identity}&$0=2\cdot 0\in 2\Z$ is the additive identity\\
% 		\emph{Inverse}&$2x$ has inverse $-2x=2(-x)\in 2\Z$: indeed $2x+2(-x)=0=2(-x)+2x$\\
% 		\emph{Commutativity}&$2x+2y=2y+2x$
% 	\end{tabular}
% 	\end{quote}
	%Indeed the same is true for $n\Z=\{nz:z\in\Z\}$ and any constant $n\in\R$ (Exercise \ref{exs:nzgroup}).
  
  \item The \emph{odd} integers $1+2\Z=\{1+2n:n\in\Z\}$ do not form a group under addition since they are not closed: for instance, $1+1=2\not\in 1+2\Z$.
  
  \item Every vector space is an abelian group under addition. %This includes the $m\times n$ matrices $M_{m\times n}(\R)$, viewed as an $mn$-dimensional vector space.
	
	\item $(\R,\cdot)$ is \emph{not} a group, since $0$ has no multiplicative inverse. Similarly $(\Q,\cdot)$, $(\C,\cdot)$ are not groups.
	

	\begin{minipage}[t]{0.6\linewidth}\vspace{-9pt}
	\item\label{ex:smallcayley1} Groups of small order may be depicted in \emph{Cayley tables\footnotemark}.\smallbreak
	Groups of orders 1, 2 and 3 are shown: you should check that these are groups.\smallbreak
	Note the \emph{magic square property}: each row/column contains every element exactly once (see Exercise \ref{exs:magicsquare}).
	\end{minipage}\begin{minipage}[t]{0.4\linewidth}\vspace{-8pt}
	\flushright\smash[b]{$\begin{array}[t]{c||c}
			* & e\\\hline\hline
			e & e
		\end{array}
		\quad
		\begin{array}[t]{c||c|c}
			* & e & a\\\hline\hline
			e & e & a\\\hline
			a & a & e
		\end{array}
		\quad
		\begin{array}[t]{c||c|c|c}
			* & e & a & b\\\hline\hline
			e & e & a & b\\\hline
			a & a & b & e\\\hline
			b & b & e & a
		\end{array}$}
	\end{minipage}	
	\end{enumerate}
	\end{examples}

	\footnotetext{Englishman Arthur Cayley (1821--1895) was a pioneer of group theory. \emph{Abelian} similarly honors the Norwegian mathematician Niels Abel (1802--1829).}

\goodbreak

\begin{thm}{Cancellation laws \& inverses}{canc}
Suppose $G$ is a group and $x,y,z\in G$. Then
\begin{enumerate}\itemsep2pt
  \item $xy=xz\implies y=z$ \qquad\qquad 2.\lstsp $xz=yz\implies x=y$\setcounter{enumi}{2}
  \item $(xy)^{-1}=y^{-1}x^{-1}$
\end{enumerate}
\end{thm}

\begin{proof}
The first two parts are exercises. For the third,
\[y^{-1}x^{-1}(xy)=y^{-1}(x^{-1}x)y=y^{-1}ey=y^{-1}y=e\]
Thus $y^{-1}x^{-1}$ is an inverse of $xy$. Since inverses are unique, (Lemma \ref{lemm:uniqueinv}) we are done.
\end{proof}

%The third part of the theorem should remind you of \emph{matrix multiplication}: read on\ldots

\boldsubsubsection{Associativity and Functional Composition}\phantomsection\label{sec:assoc}

%Thankfully associativity often comes for free due to a simple result.

\begin{thm}{}{funcassoc}
Let $X$ be a set. Composition of functions $f:X\to X$ is associative.%\smallbreak
%If $X$ has at least 2 elements, then composition is non-commutative.
\end{thm}

\begin{proof}
Let $f,g,h:X\to X$. We have equality $(f\circ g)\circ h=f\circ(g\circ h)$ if and only if these functions do the same thing to every element $x\in X$. But this is trivial:
\begin{gather*}
\bigl((f\circ g)\circ h\bigr)(x)=(f\circ g)\bigl(h(x)\bigr)=f\bigl(g(h(x))\bigr)\quad\text{and,}\\
\bigl(f\circ(g\circ h)\bigr)(x)=f\bigl((g\circ h)(x)\bigr)=f\bigl(g(h(x))\bigr)
\end{gather*}
It follows that $\circ$ is associative. %Non-commutativity is Exercise \ref{exs:funcnoncomm}.
\end{proof}

By viewing rotations and reflections as functions, the theorem verifies associativity for the following.

\begin{cor}{}{rotationgroup}
The \emph{rotations} of a geometric figure form a group under composition.\smallbreak
The \emph{symmetries (rotations and reflections)} of a geometric figure form a group under composition.
\end{cor}

Checking the other axioms is an exercise: the identity is considered a rotation (by \ang 0!). %See, for instance, the symmetry group of the equilateral triangle (Example \ref{ex:motiv}).

\begin{defn}{}{rotdiklein}
\exstart If $\rho_k$ is rotation counter-clockwise by $\frac{2\pi k}n$ radians, then $R_n=\{\rho_0,\ldots,\rho_{n-1}\}$ is the \emph{rotation group} of a regular $n$-gon.
\begin{enumerate}\setcounter{enumi}{1}
  \item The \emph{dihedral group} $D_n$ is the symmetry group of a regular $n$-gon. 
  \item The \emph{Klein four-group\footnotemark} (denoted $V$) is the symmetry group of a \textcolor{Purple}{rectangle} (or a \textcolor{Brown}{rhombus}), where $\textcolor{Green}{a}$ represents rotation by \ang{180} and $\textcolor{red}{b},\textcolor{blue}{c}$ are reflections.
\end{enumerate}
\begin{center}
\begin{minipage}[t]{0.25\linewidth}\vspace{-12pt}
$\begin{array}[t]{c||c|c|c|c}
	\circ & e & \textcolor{Green}{a} & \textcolor{red}{b} & \textcolor{blue}{c}\\\hline\hline
	e & e & \textcolor{Green}{a} & \textcolor{red}{b} & \textcolor{blue}{c}\\\hline
	\textcolor{Green}{a} & \textcolor{Green}{a} & e & \textcolor{blue}{c} & \textcolor{red}{b}\\\hline
	\textcolor{red}{b} & \textcolor{red}{b} & \textcolor{blue}{c} & e & \textcolor{Green}{a}\\\hline
	\textcolor{blue}{c} & \textcolor{blue}{c} & \textcolor{red}{b} & \textcolor{Green}{a} & e
\end{array}$
\end{minipage}\begin{minipage}[t]{0.4\linewidth}\vspace{-17pt}
\flushright
%\includegraphics{group-rhombus}
\includegraphics{group-klein}
\end{minipage}
\end{center}
\end{defn}

\footnotetext{From the German \emph{Vierergruppe.} Felix Klein (1849--1925) was a pioneer in the application of group theory to geometry.}
	
\goodbreak

Since multiplication by an $n\times n$ matrix amounts to a function (e.g.\ $A\in \rM_n(\R)$ corresponds to a linear map $\R^n\to\R^n:\vx\mapsto A\vx$), we immediately conclude:

\begin{cor}{}{matrixmult}
Multiplication of square matrices is associative.
\end{cor}

\begin{example}{}{gln}
The \emph{general linear group} comprises the invertible $n\times n$ matrices under multiplication
\[\rGL_n(\R)=\{A\in \rM_n(\R):\det A\neq 0\}\]
Invertibility is assumed, associativity is the corollary, and closure follows from the familiar result
\[\det AB=\det A\det B\]
Finally the identity is given by (drum roll\ldots) the \emph{identity matrix} $I=\smash{\scalebox{0.6}{$\begin{pmatrix}
1&0&&\\
0&1&\ddots&\\
&\ddots&\ddots&0\\
&&0&1
\end{pmatrix}$}}$!!\smallbreak
This group is \emph{non-abelian} (when $n\ge 2$).
\end{example}

Look again at part 3 of Theorem \ref{thm:canc}: seem familiar?

%More examples of matrix groups are in the exercises.

% \begin{minipage}{0.72\textwidth}\vspace{0pt}
% \boldinline{Geometric-symmetry groups}
% 
% The symmetries of any geometric figure form a group under composition. A \emph{symmetry} is a transformation of the figure to occupy the same space as the original; essentially a \emph{self-congruence.} In the plane, these amount to \emph{rotations} and \emph{reflections.}\smallbreak
% In the introduction we encountered the \emph{dihedral group} $D_3$ of symmetries of the equilateral triangle. Recall that $D_3$ consists of three rotations (by \ang 0, \ang{120} and \ang{240}), and three reflections (across each altitude of the triangle).
% \end{minipage}\begin{minipage}{0.25\textwidth}\vspace{0pt}
% \flushright\includegraphics[scale=0.65]{intro-s3}
% \end{minipage}


\vfil

\begin{exercises}
Key concepts/definitions: make sure you can state the formal definitions
\begin{quote}
	\emph{Group\,(closure,\,associativity,\,identity,\,inverse)\quad Commutativity/abelian\quad Cayley table\quad $V$\quad \makebox[0cm][l]{$\rGL_n(\R)$}}
\end{quote}

% The point of this chapter is to become familiar with examples and notation, which only happens with \emph{practice\ldots} With the exception perhaps of the last 2--3, all the exercises should all be straightforward; ask questions if not.  


\begin{enumerate}
  \begin{minipage}[t]{0.72\linewidth}\vspace{0pt}
	\item Given the binary operation table, calculate
	\begin{enumerate}\itemsep0pt
		\item \makebox[150pt][l]{$c\opast d$\hfill (b)\lstsp} $a\opast (c\opast b)$
		\item[(c)] \makebox[150pt][l]{$(c\opast b)\opast a$\hfill (d)\lstsp} $(d\opast c)\opast (b\opast a)$
	\end{enumerate}
	\end{minipage}\hfill\begin{minipage}[t]{0.2\linewidth}\vspace{0pt}
	\flushright \scalebox{0.95}{$\begin{array}{c||c|c|c|c}
								* & a & b & c & d\\\hline\hline
								a & c & d & a & b\\\hline
								b & d & c & b & a\\\hline
								c & a & b & c & d\\\hline
								d & b & a & d & c
			\end{array}$}
	\end{minipage}\smallbreak

	\begin{minipage}[t]{0.72\linewidth}\vspace{0pt}
	\item A table for a binary operation on $\{a,b,c\}$ is given. Compute $a*(b*c)$ and $(a*b)*c$. Does the expression $a\opast b\opast c$ make sense? Explain why/why not.
	\end{minipage}\hfill\begin{minipage}[t]{0.2\linewidth}\vspace{0pt}
	\flushright \scalebox{0.95}{$\begin{array}{c||c|c|c}
			* & a & b & c\\\hline\hline
			a & b & c & b\\\hline
			b & c & a & a\\\hline
			c & b & a & c
	\end{array}$}
	\end{minipage}\par

	\item Are the binary operations in the previous questions commutative? Explain.


	\item\begin{enumerate}\itemsep0pt
	  \item Describe (\emph{don't write them all out!}) all possible binary operation tables on a set of two elements $\{a,b\}$. Of these, how many are commutative?
	  \item How many commutative/non-commutative operations are there on a set of \emph{$n$} elements?\par
	  (\emph{Hint: a commutative table has what sort of symmetry?})
	\end{enumerate}
	
  
  \item Which are binary structures? For those that are, which are commutative and which associative?\vspace{-6pt}
  \begin{enumerate}\itemsep0pt
    \item[(a)] \makebox[180pt]{$(\Z,*),\ a\opast b=a-b$\hfill (b)\lstsp} $(\R,*),\ a\opast b=2(a+b)$
    \item[(c)] \makebox[180pt]{$(\R,*),\ a\opast b=2a+b$\hfill (d)\lstsp} $(\R,*),\ a\opast b=\frac ab$\setcounter{enumii}{4}
    \item[(e)] \makebox[180pt]{$(\N,*),\ a\opast b=a^b$\hfill (f)\lstsp} $(\Q^+,*),\ a\opast  b=a^b$,  where $\Q^+=\{x\in\Q:x>0\}$
    \item[(g)] $(\N,*),\ a\opast b=$ product of the distinct prime factors of $ab$. Also define $1\opast 1=1$.\par
    (e.g.~$42\opast 10=(2\cdot 3\cdot 7)\opast (2\cdot 5)=2\cdot 3\cdot 5\cdot 7=210$) 
  \end{enumerate}
  

  \goodbreak
  
	\item For each axiom of an abelian group: if true, write it down; if false, provide a counter-example.\vspace{-6pt}
	\begin{enumerate}\itemsep2pt
	  \item \makebox[210pt]{$\N=\{1,2,3,\ldots\}$ under addition.\hfill (b)\lstsp} $\Q$ under multiplication.\setcounter{enumii}{2}
	  \item[(c)] \makebox[210pt]{$X=\{a,b,c\}$ with $x\opast y:=y$.\hfill (d)\lstsp} $\R^3$ with the cross/vector product $\times$.\setcounter{enumii}{4}
	  \item\label{exs:nzgroup} For each $n\in\R$, the set $n\Z=\{nz:z\in\Z\}$ of multiples of $n$ under addition.
  \end{enumerate}
  
  
  \item Determine whether each of the following sets of matrices is a group under multiplication.\vspace{-6pt}
  \begin{enumerate}\itemsep2pt
    \item \makebox[210pt]{$\mathcal K=\{A\in \rM_2(\R):\det A=\pm 1\}$\hfill (b)\lstsp} $\mathcal L=\{A\in \rM_2(\R):\det A=7\}$\setcounter{enumii}{2}
    \item $\mathcal N=\bigl\{\begin{smatrix}
    a&b\\0&d
    \end{smatrix}\in\rM_2(\R):ad\neq 0\bigr\}$
  \end{enumerate}

  
  \item\begin{enumerate}\itemsep2pt
    \item Prove the cancellation laws (Theorem \ref{thm:canc} parts 1 \& 2).
    \item True or false: in a group, if $xy=e$, then $y=x^{-1}$.
    \item\label{exs:multinverse2} In a (multiplicative) group, \emph{prove} that $(x^{-1})^n=(x^n)^{-1}$ for any $x$ and any $n\in\N$. How would we write this in an \emph{additive} group (see footnote \ref{fn:additive})?
  \end{enumerate}
    
  \item Let $G$ be a group. Prove the following:\vspace{-6pt}
  \begin{enumerate}\itemsep2pt
    \item $\forall x,y\in G,\ (xy x^{-1})^2=xy^2x^{-1}$
    \item $\forall x\in G,\ (x^{-1})^{-1}=x$
    \item $G$ is abelian $\iff\forall x,y\in G,\ (xy)^{-1}=x^{-1}y^{-1}$
  \end{enumerate}
  
%   \item\label{exs:pullback} Prove or disprove: $(\R\setminus\{1\},*)$ is an abelian group, where $x\opast y=x+y-xy$.\par
%   (\emph{Hint: factorize $x\opast y-1$})
  
  \item\begin{enumerate}\itemsep2pt
	  \item\label{exs:funcnoncomm} Suppose $X$ contains at least two distinct elements $x\neq y$. Prove that there exist functions $f,g:X\to X$ for which $f\circ g\neq g\circ f$.
		\item Show that multiplication of $n\times n$ matrices is non-commutative when $n\ge 2$.
	\end{enumerate}
  
  \item\begin{enumerate}\itemsep2pt
    \item Describe the symmetry group and Cayley table of a non-equilateral isosceles triangle.
    \item\label{exs:squarerot} Explicitly state the Cayley table for the rotation group $R_4$ of a square.
    \item Explain why the order of the dihedral group $D_n$ is $2n$.
    \item Prove the \emph{rotation} part of Corollary \ref{cor:rotationgroup}.
  \end{enumerate}
	
	
% 	\item The Lie bracket $[\ ,\ ]$ of two $n\times n$ matrices is defined by $[A,B]=AB-BA$.
% 	\begin{enumerate}
%   	\item Show that $[\ ,\ ]$ is an \emph{anti-commutative} binary operation on $\rM_n(\R)$, that is,
% 		\[\forall A,B\in \rM_n(\R),\qquad [A,B]=-[B,A]\]
% 		\item When $n=2$, give a counter-example to show that the Lie bracket is \emph{non-associative.}
%   	\item Let $\fso(n)=\{A\in \rM_n(\R):A^T=-A\}$ be the set of skew-symmetric $n\times n$ matrices.
% 		\begin{enumerate}
%    		\item Is matrix multiplication a binary operation on $\fso(n)$?
%    		\item Is the Lie bracket a binary operation on $\fso(n)$?
% 		\end{enumerate}
% 	\end{enumerate}
  


  \item Let $\mathcal U$ be a set and $\cP(\mathcal U)$ its power set (the set of subsets of $\mathcal U$).
  \begin{enumerate}\itemsep2pt
    \item Which of the group axioms is satisfied by the union operator $\cup$ on $\cP(\mathcal U)$?
    \item Repeat part (a) for the intersection operator.
    \item The \emph{symmetric difference} of sets $A,B\subseteq\mathcal U$ is the set
    \[A\triangle B:=(A\cup B)\setminus(A\cap B)\]
    \begin{enumerate}
      \item Use Venn diagrams to give a sketch argument that $\triangle$ is associative on $\cP(\mathcal U)$.
      \item Is $\bigl(\cP(\mathcal U),\triangle\bigr)$ a group? Explain your answer.
  	\end{enumerate}
  \end{enumerate}
  
  
  \item\label{exs:magicsquare} (Magic Square)\quad Suppose $(G,*)$ is associative and $G$ is finite.\par
  Prove that $(G,*)$ is a group if and only if its (multiplication) table satisfies two conditions:
		\begin{itemize}\itemsep0pt
		  \item[i.] One row and column (by convention the first) is a perfect copy of $G$ itself.
  		\item[ii.] Every element of $G$ appears exactly once in each row and column.
		\end{itemize}
	
 
%   \item Let $C[0,1]$ denote the set of continuous functions $f:[0,1]\to\R$, and $C^1[0,1]$ the \emph{differentiable} functions for which $f'$ is continuous.
% 	\begin{enumerate}
% 		\item Define $*$ on $C^1[0,1]$ by
% 		\[(f*g)(x):=\int_0^x f'(t)g'(t)\,\dt+f(0)+g(0)\]
% 		Is $*$ a binary operation? Is it commutative? Associative? Prove your assertions.\par
% 		(\emph{Hint: use the Fundamental Theorem of Calculus})
% 		
%     \item Suppose $e\in C^1[0,1]$ is an identity for $*$. Show that $e$ satisfies the integral equation
%     \[\int_0^x f'(t)e'(t)\,\dt+f(0)+e(0)=f(x)\]
%     for all functions $f\in C^1[0,1]$. Does such an $e$ exist? Is it unique?
%     
%     \item (If you've done a little analysis)\quad Define $*$ by
%     \[f*g=\begin{cases} f\ \mathrm{if}\ \max f\ge \max g,\\ g\ \mathrm{if}\ \max g > \max f. \end{cases}\]
%     Which result from elementary analysis guarantees that this is a binary operation on $C[0,1]$? Is the binary operation commutative? Does it have an identity? Explain your answer.
%   \end{enumerate}
  

	\end{enumerate}
	\end{exercises}

\clearpage

\subsection{Subgroups}\label{sec:subgroup}

In mathematics, the prefix \emph{sub-} usually indicates a \emph{subset} that retains whatever structure follows.

\begin{defn}{Subgroup}{}
Let $G$ be a group. A \emph{subgroup} of $G$ is a subset $H\subseteq G$ which is a group with respect to the \emph{same} binary operation; we write $H\le G$.\smallbreak
A subgroup $H$ is a \emph{proper subgroup} if $H\neq G$; this is written $H<G$.\smallbreak
The \emph{trivial subgroup} is the 1-element set $\{e\}$; all other subgroups are \emph{non-trivial}.
\end{defn}


\begin{examples}{}{basicsubgroup}
The following are immediate from the definition:
\begin{enumerate}\itemsep2pt
  \item\label{ex:basicsubgroup1} \makebox[230pt]{$\{e\}\le G$ and $G\le G$ for \emph{any} $G$ \hfill 2.\lstsp}$(\Z,+)<(\Q,+)<(\R,+)<(\C,+)$
	\item[3.] \makebox[230pt]{$(\Q^\times,\cdot)<(\R^\times,\cdot)<(\C^\times,\cdot)$ \hfill 4.\lstsp}$(\R^n,+)<(\C^n,+)$
% 	\item $(\R^m,+)\le (\R^n,+)$ if $m\le n$. This can be visualized in many ways, the simplest is to consider $\R^m=\Span\{\vect em\}\le \R^n=\Span\{\vect en\}$: the subgroup consists of all column vectors whose last $n-m$ entries are zero.
	\item[5.] \makebox[230pt]{$(2\Z,+)<(\Z,+)$  \hfill 6.\lstsp} $(R_3,\circ)\le (R_6,\circ)$\quad (rotation groups)
	%\item $(C(\R),+)<(C^1(\R),+)$ (all differentiable functions are continuous)
\end{enumerate}
\end{examples}

Thankfully you don't have to check all the group axioms to see that a subset is a subgroup.

\begin{thm}{Subgroup criterion}{subgroup}
Let $G$ be a group. A non-empty subset $H\subseteq G$ is a subgroup if and only if it is closed under the group operation and inverses. Otherwise said,
\[\forall h,k\in H,\ hk\in H \text{ and }  h^{-1}\in H\]
\end{thm}

\begin{proof}
($\Rightarrow$) \ $H$ is a group and therefore satisfies all the axioms, including closure and inverse.\smallbreak
$(\Leftarrow$) \ Since $H$ is a subset of $G$, the group operation on $G$ is automatically associative\footnotemark on $H$. By assumption, $H$ also satisfies the closure and inverse axioms, so it remains only to check the identity.\smallbreak
Since $H\neq\emptyset$, we may choose some (any!) $h\in H$, from which
\[e=hh^{-1}\in H\]
since inverses and products remain in $H$. The identity $e$ of $G$ therefore in $H$, and so $H$ is a group.
\end{proof}

\footnotetext{Definition \ref{defn:assoc} makes no claim as to \emph{where} $x(yz)=(xy)z$ lives!}

\vspace{-5pt}

\begin{examples}{}{}
\exstart All the above examples can be confirmed using the theorem. For instance,
\[2\Z=\{\ldots,-2,0,2,4,\ldots\}=\{2z:z\in\Z\}\]
is certainly a non-empty subset of the integers. Moreover, if $2m,2n\in 2\Z$, then
\[2m+2n=2(m+n)\in 2\Z\quad \text{and}\quad \!-(2m)=2(-m)\in 2\Z\]
whence $2\Z$ is closed under addition and inverses (negation).
% We consider some non-examples of subgroups 
\begin{enumerate}\setcounter{enumi}{1}
  \item The positive integers $\N=\{1,2,\ldots \}$ are closed under addition but not inverses (for instance no $x\in\N$ satisfies $x+2=0$). Thus $\N$ is not a subgroup of $\Z$ under addition.
	\item\label{intro:modex} Let $1+3\Z$ be the set of integers with remainder 1 when divided by 3:
	\[1+3\Z=\{1+3n:n\in\Z\}=\{1,4,7,10,13,\ldots, -2,-5,-8,\ldots\}\]
	Since $1\in 1+3\Z$ but $1+1=2\not\in 1+3\Z$, we see that $1+3\Z$ is not a subgroup of $(\Z,+)$.
\end{enumerate}
\end{examples}

\goodbreak

\begin{minipage}[t]{0.83\linewidth}\vspace{0pt}
\boldinline{Subgroup Diagrams}

It can be helpful to represent subgroup relations pictorially, where a descending line indicates a subgroup relationship. For instance, the diagram on the right summarizes \emph{four} subgroup relations
\[6\Z<2\Z<\Z\quad\text{and}\quad 6\Z<3\Z<\Z\]
\end{minipage}\hfill\begin{minipage}[t]{0.15\linewidth}\vspace{0pt}
\flushright$\xymatrix @C5pt @R12pt{
	& \Z \ar@{-}[dl] \ar@{-}[dr] &\\
	2\Z \ar@{-}[dr] & & 3\Z \ar@{-}[dl]\\
	& 6\Z &\\
}$
\end{minipage}\medbreak
where all four are groups under addition. If $G$ has only \emph{finitely many subgroups,} then its \emph{subgroup diagram} is the complete depiction of all subgroups.


\boldinline{Matrix subgroups}

In Example \ref{ex:gln} we saw that the invertible matrices $\rGL_n(\R)$ form a group under multiplication; here is one of its many subgroups, some others are in Exercise \ref{exs:subgpmatrix}.

\begin{example}{}{orthogonal}
The set $\rO_n(\R)=\{A\in\rM_n(\R):A^TA=I\}$ forms a subgroup of $\rGL_n(\R)$.
	\begin{itemize}\itemsep0pt
	  \item $I\in \rO_n(\R)$ so we have a non-empty set. Moreover, if $A\in \rO_n(\R)$, then
	  \[1=\det I=\det A\det A^T=(\det A)^2\implies \det A\neq 0\implies A\in\rGL_n(\R)\]
	  \item If $A,B\in\rO_n(\R)$, then
	  \begin{gather*}
	  (AB)^T(AB)=B^TA^TAB=B^TIB=B^TB=I,\quad\text{and,}\\
	  (A^{-1})^TA^{-1}=(A^T)^TA^T=(AA^T)^T=I^T=I
	  \end{gather*}
	  whence $AB$ and $A^{-1}\in\rO_n(\R)$.
	\end{itemize}
	We call this the \emph{orthogonal group.} When $n=2$ or 3, its elements may be recognized as rotations and reflections. For instance, the matrix $\frac 1{\sqrt 2}\begin{smatrix}
	1&-1\\1&1
	\end{smatrix}\in\rO_2(\R)$ rotates $\R^2$ counter-clockwise by \ang{45}.
\end{example}



\boldinline{Geometric subgroup proofs}\hypertarget{sec:geomdih}{}

Arranging figures such that every symmetry of one is also a symmetry of the other immediately results in a subgroup relationship!

% \begin{example}[lower separated=false, sidebyside, sidebyside align=top seam, sidebyside gap=0pt, righthand width=0.33\linewidth]{}{}
% The Klein four-group was \emph{defined} geometrically! Since each non-identity element $a,b,c$ is its own inverse, we have \emph{three distinct} proper non-trivial subgroups, each containing two elements,
% \[\{e,a\},\quad\{e,b\},\quad\{e,c\}\]
% Why are there no other interesting subgroups? If $a$ and $b$ both lie in a subgroup, then so must $ab=c$ and we'd obtain the entirety of $V$! The same argument applies to the other pairs.
% \tcblower
%  \flushright$\xymatrix{ & V \ar@{-}[dl] \ar@{-}[dr] \ar@{-}[d] & \\
%  \{e,a\} & \{e,b\} & \{e,c\} \\
%  & \{e\} \ar@{-}[ul] \ar@{-}[ur] \ar@{-}[u] &
% }$
% \end{example}



\begin{example}[lower separated=false, sidebyside, sidebyside align=top seam, sidebyside gap=0pt, righthand width=0.35\linewidth]{}{hexsubgroup}
A \textcolor{blue}{regular hexagon} has symmetry group $D_6=\{\rho_0,\ldots,\rho_5,\mu_0,\ldots,\mu_5\}$ consisting of six rotations and six reflections:
\begin{itemize}
  \item $\rho_k$ is rotation counter-clockwise by $\ang{60k}$; the identity is $\rho_0$.
  \item The $\mu_k$ are reflections across `diameters' of the hexagon as indicated in the pictures below.
\end{itemize}

Now draw two \textcolor{red}{equilateral triangles} inside the hexagon.

Each of the six symmetries of the equilateral triangle is also a symmetry of the hexagon! It follows that the symmetry group $D_3$ of the triangle is a subgroup of $D_6$ in two different ways:
\[\{e,\rho_2,\rho_4,\mu_0,\mu_2,\mu_4\}< D_6\quad\text{and}\quad \{e,\rho_2,\rho_4,\mu_1,\mu_3,\mu_5\}< D_6\]

\tcblower
\flushright\includegraphics[scale=0.95]{group-hexagon1}\\
\includegraphics[scale=0.95]{group-hexagon2}
\end{example}


%We'll properly discuss the groups in this example in Chapter \ref{chap:perm}.

\goodbreak

\begin{exercises}
Key concepts/definitions:
\begin{quote}
	\emph{(Proper/trivial/non-trivial) Subgroup
	\quad Closure under operation/inverses
	\quad Subgroup diagram}
\end{quote}

\begin{enumerate}
  \item Use Theorem \ref{thm:subgroup} to verify that $\Q^\times$ is a subgroup of $\R^\times$ under multiplication.
  
	\item Give two reasons why the \emph{non-zero} integers do not form a subgroup of $\Z$ under addition.
  	
  \item Explain the relationship between positive integers $m$ and $n$ whenever $(m\Z,+)\le (n\Z,+)$.

  \item Prove or disprove: the set $H=\{\frac a{2^n}:a\in\Z,n\in\N_0\}$ forms a group under addition.
    
  \item Use Theorem \ref{thm:subgroup} to explain why the set of \emph{rotations} of a planar geometric figure is a subgroup of the group of its rotations \emph{and} reflections.
  
 
  \item\begin{enumerate}
    \item Find the complete subgroup diagram of the Klein four-group.
  	\item Modelling Example \ref{ex:hexsubgroup}, draw three pictures which describe different ways in which the Klein four-group may be viewed as a subgroup of $D_6$.
  \end{enumerate}
  
  \item Find the subgroups and subgroup diagram of the rotation group $R_6=\{\rho_0,\ldots,\rho_5\}$, where $\rho_k$ is counter-clockwise rotation by $\ang{60k}$. 
  
  
%   \item\begin{enumerate}
%     \item How many elements are in the rotation group of a regular tetrahedron?\par
%     (\emph{Hint: any face can be rotated to any other and then oriented\ldots})
%     \item Repeat the question for a cube and a regular octahedron. Your answer should be the \emph{same}: can you think of a \emph{geometric} reason why?\par
%     (\emph{Hint: what does an octahedron have six of?})
% %     \item How large are the rotation groups of the tetrahedron, dodecahedron and icosahedron?There are three other platonic solids. What about the dodecahedron and the icosahedron?
%   \end{enumerate}
  
	\item Suppose $H$ and $K$ are subgroups of $G$. Prove that $H\cap K$ is also a subgroup of $G$.
   
	\item Let $H$ be a non-empty subset of a group $G$. Prove that $H$ is a subgroup of $G$ if and only if
  \[\forall x,y\in H,\ xy^{-1}\in H\]
  
  
  \item\label{exs:subgpmatrix} Prove that the following sets of matrices are groups under multiplication.
  \begin{enumerate}
    \item Special linear group: $\rSL_n(\R)=\bigl\{A\in \rM_n(\R):\det A=1\bigr\}$
    \item Special orthogonal group: $\rSO_n(\R)=\{A\in\rM_n(\R):A^TA=I\text{ and }\det A=1\}$
    \item $\mathcal Q_n=\bigl\{A\in \rM_n(\R):\det A\in\Q^\times\bigr\}$
    \item Symplectic group: $\rSp_{2n}(\R)=\bigl\{A\in \rM_{2n}(\R):A^TJA=J\bigr\}$, where $J=\scalebox{0.65}{$\left(\!\begin{array}{c|c}
    0&I_n\\\hline -I_n&0
    \end{array}\!\right)$}$ is a block matrix and $I_n$ the $n\times n$ identity matrix.
    \item $\rSL_n(\Z)=\bigl\{A\in \rM_n(\Z):\det A=1\bigr\}$: all entries in these matrices are \emph{integers.}\par
    (\emph{Hint: look up the classical adjoint $\operatorname{adj}A$ of a square matrix})
  \end{enumerate}
  Now construct a diagram showing the subgroup relationships between the groups
  \[\rGL_n(\R),\quad \rSL_n(\R),\quad \rO_n(\R),\quad \rSO_n(\R),\quad \mathcal Q_n,\quad \rSL_n(\Z)\tag{\emph{ignore $\rSp_{2n}(\R)$}}\]
 	
 	\item\label{exs:quaternion} The set $Q_8=\{\pm 1,\pm i,\pm j,\pm k\}$ forms a group of order eight under `multiplication' subject to the following properties:
  \begin{itemize}\itemsep0pt
    \item 1 is the identity.
    \item $-1$ commutes with everything; e.g.\ $(-1)i=-i=i(-1)$, etc.
    \item $(-1)^2=1$,\quad $i^2=j^2=k^2=-1$ \ and \ $ij=k$.
    \item Multiplication is associative.
  \end{itemize}
  \begin{enumerate}
    \item Find the Cayley table of $(Q_8,\cdot)$.\par
  	(\emph{Hint: You should easily be able to fill in 44 of 64 entries; now use associativity\ldots})
  
  	\item Find all subgroups of $Q_8$ and draw its subgroup diagram.
	\end{enumerate}
 
\end{enumerate}
\end{exercises}


\clearpage



\subsection{Homomorphisms \& Isomorphisms}\label{sec:morph}

A key goal of abstract mathematics is the comparison of similar/identical structures with outwardly different appearances. We describe such comparisons using \emph{functions.}

\begin{defn}{Homomorphism}{homo}
Suppose $(G,\ast)$ and $(H,\star)$ are binary structures and $\phi:G\to H$ a function. We say that $\phi$ is a \emph{homomorphism} of binary structures if
\[\forall x,y\in G,\ \phi(x\opast y)=\phi(x)\opstar\phi(y)\]
\end{defn}

For most of this course (certainly after this chapter), the binary structures will be groups.



\begin{examples}{}{}
\exstart The%
\def\opbl{\operatorname{\textcolor{blue}{+}}}
\def\oprd{\operatorname{\textcolor{red}{+}}} 
function $\phi:(\N,\textcolor{red}{+})\to(\R,\textcolor{blue}{+})$ defined by $\phi(x)=\sqrt 2x$ is a homomorphism,\vspace{-2pt}
\[\phi(x\oprd y)=\sqrt 2(x\oprd y)=\sqrt 2x\opbl \sqrt 2y=\phi(x)\opbl \phi(y)\]
It is worth spelling this out, since there are \emph{two} ways to combine addition and $\phi$:\vspace{-5pt}
\begin{enumerate}\setcounter{enumi}{1}\itemsep2pt
  \item[]\begin{itemize}
  	\item Sum $x\oprd y$, then map to $\R$ to obtain $\phi(x\oprd y)$.
  	\item Map to $\R$, then sum to obtain $\phi(x)\opbl\phi(y)$.
	\end{itemize}
	The homomorphism property says the results are \emph{always identical.}
	
  \item If $V,W$ are vector spaces then every linear map $\rT:V\to W$ is a group homomorphism:\footnotemark
  \[\forall\vv_1,\vv_2\in V,\quad \rT(\vv_1+\vv_2)=\rT(\vv_1)+\rT(\vv_2)\]
  This shows that you've been encountering homomorphisms your entire mathematical career, even in calculus: $\diff x(f+g)=\diff[f]{x}+\diff[g]{x}$ is a homomorphism property! 
\end{enumerate}
\end{examples}


\footnotetext{The scalar multiplication condition $\rT(\lambda\vv)=\lambda \rT(\vv)$ of a linear map is not relevant here.}

The most useful homomorphisms are \emph{bijective}: these get a special name.

\begin{defn}{Isomorphism}{iso}
An \emph{isomorphism} is a bijective/invertible homomorphism.\footnotemark\par
We say that $G$ and $H$ are \emph{isomorphic,} written $G\cong H$, if there exists an isomorphism $\phi:G\to H$.
\end{defn}

Why do we care about isomorphisms? It is because isomorphic groups have exactly the same structure; one is simply a relabelled version of the other!\smallbreak

Here is the procedure for showing that binary structures $(G,*)$ and $(H,\star)$ are isomorphic:
\begin{quote}
\begin{description}\itemsep2pt
  \item[\normalfont\emph{Definition}:] Define $\phi:G\to H$ (if necessary). As we'll see starting in Chapter \ref{chap:cyclic}, if $G$ is a set of equivalence classes you might need to check that $\phi$ is \emph{well-defined.}
  \item[\normalfont\emph{Homomorphism}:] Verify that $\phi(x\opast y)=\phi(x)\opstar\phi(y)$ for all $x,y\in G$.
  \item[\normalfont\emph{Injectivity/1--1}:] Check $\phi(x)=\phi(y)\implies x=y$.
  \item[\normalfont\emph{Surjectivity/onto}:] Check $\operatorname{range}\phi=H$. Equivalently $\forall h\in H,\ \exists g\in G$ such that $h=\phi(g)$.
\end{description}
\end{quote}

The last three steps can be done in any order. Injectivity/surjectivity might also be combined by exhibiting an explicit \emph{inverse function} $\phi^{-1}:H\to G$.

\footnotetext{These terms come from ancient Greek: \emph{homo-} (similar, alike), \emph{iso-} (equal, identical), and \emph{morph(e)} (shape, structure).}

\goodbreak

\begin{examples}{}{expiso}
\exstart We show that $(2\Z,+)$ and $(3\Z,+)$ are isomorphic groups.\vspace{-2pt}
\begin{enumerate}\setcounter{enumi}{1}
  \item[]\begin{description}
  	\item[\normalfont\emph{Definition}:] The obvious function is $\phi(x)=\frac 32x$; plainly $\phi(2m)=3n$ whence $\phi:2\Z\to 3\Z$.
  	\item[\normalfont\emph{Homomorphism}:] $\phi(x+y)=\frac 32(x+y)=\frac 32x+\frac 32y=\phi(x)+\phi(y)$
  	\item[\normalfont\emph{Injectivity}:] $\phi(x)=\phi(y)\implies \frac 32x=\frac 32y\implies x=y$.
  	\item[\normalfont\emph{Surjectivity}:] If $z=3n\in 3\Z$, then $z=\frac 32\cdot\frac 23z=\frac 32(2n)=\phi(2n)\in\operatorname{range}\phi$.
	\end{description}
	In the last step we essentially observed that the inverse function is $\phi^{-1}(z)=\frac 23z$.\medbreak
	More generally, whenever $m,n\neq 0$, the groups $(m\Z,+)$ and $(n\Z,+)$ are isomorphic. 

	\item\label{ex:expiso1}	The function $\phi(x)=e^x$ is an isomorphism of abelian groups $\phi:(\R,+)\cong(\R^+,\cdot)$.\vspace{-2pt}
	\begin{description}%\itemsep2pt
  	\item[\normalfont\emph{Definition}:] This is unnecessary since $\phi$ is given. However, note that both domain and codomain are \emph{abelian groups} and that $\R^+=(0,\infty)$ means the \emph{positive real numbers.}
  	\item[\normalfont\emph{Homomorphism}:] This is the familiar exponential law!
  	\[\phi(x+y)=e^{x+y}=e^xe^y=\phi(x)\phi(y)\]
  	\item[\normalfont\emph{Bijectivity}:] $\phi^{-1}(z)=\ln z$ is the inverse function of $\phi$.
	\end{description}
\end{enumerate}
\end{examples}



\boldsubsubsection{Non-isomorphicity \& Structural Properties}

Unless you have very small sets, you cannot realistically test every function $\phi:G\to H$ to see that structures are non-isomorphic! Instead we have to be a little more cunning.

\begin{defn}{Structural properties}{}
A \emph{structural property} is any property which is preserved under isomorphism: i.e.\ if $\phi:(G,*)\to (H,\star)$ is an isomorphism then $(G,*)$ and $(H,\star)$ have identical structural properties.
\end{defn}


The following is a non-exhaustive list of structural properties: we'll check a few in Exercise \ref{exs:structural1}.
\begin{description}
  \item[\normalfont\emph{Cardinality/order}:] Since $G$ and $H$ are bijectively paired, their cardinalities are the same.
  \item[\normalfont\emph{Commutativity \& Associativity}:] For instance, if $\ast$ is commutative, then
  \[\forall x,y\in G,\ \phi(x)\opstar\phi(y)=\phi(x\opast y)=\phi(y\opast x)=\phi(y)\opstar\phi(x)\]
  Since $\phi$ is bijective, this says that $\star$ is commutative on $H$.
  \item[\normalfont\emph{Identities \& Inverses}:] For instance, if $G$ has identity $e$, then $\phi(e)$ is the identity for $H$.
  \item[\normalfont\emph{Solutions to equations}:] Related equations in $G$ and $H$ have the same number of solutions: e.g.
  \[x\opast x=x\iff \phi(x)\opstar\phi(x)=\phi(x)\]
  The equations $x\opast x=x$ and $z\opstar z=z$ therefore have the same number of solutions.%\footnotemark
  \item[\normalfont\emph{Being a group}] If $G$ is a group, so also is $H$.
\end{description}

%\footnotetext{Such solutions are called \emph{idempotents}; thus existence of idempotents is a structural property.}

\goodbreak

\begin{examples}{}{}
\exstart The binary structures $(\N_0,+)$ and $(\N,+)$ are non-isomorphic, since $\N_0=\{0,1,2,3,\ldots\}$ contains an identity element 0 while $\N$ does not.

\begin{enumerate}\setcounter{enumi}{1}\itemsep2pt
  \begin{minipage}[t]{0.72\linewidth}\vspace{0pt}
  \item The binary structures defined by the two tables are non-isomorphic; the first is commutative while the second is not.
  \end{minipage}\begin{minipage}[t]{0.28\linewidth}\vspace{0pt}
  \flushright $\begin{array}[t]{c||c|c}
	* & a & b\\
	\hline\hline a & a & b\\
	\hline b & b & a
  \end{array}\quad
  \begin{array}[t]{c||c|c}
	\star & c & d\\
	\hline\hline c & c & d\\
	\hline d & c & d
  \end{array}$
  \end{minipage}\par 
% 	
%   \item The groups $(\Z_6,+_6)$ and $D_3$ both have order six, but are non-isomorphic since $\Z_6$ is abelian and $D_3$ is not.
  
  \item To see that $(\Q,+)$ and $(\R,+)$ are non-isomorphic groups, it is enough to recall that the sets have different cardinalities: $\Q$ is \emph{countably infinite} while $\R$ is \emph{uncountable.}
  
  \item $\rGL_2(\R)$ and $(\R,+)$ have the same cardinality; however, since the first is non-abelian and the second abelian, the two groups are non-isomorphic. 
\end{enumerate}
\end{examples}

Many properties are non-structural and therefore \emph{cannot} be used to show non-isomorphicity: the type of element (number, matrix, etc.), the type of binary operation (addition, multiplication, etc.).

\goodbreak



\boldsubsubsection{Transferring a Binary Structure}


We can turn a bijection into an isomorphism by imposing the homomorphism property. If $(H,\star)$ and a bijection $\phi:G\to H$ are given, we can \emph{define} a binary operation $\ast$ on $G$ by \emph{pulling-back} $\star$:
\[\forall x,y\in G,\ x\opast y:=\phi^{-1}\bigl(\phi(x)\opstar\phi(y)\bigr)\]
Plainly $\phi:(G,*)\cong(H,\star)$ is an isomorphism!
We can similarly \emph{push-forward} a structure from $G$ to $H$:
\[w\opstar z:=\phi\bigl(\phi^{-1}(w)\opast\phi^{-1}(z)\bigr)\]

\begin{example}{}{}
% \exstart As already seen, $\phi:2\Z\to3\Z:x\mapsto\frac 32x$ is a bijection. Define $\star$ on $3\Z$ by
% \[w\opstar z:=\frac 13wz \tag{$3m\opstar 3n=\frac 13(3m)(3n)=3mn\in 3\Z$}\]
% The pull-back of $\star$ to $2\Z$ is then
% \[x\opast y= \frac 23\left(\frac 32 x\opstar\frac 32y \right) =\frac 23\cdot \frac 34xy=\frac 12xy\]
% which is easily confirmed to be a binary operation on $2\Z$. Since $\phi$ is an isomorphism, we can verify certain shared structural properties:
% \begin{quote}
% \begin{description}
%   \item[\normalfont\emph{Cardinality}:] $2\Z$ and $3\Z$ are both uncountably infinite sets.
%   \item[\normalfont\emph{Commutativity \& Associativity}:] Both are commutative and associative, e.g.
%   \[a\opast (b\opast c)=(a\opast b)\opast c=\frac 14abc\]
%   \item[\normalfont\emph{Identities}:] $(2\Z,\ast)$ has identity 2 (e.g.\ $2\opast x=\frac 12\cdot 2x=x$) and $(3\Z,\star)$ has identity $\phi(2)=3$.
%   \item[\normalfont\emph{Inverses}:] Neither has inverses and therefore neither is a group: e.g. $4\in 2\Z$ has no inverse
%   \[4\opast x=2\iff 2x=2\iff x=1\not\in 2\Z\]
% \end{description}
% \end{quote}
%\begin{enumerate}\setcounter{enumi}{1}
  %\item 
  $\phi(x)=x^3+8$ is a bijection $\R\to\R$. If $\phi:(\R,*)\to(\R,+)$ is an isomorphism, then
	\[x\opast y:=\phi^{-1}\bigl(\phi(x)+\phi(y)\bigr)=\phi^{-1}(x^3+y^3+16)=\sqrt[3]{x^3+y^3+8}\]
	Since $(\R,+)$ is an abelian group and $\phi^{-1}$ an isomorphism, it follows that $(\R,*)$ is also an abelian group. Moreover, its identity must be
	\[\phi^{-1}(0)=\sqrt[3]{-8}=-2\]
	As a sanity check, observe that
	\[x\opast (-2)=\sqrt[3]{x^3+(-2)^3+8}=x\]
%\end{enumerate}
\end{example}



\boldsubsubsection{Up to Isomorphism: a common shorthand}\phantomsection\label{sec:uptoiso}

\begin{minipage}[t]{0.85\linewidth}\vspace{-10pt}
This phrase is ubiquitous in abstract mathematics. For an example of how it is used, note that if $(\{e,a\},*)$ is a group with identity $e$, then its Cayley table must be as shown (recall Example \ref*{ex:rx}.\ref{ex:smallcayley1}). This might be summarized by the phrase:
\end{minipage}\begin{minipage}[t]{0.14\linewidth}\vspace{-10pt}
\flushright $\begin{array}[t]{c||c|c}
			* & e & a\\\hline\hline
			e & e & a\\\hline
			a & a & e
		\end{array}$
\end{minipage}\par

\begin{quote}
\emph{Up to isomorphism,} there is a unique group of order two.
\end{quote}

More precisely: if $G$ is \emph{any} group of order two, then there exists an isomorphism $\phi:\{e,a\}\to G$. The expression `up to isomorphism' is essential; without it, the sentence is \emph{false,} since there are \emph{infinitely many} distinct groups of order two!

\goodbreak

% For the present, it is enough to note that this includes the freedom to relabel elements and reorder the rows/columns of a Cayley table (except the first).


\vfil

\begin{exercises}{}
Key concepts/definitions:
\begin{quote}
	\emph{Homomorphism\qquad Injective/surjective/bijective\qquad Isomorphism\qquad Structural property}\par
	\emph{`Up to isomorphism'}
\end{quote}

\begin{enumerate}
	\item Which of the following are homomorphisms/isomorphisms of binary structures? Explain.
  \begin{enumerate}
    \item \makebox[210pt][l]{$\phi:(\Z,+)\to (\Z,+)$, \ $\phi(n)=-n$\hfill (b)\lstsp} $\phi:(\Z,+)\to (\Z,+)$, \ $\phi(n)=n+1$
    \item[(c)] \makebox[210pt][l]{$\phi:(\Q,+)\to (\Q,+)$, \ $\phi(x)=\frac{4}{3}x$\hfill (d)\lstsp} $\phi:(\Q,\cdot)\to (\Q,\cdot)$, \ $\phi(x)=x^2$
    \item[(e)] \makebox[210pt][l]{$\phi:(\R,\cdot)\to (\R,\cdot)$, \ $\phi(x)=x^5$\hfill (f)\lstsp} $\phi:(\R,+)\to (\R,\cdot)$, \ $\phi(x)=2^x$
    \item[(g)] $\phi:(\rM_2(\R),\cdot)\to (\R,\cdot)$, \ $\phi(A)=\det A$
    \item[(h)] $\phi:(\rM_n(\R),+)\to (\R,+)$, $\phi(A)=\tr A=$ trace of the matrix $A$ (add the entries on the main diagonal).
  \end{enumerate}
  
  \item Show that $(\Z,+)\cong (n\Z,+)$ for any \emph{non-zero} constant $n$.
  
  \item Prove or disprove: $(\R^3,+)\cong(\R^3,\times)$ (cross product). 
  
  
  \item $\phi(n)=2-n$ is a bijection of $\Z$ with itself. For each of the following, define a binary relation $*$ on $\Z$ such that $\phi$ is an isomorphism of binary relations.
  \begin{enumerate}
    \item $\phi:(\Z,*)\cong (\Z,+)$
    \item $\phi:(\Z,*)\cong (\Z,\cdot)$
    \item $\phi:(\Z,*)\cong (\Z,\max(a,b))$
  \end{enumerate}
  
  
	\item $\phi(x)=x^2$ is a bijection $\phi:\R^+\to\R^+$. Find $x\opast y$ if $\phi$ is to be an isomorphism of binary structures
	\begin{enumerate}
	  \item $\phi:(\R^+,\ast)\to(\R^+,+)$
	  \item $\phi:(\R^+,+)\to(\R^+,\ast)$
	\end{enumerate}
  
%   \item Recall Show that $x\opast y=x+y-xy$ is the pull-back of $(\R^\times,\cdot)$ by $\phi(x)=1-x$. Hence provide a much faster argument that $(\R\setminus\{1\},*)$ is an abelian group.

  
  \item\label{exs:structural1} Suppose $\phi:(G,*)\to (H,\star)$ is an isomorphism of binary structures. Prove the following:
  \begin{enumerate}
    \item\label{exs:structidentity} If $e$ is an identity for $G$, then $\phi(e)$ is an identity for $H$.
    \item If $x\in G$ has an inverse $y$, then $\phi(x)\in H$ has an inverse $\phi(y)$.
    \item If $*$ is associative, so is $\star$.
  \end{enumerate}
  
  \item Let $\phi:(G,*)\to (H,\star)$ be a homomorphism of binary structures. Prove that the \emph{image}
  \[\phi(G)=\image \phi=\{\phi(x):x\in G\}\]
  is closed under $\star$ (thus $(\phi(G),\star)$ is a binary structure). If $(G,*)$ and $(H,\star)$ are both groups, show that $\phi(G)$ is a subgroup of $H$.
  
  \item Revisit Exercise \ref{exs:structidentity}. Suppose $e$ is an identity for $(G,*)$ and that $\phi:G\to H$ is merely a \emph{homomorphism.} Must $\phi(e)$ be an identity for $H$? Explain why/why not: does it matter whether $\phi$ is a homomorphism of groups?
  
  \item Let $G$ be the group of rotations of the plane about the origin under composition.
  \begin{enumerate}
		\item Show that $\phi:(\R,+)\to G$ defined by
		\[\phi(x)=\text{rotate counter-clockwise $x$ radians}\]
  	is a homomorphism of groups.
		\item Prove or disprove: $\phi$ is an \emph{isomorphism.}
	\end{enumerate}
	

  \item\begin{enumerate}
    \item Prove that $S:=\left\{\begin{smatrix}a&-b\\b&a\end{smatrix}\in \rM_2(\R)\right\}$ forms a group under matrix addition.
    \item Prove that $T=S\setminus\{0\}$ \ ($S$ except the zero matrix) forms a group under matrix \emph{multiplication.}
    \item Define $\phi\begin{smatrix}a&-b\\b&a\end{smatrix}=a+ib$. Prove that $\phi:S\to\C$ and $\phi_T:T\to\C^\times$ are \emph{both} isomorphisms
    \[\phi:(S,+)\cong(\C,+),\qquad \at\phi T:(T,\cdot)\cong(\C^\times,\cdot)\]
    (\emph{In a future class, $\phi$ will be described as an isomorphism of rings/fields})
  \end{enumerate}
  
  

	
% 	\item A group structure on $X=(-\frac\pi 2,\frac\pi 2)$ could be defined as follows:
% 	\[\forall x_1,x_2\in X,\text{ define }x_1*x_2=\tan^{-1}(\tan(x)+\tan(y))\]
% 	Since $\tan:(-\frac\pi 2,\frac\pi 2)\to\R$ is a bijection, we see that $*$ is simply the pull-back of the group structure $(\R,+)$ to $(-\frac\pi 2,\frac\pi 2)$ so that $\tan:X\to\R$ is an isomorphism.
  
	
	
	\item The groups $(\Q,+)$ and $(\Q^+,\cdot)$ are both abelian and both have the same cardinality. Assume, for contradiction, that $\phi:\Q\to\Q^+$ is an isomorphism.
	\begin{enumerate}
	  \item If $c\in\Q$ is constant, what equation in $\Q^+$ corresponds to $x+x=c$?
	  \item By considering how many solutions these equations have, obtain a contradiction and hence conclude that $(\Q,+)\ncong(\Q^+,\cdot)$.
	\end{enumerate}
	(Extra challenge) \ Suppose $\psi:(\Q,+)\to(\R,\cdot)$ is a \emph{homomorphism} and that $\psi(1)=a$: find a formula for $\psi(x)$.
	
% 	Suppose $\phi:\Q\to\Q^+$. $(\Q,+)\ncong(\Q^+,\cdot)$. This is harder, since \emph{both} structures are abelian groups and both have the same cardinality. This is where solutions to equations come in.\par
% 	If $c\in \Q$ is given, then the equation $x+x=c$ has exactly one solution $x=\frac c2$. If $\phi:\Q\to\Q^+$ were an isomorphism, then
% 	\[\phi(x)\cdot\phi(x)=\phi(c)\implies y=\phi(x)\text{ solves }y^2=\phi(c)\]
% 	However, $\phi$ is surjective, whence $\exists c\in\Q$ such that $\phi(c)=2$, and our claim is now that $y^2=2$ has a solution $y\in\Q^+$: a contradiction.\par
% 	To summarize, the equation $y^2=2$ has no solution in $\Q^+$, but the `corresponding equation' $x+x=c$ in $\Q$ has a solution for every $c\in\Q$.

	\item Recall the magic square property (Exercise \ref*{sec:groupaxioms}.\ref{exs:magicsquare}).
	\begin{enumerate}
	  \item Up to isomorphism, explain why there is a unique group of order 3; its Cayley table should look like that of the rotation group $R_3$.
	  
	  \item Show that there are only two ways to complete a Cayley table of order 4 up to isomorphism.\par
		(\emph{Hints: if $G=\{e,a,b,c\}$, why may we assume, without loss of generality, that $b^2=e$? Your answers should look like the Klein four-group $V$ and the rotation group $R_4$.})
	\end{enumerate}
		
	\item\label{exs:isomorphiccomposition} Prove that \emph{isomorphic} is an equivalence relation on any collection of groups: that is, for all groups $G,H,K$, we have
	\begin{quote}
	\begin{description}
	  \item[\normalfont Reflexivity] $G\cong G$
	  \item[\normalfont Symmetry] $G\cong H\implies H\cong G$
	  \item[\normalfont Transitivity] $G\cong H$ and $H\cong K\implies G\cong K$
	\end{description}
	\end{quote}
	
\end{enumerate}
\end{exercises}

