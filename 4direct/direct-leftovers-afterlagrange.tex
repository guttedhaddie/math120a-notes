
\begin{aside}
\vspace{-13pt}
\subsubsection*{Aside: Subgroups of Abelian Groups}

Recall Lagrange's Theorem and how it doesn't have a converse: if $m$ is a divisor of $\nm G$ then we cannot, in general, claim that $G$ has a subgroup of order $m$. The smallest group for which this is evident is $A_4$ which, though it has order 12, has no subgroup of order 6. When $G$ has certain special forms however, we can find partial converses to Lagrange: for instance, if $G$ is cyclic then $G$ has exactly one subgroup for every integer dividing $\nm G$. The following corollaries of the Fundamental Theorem illustrate that a similar thing happens for finite Abelian groups.

\begin{cor}\label{thm:abelsub}
Let $G$ be a finite Abelian group. Then $G$ has a subgroup of order $m$ for every divisor $m$ of $\nm G$.
\end{cor}

\begin{proof}
By the Fundamental Theorem we can write $G\cong\Z_{p_1^{r_1}}\times\cdots\times\Z_{p_n^{r_n}}$. The order of $G$ is then $p_1^{r_1}\cdots p_n^{r_n}$. If $m$ is a divisor of $\nm G$ then $m=p_1^{s_1}\cdots p_n^{s_n}$ for some integers $s_i\le r_i$.\footnote{Even though the primes $p_i$ do not have to be distinct, we still write $m$ in the above manner and distinguish between exponents $s_i,s_j$ even if $p_i=p_j$.} It is straightforward to check the following:
\begin{itemize}
  \item $p_i^{r_i-s_i}$ has order $p_i^{s_i}$ in $\Z_{p_i^{r_i}}$.
  \item $\ip{p_1^{r_1-s_1}}\times\cdots\times\ip{p_n^{r_n-s_n}}$ is a subgroup of $G$ with order $m$.\qedhere
\end{itemize}
\end{proof}

There is no uniqueness here. For example the Klein 4-group $\Z_2\times\Z_2$ has \emph{three} distinct subgroups of order 2.
\end{aside}


\subsubsection*{Factor Groups and Direct Products}

Identifying factor groups of direct products, perhaps using the first isomorphism, is something of a challenge, bringing together everything in the last few sections.

\paragraph{Examples}
\begin{enumerate}
\item Classify the factor group $\quotient{(\Z_4\times\Z_8)}{\ip{(0,1)}}$ in terms of the Fundamental Theorem \ref{thm:fund}. Otherwise said, the factor group is a finite Abelian group, so we must be able to write it as a direct product of $\Z_i$'s.\\
The cyclic subgroup $\ip{(0,1)}$ has order 8 and its cosets are
\[\quotient{(\Z_4\times\Z_8)}{\ip{(0,1)}}=\Bigl\{(0,0)+\ip{(0,1)},(1,0)+\ip{(0,1)}, (2,0)+\ip{(0,1)},(3,0)+\ip{(0,1)}\Bigr\}\]
There are no further cosets (either observe that a non-zero entry in the $\Z_8$ part of $(i,j)$ can be subtracted away by an element in the cyclic group $\ip{(0,1)}$, or count elements via the index of $\ip{(0,1)}$; there are $(\Z_4\times\Z_8:\ip{(0,1)})=\frac{4\cdot 8}{8}=4$ cosets).\\
We therefore have two possibilities: the factor group is isomorphic to $\Z_4$ or $\Z_2\times\Z_2$. Which?\\
A straightforward answer comes by observing that the factor group is generated by the coset
\[(1,0)+\ip{(0,1)}\]
whence the factor group is cyclic and thus isomorphic to $\Z_4$. How about obtaining an explicit isomorphism $\quotient{(\Z_4\times\Z_8)}{\ip{(0,1)}}\cong\Z_4$? Given that the above coset is a generator, simply map it to a generator of $\Z_4$: indeed
\[\mu:\bigl((i,0)+\ip{(0,1)}\bigr)=i\]
is easily seen to be a well-defined isomorphism.\\
Comparing with the first isomorphism theorem we see that $\ip{(0,1)}$ is the kernel of the homomorphism
\[\phi:\Z_4\times\Z_8\to\Z_4:(x,y)\mapsto x\]
and that the following diagrams commute.
\[\xymatrix{\Z_4\times\Z_8 \ar[rr]^{\phi} \ar[dr]_\gamma && \Z_4\\
&\quotient{(\Z_4\times\Z_8)}{\ip{(0,1)}} \ar[ur]_{\mu} &}
\qquad\qquad
\xymatrix{(x,y) \ar@{|->}[rr]^{\phi} \ar@{|->}[dr]_\gamma && x\\
& (x,y)+\ip{(0,1)} \ar@{|->}[ur]_{\mu} &}\]

\item We can do something similar for $\quotient{(\Z_4\times\Z_8)}{\ip{(0,2)}}$. This time the cyclic subgroup has order 4. As with the previous example, the elements of of the cyclic subgroup only influces second factor $\Z_8$. We therefore expect to see
\[\quotient{(\Z_4\times\Z_8)}{\ip{(0,2)}}\cong\Z_4\times\quotient{\Z_8}{\ip{2}}\cong\Z_4\times\Z_2.\]
This time we try to produce a homomorphism first:
\[\phi:\Z_4\times\Z_8\to\Z_4\times\Z_2:(x,y)\mapsto (x,y\negthickspace\negthickspace\mod 2)\]
This is easily checked to be a well-defined homomorphism. In the context of the first isomorphism theorem we have the following diagrams:
\[\xymatrix{\Z_4\times\Z_8 \ar[rr]^{\phi} \ar[dr]_\gamma && \Z_4\times\Z_2\\
&\hspace{-10pt}\quotient{(\Z_4\times\Z_8)}{\ip{(0,2)}}\hspace{-10pt} \ar[ur]_{\mu} &}
\qquad
\xymatrix{(x,y) \ar@{|->}[rr]^{\phi} \ar@{|->}[dr]_\gamma && (x,y\negthickspace\negthickspace\mod 2)\\
&\hspace{-10pt} (x,y)+\ip{(0,2)}\hspace{-10pt} \ar@{|->}[ur]_{\mu} &}\]

\item The previous examples may have lulled you into a false sense of security: to compute a factor group you DON'T simply divide by the order of each number in the lower cyclic group. Here is an example where this approach goes wrong.\\
You might assume that the factor group $\quotient{(\Z_4\times\Z_8)}{\ip{(2,4)}}$ is isomorphic to 
\[\quotient{\Z_4}{\ip 2}\times\quotient{\Z_8}{\ip 4}\cong\Z_2\times\Z_4\tag*{($\ast$)}\]
but you'd be wrong! Indeed this is very easy to see by counting cosets: the cyclic subgroup $\ip{(2,4)}$ has order 2, whence it has
\[\Bigl(\Z_4\times\Z_8:\ip{(2,4)}\Bigr)=\frac{4\cdot 8}2=16\]
cosets. If ($\ast$) was correct, then'd only have 8 cosets!\\
The factor group $\quotient{(\Z_4\times\Z_8)}{\ip{(2,4)}}$ is Abelian of order 16. Checking the table on page \pageref{pg:fundabel} we see that there are five possibilities for this group. A little scratch work allows us to identify which it is:
\begin{itemize}
  \item Suppose that $(x,y)$ is an element of the coset $(a,b)+\ip{(2,4)}$. Then $(x-2,y-4)$ is also a member of the same coset. By considering the first term, this means that there is precisely one element of each coset whose first entry is 0 or 1.
  \item It should now be clear that the following elements all lie in distinct cosets of $\ip{(2,4)}$:
	\[(0,0),\ (0,1),\ \ldots,\ (0,7),\ (1,0),\ \ldots,\ (1,7).\]
	\item It now seems reasonable to conjecture that the factor group is isomorphic to $\Z_2\times\Z_8$.
\end{itemize}
	Here are two possible ways to prove this:
\begin{enumerate}
  \item Observe that the coset $(0,1)+\ip{(2,4)}$ has order 8 in the factor group. This narrows our choice to $\Z_{16}$ or $\Z_2\times\Z_8$. Now check that \emph{all} cosets have order at most 8:
  \[8\Bigl((x,y)+\ip{(2,4)}\Bigr)=(8i,8j)+\ip{(2,4)}=(0,0)+\ip{(2,4)}\]
  There are no elements of order 16 whene we can rule out $\Z_{16}$ as a candidate. The only possibility remaining is $\Z_2\times\Z_8$.
  \item We can define an explicit homomorphism and appeal to the first isomorphism theorem.
	\[\phi(x,y)=(x\negthickspace\negthickspace\mod 2,\ y-2x\negthickspace\negthickspace\mod 8),\qquad \phi:\Z_4\times \Z_8\to\Z_2\times\Z_8.\]
	Now check the following:
	\begin{itemize}
	  \item $\phi$ really is a homomorphism: this is obvious since $\phi$ is linear in both slots.
	  \begin{align*}
	  \hspace*{-21pt}\phi\Bigl((x,y)+(v,w)\Bigr)&=\phi(x+v,y+w)=\Bigl(x+v\spmod 2,\ y+w-2(x+v)\spmod 8\Bigr)\\
	  &=\Bigl(x\spmod 2,\ y-2x\spmod 8\Bigr)+\Bigl(v\spmod 2,\ w-2v\spmod 8\Bigr)\\
	  &=\phi(x,y)+\phi(v,w)
	  \end{align*}
	  \item $\ker\phi=\ip{(2,4)}$. For this, note that
	  \begin{align*}
	  \phi(x,y)=(0,0)&\implies x=2n\ \text{for some $n\in\Z$}\\
	  &\implies y\equiv 4n\pmod 8\\
	  &\implies (x,y)\in\ip{(2,4)}
	  \end{align*}
	  Conversely, it is clear that $\phi(2n,4n)=(0,0)$.
	  \item $\phi$ is surjective. Observe that
	  \[\forall (p,q)\in\Z_2\times\Z_8,\quad (p,q)=\phi(p,q+2p)\]
	\end{itemize}
 It follows that $\quotient{(\Z_4\times\Z_8)}{\ip{(2,4)}}\cong\image\phi=\Z_2\times\Z_8$.
\end{enumerate}\pagebreak[2]

\item As a final example we do something far more difficult: nothing this tricky is examinable!\\
Let $H=\ip{(4,2,3)}$ be a subgroup of the finitely generated Abelian group $\Z_{10}\times\Z_6\times\Z$. Identify the factor group $\quotient GH$.\\
The first problem is counting cosets. Both $G$ and $H$ are \emph{infinite} groups, so we can't simply use the index to find the cardinality of the factor group. However, similarly to the previous example, the approach is to identify a representative element of each coset. The infinite group $\Z$ in the direct product makes this more challenging so we start with it.
\begin{itemize}
  \item In every coset $(x,y,z)+H$, there is precisely one element $(x,y,z)$ with $z=0,1$ or 2.
  \item The elements $(x,y,0)$ where $x\in\Z_{10}$ and $y\in\Z_6$ all lie in different cosets: if two such were in the same coset, then
  \begin{align*}
  (x_1,y_1,0)-(x_2,y_2,0)\in H&\iff (x_1-x_2,y_1-y_2,0)\in H\\
  &\iff\begin{cases}
  x_1\equiv x_2\pmod{10}\\
  \qquad\quad\text{and}\\
  y_1\equiv y_2\pmod{6}
  \end{cases}
  \end{align*}
  Similarly all the elements $(x,y,1)$ and $(x,y,2)$ lie in different cosets.
  \item It follows that the elements $(x,y,0)$, $(x,y,1)$ and $(x,y,2)$ are representatives of different cosets, and that all cosets have one such representative. There are $10\cdot 6\cdot 3=180$ such cosets and so the factor group $\quotient GH$ is Abelian of order 180.
  \item The prime decomposition of $180$ is $2^2\cdot 3^2\cdot 5$, whence there are, up to isomorphism, \emph{four} Abelian groups of order 180, namely
  \begin{gather*}
  \Z_4\times\Z_9\times\Z_5\cong\Z_{180}\\
  \Z_2\times\Z_2\times\Z_9\times\Z_5\cong\Z_2\times\Z_{90}\\
  \Z_4\times\Z_3\times\Z_3\times\Z_5\cong\Z_3\times\Z_{60}\\
  \Z_2\times\Z_2\times\Z_3\times\Z_3\times\Z_5\cong\Z_6\times\Z_{30}
  \end{gather*}
  How do we distinguish between these? Like before we look for elements of a particular order (lots of scratch work may be required!).
  \item We compute the order of $(1,1,1)+H$. Certainly its order must be divisible by 3, since the third entry of
  \[n(1,1,1)=(n,n,n)\in H\]
  requires $3\mid n$. Thus let $n=3k$. Now
  \begin{align*}
  3k(1,1,1)+H&=(3k,3k,3k)+H=(3k,3k,3k)-k(4,2,3)+H\tag*{(since $(4,2,3)\in H$)}\\
  &=(-k,k,0)+H\\
  &=H\iff 10\mid k\ \text{and}\ 6\mid k \iff 30\mid k\iff 90\mid n
  \end{align*}
  It follows that $\quotient GH$ contains an element of order 90. Our choices are narrowed to $\Z_{180}$ and $\Z_2\times\Z_{90}$.
  \item Finally we show that \emph{every} element of $\quotient GH$ has order dividing 90:
  \begin{align*}
  90(x,y,z)+H&=(90x,90y,90z)-30z(4,2,3)+H=(90x-120z,90y-60z,0)+H\\
  &=(0,0,0)+H
  \end{align*}
  There are no elements of order 180, so our factor group cannot be isomorphic to $\Z_{180}$. We conclude that $\quotient GH\cong\Z_2\times\Z_{90}$. Phew!
\end{itemize}
What is perhaps surprising, given our representative elements, is that the factor group is \emph{not} isomorphic to $\Z_{10}\times\Z_6\times\Z_3\cong \Z_6\times\Z_{30}$. Given this, finding an explicit surjective homomorphism $\phi:G\to\Z_2\times\Z_{90}$ with kernel $H$ seems hard! The trick is to find an element of order 90 (we already have this in $(1,1,1)+H$) and one of order 2 \emph{which is not} in our order 90 subgroup. With a little inspection it can be seen that $(0,3,0)+H$ is such an element. With a bit of work we can construct a suitable function: indeed it can be shown that
\[\phi(x,y,z)=\Bigl(x-y\spmod 2,\ 27x+30y+34z\spmod{90}\Bigr)\]
is a surjective homomorphism with
\[\ker\phi=H,\qquad \phi(0,3,0)=(1,0),\qquad \phi(1,1,1)=(0,1)\]
which is inspired by the idea that the corresponding isomorphism $\mu:\quotient GH\to\Z_2\times\Z_{90}$ satisfies
\[\mu^{-1}(X,Y)=(Y,3X+Y,Y)+H\]
A straightforward calculation confirms that
\[\mu^{-1}(\phi(x,y,z))=(x,y,z)+H=\gamma(x,y,z)\]
This is all \emph{very} tricky to work through!
\end{enumerate}

\paragraph{Warning!}
It may be tempting to take the idea of `division of groups' too far. For example, suppose we are given that $H\triangleleft G$. It might seem reasonable to assume that $H\times (\quotient GH)\cong G$. Unfortunately life isn't as simple as this! For example, if we take $G=\Z_8$ and $H=\ip{4}=C_2$, then
\[\quotient GH=\quotient{\Z_8}{C_2}\cong \Z_4,\]
since the coset $1+\ip 4$ has order 4. But then
\[\Z_2\times\quotient{\Z_8}{C_2}\cong\Z_2\times\Z_4\ncong\Z_8.\]
\newpage

\begin{asidep}\vspace{-12pt}
\paragraph{Projections and Subspaces}

$\pi_i:G_1\times\cdots G_n\to G_i$ defined as projection onto the $i^{\text{th}}$ factor
\[\pi_i(g_1,\ldots,g_n)=g_i\]
is a homomorphism. Therefore
\[\ker\pi_i=G_1\times\cdots\times G_{i-1}\times\{e_i\}\times G_{i+1}\times\cdots\times G_n\triangleleft G_1\times\cdots\times G_n.\]
If we are willing to abuse notation we can say that $\ker\pi_i=G_1\times\cdots\times G_{i-1}\times G_{i+1}\times\cdots\times G_n$ is a normal subgroup of the whole direct product. Indeed any combination of projections is a homomorphism (e.g.\ $(\pi_1,\pi_4,\pi_5):G_1\times\cdots\times G_n\to G_1\times G_4\times G_5$), and so we can say that
\[\prod_{j\in I}G_j\triangleleft\prod_{i=1}^n G_i,\qquad\text{for \emph{any} subset}\ I\subseteq\{1,\ldots n\}.\]
Let us translate the case when $n=2$ using the first isomorphism theorem to identify the factor group $\quotient{(G_1\times G_2)}{G_1}$. The elements of this group are (left) cosets of the form
\[(g_1,g_2)(G_1\times\{e_2\})\]
The map $\mu:\quotient{(G_1\times G_2)}{G_1}\to G_2$ which maps the above coset to $g_2$ is an isomorphism.\\
In the context of vector spaces, this is just the idea that, say, $\R^2=\Span(\vi,\vk)$ is a subspace of $\R^3$, viewed as the kernel of the projection map
\[\pi_2:(x,y,z)\mapsto (0,y,0)\]
We could then claim that
\[\quotient{\R^3}{\R^2}\cong\R=\image\pi_2\]
\end{asidep}


