\graphicspath{{4direct/asy/}}

\section{Direct Products \& Finitely Generated Abelian Groups}\label{chap:direct}


In this short chapter we discuss a straightforward way to create new groups from old using the \emph{Cartesian product.}

\begin{example}{}{}
	Given $\Z_2=\{0,1\}$, the Cartesian product
	\[
		\Z_2\times \Z_2=\bigl\{(0,0),(0,1),(1,0),(1,1)\bigr\}
	\]
	has \emph{four} elements. This set has a natural group structure by via addition of co-ordinates
	\[
		(x,y)+(v,w):=(x+v,y+w)
	\]
	where $x+v$ and $y+w$ are both computed in $(\Z_2,+_2)$. This is a binary operation on $\Z_2\times\Z_2$, with a familiar-looking Cayley table: it has exactly the same structure as the Klein four-group!
	\begin{quote}
		\scalebox{1}{$%
			\begin{array}{c||c|c|c|c}
				+ & (0,0) & \textcolor{Green}{(0,1)} & \textcolor{red}{(1,0)} & \textcolor{blue}{(1,1)}\\\hline\hline
				(0,0) & (0,0) & \textcolor{Green}{(0,1)} & \textcolor{red}{(1,0)} & \textcolor{blue}{(1,1)}\\\hline
				\textcolor{Green}{(0,1)} & \textcolor{Green}{(0,1)} & (0,0) & \textcolor{blue}{(1,1)} & \textcolor{red}{(1,0)}\\\hline
				\textcolor{red}{(1,0)} & \textcolor{red}{(1,0)} & \textcolor{blue}{(1,1)} & (0,0) & \textcolor{Green}{(0,1)}\\\hline
				\textcolor{blue}{(1,1)} & \textcolor{blue}{(1,1)} & \textcolor{red}{(1,0)} & \textcolor{Green}{(0,1)} & (0,0)
			\end{array}%
		$}
		\qquad\scalebox{1.5}{$\leftrightsquigarrow$}\qquad
		\scalebox{1}{$%
			\begin{array}{c||c|c|c|c}
				\circ & e & \textcolor{Green}{a} & \textcolor{red}{b} & \textcolor{blue}{c}\\\hline\hline
				e & e & \textcolor{Green}{a} & \textcolor{red}{b} & \textcolor{blue}{c}\\\hline
				\textcolor{Green}{a} & \textcolor{Green}{a} & e & \textcolor{blue}{c} & \textcolor{red}{b}\\\hline
				\textcolor{red}{b} & \textcolor{red}{b} & \textcolor{blue}{c} & e & \textcolor{Green}{a}\\\hline
				\textcolor{blue}{c} & \textcolor{blue}{c} & \textcolor{red}{b} & \textcolor{Green}{a} & e
			\end{array}%
		$}
	\end{quote}
	We conclude that $\Z_2\times\Z_2\cong V$ is indeed a group.
\end{example}


This construction works in general.

\begin{thm}{Direct product}{directprod}
	The natural component-wise operation on the Cartesian product
	\[
		\smash[t]{\prod\limits_{k=1}^n} G_k=G_1\times\cdots\times G_n,\qquad (x_1,\ldots,x_n)\cdot(y_1,\ldots,y_n):=(x_1y_1,\ldots,x_ny_n)
	\]
	defines a group structure: the \emph{direct product.} This group is abelian if and only if each $G_k$ is abelian.
\end{thm}

The proof is a simple exercise. Being a Cartesian product, a direct product has order equal to the product of the orders of its components
\[
	\nm{\prod\limits_{k=1}^n G_k}=\prod\limits_{k=1}^n\nm{G_k}
\]
\vspace{2pt}


\begin{examples}{}{direct1}
	\exstart The direct product of the groups $(\Z_2,+_2)$ and $(\Z_3,+_3)$ is
	\[
		\Z_2\times\Z_3=\bigl\{(0,0),(0,1),(0,2),(1,0),(1,1),(1,2)\bigr\}
	\]
	This is abelian of order 6, so we might guess it to be isomorphic to $(\Z_6,+_6)$ and thus cyclic. This is indeed the case: simply observe that $(1,1)$ is a generator,
	\[
		\ip{(1,1)}=\bigl\{(1,1),(0,2),(1,0),(0,1),(1,2),(0,0)\bigr\}=\Z_2\times\Z_3
	\]
	The map $\phi(x)=(x,x)$ is therefore an isomorphism $\phi:\Z_6\cong\Z_2\times \Z_3$.

	\goodbreak

	\begin{enumerate}\setcounter{enumi}{1}
	% \item As observed above, $\Z_2\times\Z_2=\{(0,0),(0,1),(1,0),(1,1)\}$. is isomorphic to the Klein 4-group.
		\item If each $G_k$ is \emph{abelian} and \emph{written additively,} the direct product can instead be called the \emph{direct sum}
		\[
			\bigoplus_{k=1}^nG_k=G_1\oplus\cdots\oplus G_n
		\]
		We won't use this notation,\footnotemark\ though you've likely encountered it in linear algebra: the direct sum of $n$ copies of the real line $\R$ is the familiar vector space
		\[
			\R^n=\bigoplus_{i=1}^n\R=\R\oplus\cdots\oplus\R
		\]
	\end{enumerate}
\end{examples}

\footnotetext{%
	In these notes a direct product/sum will only ever have \emph{finitely many} terms, in which case the concepts are identical. There is a slight distinction when there are infinitely many factors.%
}


\boldsubsubsection{Orders of Elements in a Direct Product}

In Example \ref{ex:direct1}.1, we saw that the element $(1,1)\in\Z_2\times\Z_3$ had order 6 and thus generated the group. To help spot the pattern, consider another example.

\begin{example}{}{}
	What is the order of the element $(10,2)\in\Z_{12}\times \Z_8$? Recall Corollary \ref{cor:subscyclic}:
	\begin{itemize}\itemsep0pt
	  \item $10\in\Z_{12}$ has order $6=\frac{12}{\gcd(10,12)}$
	  \item $2\in\Z_8$ has order $4=\frac 8{\gcd(2,8)}$
	\end{itemize}
	If we repeatedly add $(10,2)$ to itself, then the \textcolor{red}{first co-ordinate resets after 6} summations, while the \textcolor{blue}{second resets after 4}. For \emph{both} to reset simultaneously, we need a \emph{common multiple} of \textcolor{red}{6} and \textcolor{blue}{4} summands. We can check this explicitly:
	\[
		\bigl\langle(10,2)\bigr\rangle = \bigl\{(10,2), (8,4), (6,6), (4,\textcolor{blue}{0}), (2,2), (\textcolor{red}{0},4), (10,6), (8,\textcolor{blue}{0}), (6,2), (4,4), (2,6), (\textcolor{red}{0},\textcolor{blue}{0})\bigr\}
	\]
	The order of the element $(10,2)$ is indeed the \emph{least common multiple} $12=\lcm(6,4)$.
\end{example}


\begin{thm}{}{directorder}
	Suppose $x_k\in G_k$ has order $r_k$. Then $(x_1,\ldots,x_n)\in \prod\limits_{k=1}^n G_k$ has order $\lcm(r_1,\ldots,r_n)$.
\end{thm}

\begin{proof}
	Simply appeal to Corollary \ref{cor:orderdefn}:
	\[
		(x_1,\ldots,x_n)^m=(x_1^m,\ldots,x_n^m)=(e_1,e_2,\ldots,e_n)\iff \forall k,\ x_k^m=e_k \iff \forall k,\ r_k\mid m
	\]
	The order is the minimal positive integer $m$ satisfying this, namely $m=\lcm(r_1,\ldots,r_n)$.
\end{proof}



\begin{example}{}{prodorder}
	Find the order of $(1,3,2,6)\in\Z_4\times\Z_7\times\Z_5\times\Z_{20}$.\smallbreak
	Again with reference to Corollary \ref{cor:subscyclic}, the element has order
	\[
		\lcm\left(\tfrac 4{\gcd(1,4)}, \tfrac 7{\gcd(3,7)}, \tfrac 5{\gcd(2,5)}, \tfrac{20}{\gcd(6,20)}\right) =\lcm(4,7,5,10)=140
	\]
\end{example}


\boldsubsubsection{When is a direct product of finite cyclic groups cyclic?}

Recall that $\Z_2\times\Z_2\cong V$ is non-cyclic while $\Z_2\times\Z_3\cong \Z_6$ is cyclic. It is reasonable to hypothesize that the distinction is whether the orders of the components are \emph{relatively prime.} 

\begin{cor}{}{prodprime}
	$\Z_m\times\Z_n$ is cyclic $\Longleftrightarrow \gcd(m,n)=1$, in which case $\Z_m\times\Z_n\cong\Z_{mn}$.\smallbreak
	More generally:
	\begin{itemize}\itemsep0pt
	  \item $\Z_{m_1}\times\cdots\times\Z_{m_k}\cong\Z_{m_1\cdots m_k} \Longleftrightarrow \forall i\neq j,\ \gcd(m_i,m_j)=1$.
	  \item If $n=p_1^{r_1}\cdots p_k^{r_k}$ is the prime factorization, then
	$\Z_n\cong\Z_{p_1^{r_1}}\times\cdots\times\Z_{p_k^{r_k}}$
	\end{itemize} 
\end{cor}

\begin{proof}
	We prove the first part; the generalization follows by induction.
	\begin{description}\itemsep0pt
		\item[\normalfont ($\Leftarrow$)] If $\gcd(m,n)=1$, then $(1,1)\in\Z_m\times\Z_n$ has order $\lcm(m,n)=\frac{mn}{\gcd(m,n)} =mn$. Hence $(1,1)$ is a generator of $\Z_m\times\Z_n$, which is therefore \emph{cyclic.}
		\item[\normalfont ($\Rightarrow$)] This is Exercise \ref{exs:corprodprime}.\qedhere
	\end{description}
\end{proof}


\begin{examples}{}{}
	\exstart (Example \ref{ex:prodorder})\lstsp The group $\Z_4\times\Z_7\times\Z_5\times\Z_{20}$ is non-cyclic since $\gcd(4,20)\neq 1$. Indeed the maximum order of an element in this group is
	\[
		\lcm(4,7,5,20)=140<2800=\nm{\Z_4\times\Z_7\times\Z_5\times\Z_{20}}
	\]
	\begin{enumerate}\setcounter{enumi}{1}
	  \item Is $\Z_5\times\Z_7\times\Z_{12}$ cyclic? The Corollary says yes, since no pair of 5, 7, 12 have common factors. It is ghastly to write, but there are 12 different ways (up to reordering) of expressing this group!
		\begin{align*}
			\Z_{420}\,&\cong\, \Z_3\times\Z_{140} \,\cong\, \Z_4\times\Z_{105} \,\cong\, \Z_5\times\Z_{84} \,\cong\, \Z_{7}\times\Z_{60}\\
			&\cong\, \Z_3\times\Z_4\times\Z_{35} \,\cong\, \Z_3\times\Z_5\times\Z_{28} \,\cong\, \Z_3\times\Z_7\times\Z_{20}\\
			&\cong\, \Z_4\times\Z_5\times\Z_{21} \,\cong\, \Z_4\times\Z_7\times\Z_{15} \,\cong\, \Z_5\times\Z_7\times\Z_{12}\\
			&\cong\, \Z_3\times\Z_4\times\Z_5\times\Z_7
		\end{align*}
		We may combine/permute the factors of $420=2^2\cdot 3\cdot 5\cdot 7$, provided we \emph{don't separate $2^2=4$.}
  
	%   \item Is $\Z_5\times\Z_{12}\times\Z_{43}$ cyclic? The Theorem says yes, since no pairs of the numbers 5, 12, 43 have any common factors. It is ghastly to write out, but there are 15 different ways (up to reordering) of expressing this group!
	% \begin{align*}
	% \Z_{2580}&\cong\Z_3\times\Z_{860} \cong\Z_4\times\Z_{645} \cong\Z_5\times\Z_{516} \cong\Z_{43}\times\Z_{60}\\
	% &\cong\Z_{12}\times\Z_{215} \cong\Z_{15}\times\Z_{172} \cong\Z_{20}\times\Z_{129}\\
	% &\cong\Z_3\times\Z_4\times\Z_{215} \cong\Z_3\times\Z_5\times\Z_{172} \cong\Z_3\times\Z_{20}\times\Z_{43}\\
	% &\cong\Z_4\times\Z_5\times\Z_{129} \cong\Z_4\times\Z_{15}\times\Z_{43} \cong\Z_5\times\Z_{12}\times\Z_{43}\\
	% &\cong\Z_3\times\Z_4\times\Z_5\times\Z_{43}
% \end{align*}
	\end{enumerate}
\end{examples}


\boldsubsubsection{The Fundamental Theorem}

We used the direct product to create finite abelian groups from cyclic building blocks. While we don't have the technology to prove it, our next result provides a powerful converse.

\begin{thm}{Fundamental Theorem of Finitely Generated Abelian Groups}{fund}\par
	Every finitely generated\footnotemark\ abelian group is isomorphic to a group of the form
	\[
		\Z_{p_1^{r_1}}\times\cdots\times\Z_{p_k^{r_k}}\times\Z\times\cdots\times\Z
	\]
	The $p_i$ are (not necessarily distinct) primes, each $r_j\in\N$, and there are finitely many $\Z$-factors.\par
	A finite abelian group has no $\Z$-factors.
\end{thm}

\footnotetext{Recall Exercise \ref*{sec:cyclicclass}.\ref{exs:finitegen}.}

%We won't develop the technology necessary to prove the Fundamental Theorem, but it is too useful to ignore. Our purpose is simply to classify \emph{finite abelian groups} up to isomorphism.

\goodbreak


\begin{examples}{}{}
	\exstart Up to isomorphism, there are five distinct abelian groups of order $81=3^4$:
	\[
		\Z_{81},\quad \Z_3\times\Z_{27},\quad \Z_9\times\Z_9,\quad \Z_3\times\Z_3\times\Z_9,\quad \Z_3\times\Z_3\times\Z_3\times\Z_3
	\]
	Such groups can often be distinguished by considering the orders of elements. For instance:
	\[
		\newlength\bob
		\settowidth\bob{All elements of $G$ have order $\le 27$}
		\left.
		\begin{array}{@{}p{\bob}r@{}}
			\text{$G$ is abelian of order 81} & \text{and,}\\
			\text{$G$ has an element of order 27} & \text{and,}\\
			\text{All elements of $G$ have order $\le 27$} &
		\end{array}
		\right\}
		\implies
		G\cong \Z_3\times\Z_{27}
	\]
	\begin{enumerate}\setcounter{enumi}{1}
	  \item Since $450=2\cdot 3^2\cdot 5^2$ is a prime factorization, the fundamental theorem says that every abelian group of order 450 is isomorphic to one of four groups:
	\begin{enumeratea}\itemsep0pt
	  \item $\Z_2\times\Z_{3^2}\times\Z_{5^2}\cong \Z_{450}$\hfill (cyclic, maximum order of an element 450)
	  \item $\Z_2\times\Z_3\times\Z_3\times\Z_{5^2}$\hfill (non-cyclic, maximum order $150=2\cdot 3\cdot 5^2$)
	  \item $\Z_2\times\Z_{3^2}\times\Z_5\times\Z_5$\hfill (non-cyclic, maximum order $90=2\cdot 3^2\cdot 5$)
	  \item $\Z_2\times\Z_3\times\Z_3\times\Z_5\times\Z_5$\hfill (non-cyclic, maximum order $30=2\cdot 3\cdot 5$)
	\end{enumeratea}
	As previously, there are multiple isomorphic ways to express each group as a direct product.
	\end{enumerate}
\end{examples}



We finish by listing, up to isomorphism, all groups of orders $\le 15$ and abelian groups of order 16. %The Fundamental Theorem gives us all the abelian groups.\par
\begin{center}\label{pg:fundabel}
	\begin{tabular}{|l|l|l|}
		\hline
		order & abelian & non-abelian\\
		\hline
		1 & $\Z_1$ & \\
		2 & $\Z_2$ & \\
		3 & $\Z_3$ & \\
		4 & $\Z_4$,\ \ $V\cong\Z_2\times\Z_2$ & \\
		\hline
		5 & $\Z_5$ & \\
		6 & $\Z_6\cong\Z_2\times\Z_3$ & $D_3\cong S_3$\\
		7 & $\Z_7$ & \\
		8 & $\Z_8$,\ \ $\Z_2\times\Z_4$,\ \ $\Z_2\times\Z_2\times\Z_2$ & $D_4$,\ \ $Q_8$\\
		\hline
		9 & $\Z_9$,\ \ $\Z_3\times\Z_3$ & \\
		10 & $\Z_{10}\cong\Z_2\times\Z_5$ & $D_5$\\
		11 & $\Z_{11}$ & \\
		12 & $\Z_{12}\cong\Z_3\times\Z_4$,\ \ $\Z_2\times\Z_6\cong\Z_2\times\Z_2\times\Z_3$ & $D_6$,\ \ $A_4$,\ \ $Q_{12}$\\
		\hline
		13 & $\Z_{13}$ & \\
		14 & $\Z_{14}\cong\Z_2\times\Z_7$ & $D_7$\\
		15 & $\Z_{15}\cong\Z_3\times\Z_5$ & \\
		16 & $\Z_{16}$,\ \ $\Z_4\times\Z_4$,\ \ $\Z_2\times\Z_8$,\ \ $\Z_2\times\Z_2\times\Z_4$,\ \ $\Z_2\times\Z_2\times\Z_2\times\Z_2$ & Many\\
		\hline
	\end{tabular}
\end{center}

The list of non-abelian groups contains some unfamiliarity though we've met most already:
\begin{itemize}\itemsep0pt
  \item The dihedral groups $D_n$ are the rotations/reflections of a regular $n$-gon (Definition \ref{defn:rotdiklein}).
  \item $S_3$ is in the introduction, though it and $A_4$ will be described properly in Chapter \ref{chap:perm}.
  \item $Q_8$ is the quaternion group (Exercise \ref*{sec:subgroup}.\ref{exs:quaternion}). The generalized quaternion group $Q_{12}$ is related.
\end{itemize}
There are \emph{nine} non-isomorphic non-abelian groups of order 16: $D_8$ and the direct product $\Z_2\times Q_8$ being explicit examples. You might suspect from the table that all non-abelian groups have even order: this is not so, though the smallest counter-example has order 21.


\begin{tcolorbox}[exercisestyle,title={Exercises \thesection.\phantom{b}}]
Key concepts:
\begin{quote}
	\emph{Direct product \qquad Order of element via lcm \qquad Cyclic/gcd criteria\qquad Fundamental theorem}
\end{quote}

\begin{enumerate}
	\item List the elements of the following direct product groups:
	\begin{enumerate}
  	\item $\Z_2\times\Z_4$\qquad\qquad
  	(b)\lstsp $\Z_3\times\Z_3$\qquad\qquad
  	(c)\lstsp $\Z_2\times\Z_2\times\Z_2$
	\end{enumerate}
	
  \item Prove Theorem \ref{thm:directprod} by checking each of the axioms of a group.

	\item Prove that $G\times H\cong H\times G$.
	
	\item Prove that a direct product $\prod G_k$ is abelian if and only if its components $G_k$ are all abelian.
	
	\item Find the orders of the following elements and write down the cyclic subgroups generated by each (list all of the elements explicitly):
	\begin{enumerate}
  	\item $(1,3)\in\Z_2\times\Z_4$\qquad\qquad
  	(b)\lstsp $(4,2,1)\in\Z_6\times\Z_4\times\Z_3$
	\end{enumerate}
	
	\item Is the group $\Z_{12}\times \Z_{27}\times\Z_{125}$ cyclic? Explain.

	\item Find a generator of the group $\Z_3\times\Z_4$ and hence define an isomorphism $\phi:\Z_{12}\cong\Z_3\times\Z_4$.\par
	(\emph{Hint: read the proof of Corollary \ref{cor:prodprime}})


	\item State three non-isomorphic groups of order 50.
	
	\item Suppose $p,q$ are distinct primes. Up to isomorphism, how many abelian groups are there of order $p^2q^2$?
	
	\item\label{exs:corprodprime} Complete the proof of Corollary \ref{cor:prodprime}: if $\Z_m\times\Z_n$ is cyclic, then $\gcd(m,n)=1$.\par
	(\emph{Hint: if $\gcd(m,n)\ge 2$, what is the maximum order of an element in $\Z_m\times\Z_n$?})

	\item Suppose $G$ is an abelian group of order $m$, where $m$ is a square-free positive integer ($\nexists k\in \Z_{\ge 2}$ such that $k^2\!\mid\! m$). Prove that $G$ is cyclic.


	\item\begin{enumerate}
  	\item Let $G$ be a finitely generated abelian group and let $H$ be the subset of $G$ consisting of the identity $e$ together with all the elements of order 2 in $G$. Prove that $H$ is a subgroup of $G$.
  	\item In the language of the Fundamental Theorem, to which direct product is $H$ isomorphic?
	\end{enumerate}
	
	
	\item\label{exs:abeliansubgroup} Suppose $G$ is a finite abelian group and that $m$ is a divisor of $\nm G$. Prove that $G$ has a subgroup of order $m$.\par
	(\emph{Hint: use the prime decomposition of $m$ and the fundamental theorem to identify a suitable subgroup of $\Z_{p_1^{r_1}}\times\cdot\times\Z_{p_k^{r_k}}$ })
	
	
	\item Give a simple explanation as to why $D_8$ is not isomorphic to $\Z_2\times Q_8$.
	
% 	
% 	\item Suppose $G$ is an abelian group and let $T\subseteq G$ be the subset of elements with \emph{finite order.} Prove that $T$ is a subgroup of $G$.\par
% 	(\emph{The Fundamental Theorem almost makes this easy, but it doesn't apply---why not?}) 

\end{enumerate}
\end{tcolorbox}
