\graphicspath{{7homo/asy/}}

\section{Homomorphisms and the First Isomorphism Theorem}\label{chap:homo}

In this chapter we further discuss homomorphisms. Of particular importance is the relationship between normal subgroups, homomorphisms and factor groups, which will lead us (in the next section) to a superior method for identifying factor groups.\smallbreak
Unless otherwise stated, all homomorphisms in this chapter are between \emph{groups.}

\subsection{Kernels and Images}\label{sec:kerimage}

\begin{defn}{}{kernel}
	Let $\phi:G\to L$ be a homomorphism. The \emph{kernel} and \emph{image} (or \emph{range}) of $\phi$ are the sets
	\[
		\ker\phi=\bigl\{g\in G:\phi(g)=e_L\bigr\}
		\qquad\qquad
		\image\phi=\bigl\{\phi(g):g\in G\bigr\}
	\]
\end{defn}

The image is sometimes denoted $\phi(G)$. Note that $\ker\phi\subseteq G$ while $\image\phi\subseteq L$.

\begin{examples}{}{}
	\exstart $\phi(x)=2x\pmod 4$ defines a homomorphism $\phi:\Z\to\Z_4$, with
	\[
		\ker\phi=\{x\in\Z:2x\equiv 0\negthickspace\pmod 4\}=2\Z,
		\quad
		\image\phi=\{0,2\}
	\]
	\begin{enumerate}\setcounter{enumi}{1}
	  \item The kernel should feel familiar from linear algebra: if $\rT:V\to W$ is a linear map between vector spaces, then the kernel is simply the \emph{nullspace}
		\[
			\ker\rT=\{\vv\in V:\rT(\vv)=\V0\}
		\]
		Moreover, if $\rT=\rL_A:M_n(\R)\to M_m(\R)$ is left-multiplication by a matrix $A$, then $\image\rT$ is the \emph{column space} of $A$.
	\end{enumerate}
\end{examples}


\begin{lemm}{}{homosub}
	Let $\phi:G\to L$ be a homomorphism. Then,
	\begin{enumerate}
	  \item $\phi(e_G)=e_L$\hfill ($\phi$ maps identity to identity)
	  \item $\forall g\in G$, \ $\bigl(\phi(g)\bigr)^{-1}=\phi(g^{-1})$\hfill($\phi$ maps inverses to inverses)
	  \item $\ker\phi\triangleleft G$\hfill($\ker\phi$ is a normal subgroup of $G$)
	  \item $\image\phi\le L$\hfill($\image\phi$ is a subgroup of $L$)
	\end{enumerate}
\end{lemm}

\begin{proof}
	1 \& 2 were in Exercise \ref*{sec:morph}.\ref{exs:structural1} and we leave 4 as an exercise. We prove 3 explicitly.
	\begin{enumerate}\setcounter{enumi}{2}
	%   \item Suppose that $g\in G$. Then
	%   \[\phi(g)=\phi(ge_G)=\phi(g)\phi(e_G)\implies e_L=\phi(e_G)\]
	%   by the left cancellation law.
	%   \item Suppose that $g\in G$. Then
	%   \[e_L=\phi(e_G)=\phi(gg^{-1})=\phi(g)\phi(g^{-1})\implies (\phi(g))^{-1}=\phi(g^{-1})\]
	  \item Suppose $k_1,k_2\in\ker\phi$. Then
	  \begin{gather*}
		  \phi(k_1k_2)=\phi(k_1)\phi(k_2)=e_L\implies k_1k_2\in\ker\phi\\
		  \phi(k_1^{-1})=\bigl(\phi(k_1)\bigr)^{-1}=e_L\implies k_1^{-1}\in\ker\phi
	  \end{gather*}
	  It follows that $\ker\phi$ is a subgroup of $G$.\par
	  To see that $\ker\phi$ is normal, recall Corollary \ref{cor:normalconj}: if $g\in G$ and $k\in\ker\phi$, then
	  \[
	  	\phi(gk g^{-1})=\phi(g)\phi(k)\phi(g)^{-1}
	  	=\phi(g)\phi(g)^{-1}=e_L
	  	\implies gk g^{-1}\in\ker\phi
	  	\tag*{\qedhere}
	  \]
	\end{enumerate}
\end{proof}


\goodbreak


\begin{examples}{}{}
	\exstart For the homomorphism $\phi:\Z\to\Z_4:x\mapsto 2x$, we see that $\ker\phi=2\Z$ is a normal subgroup of $\Z$, and $\image\phi=\{0,2\}=\ip{2}$ a subgroup of $\Z_4$. 
	\begin{enumerate}\setcounter{enumi}{1}
	  \item The nullspace of a linear map $\rT:V\to W$ is indeed a \emph{subspace} and thus a subgroup $\ker\rT\le V$: since $V$ is abelian, this is a normal subgroup. Moreover, $\image\rT$ is also a subspace/group of $W$.
	  
		\item $\det:\rGL_n(\R)\to\R^\times$ is a homomorphism, whence we obtain a normal subgroup
		\[
			\ker\det=\rSL_n(\R)=\{A\in\rGL_n(\R):\det A=1\}
			\triangleleft\rGL_n(\R)
		\]
		
		\item $\phi:\Z\to\Z_{20}:x\mapsto 4x\pmod{20}$ is a homomorphism, as may be checked:
		\[
			\phi(x+y)=4(x+y)=4x+4y
			=\phi(x)+\phi(y)\in\Z_{20}
		\]
		Its kernel and image are $\ker\phi=5\Z\le\Z$ and $\image\phi=\ip{4}=\{0,4,8,12,16\}\le\Z_{20}$.
	\end{enumerate}
\end{examples}

Since every kernel is a normal subgroup, it is worth identifying the distinct cosets with a view to describing the factor group $\quotient G{\ker\phi}$.

\begin{lemm}{}{homobij}
	Let $\phi:G\to L$ be a homomorphism. Then
	\[
		g_1\ker\phi=g_2\ker\phi
		\iff \phi(g_1)=\phi(g_2)
	\]
\end{lemm}

There is precisely one coset of $\ker\phi$ for each element of $\image\phi$; otherwise said $(G:\ker\phi)=\nm{\image\phi}$.

\begin{proof}
	For all $g_1,g_2\in G$, we have
	\begin{align*}
		g_1\ker\phi=g_2\ker\phi&\iff g_2^{-1}g_1\in\ker\phi \tag{Theorem \ref{thm:cosetpart}}\\
		&\iff\phi(g_2^{-1}g_1)=e_L\tag{Definition \ref{defn:kernel}}\\
		&\iff \phi(g_2)^{-1}\phi(g_1)=e_L\tag{Lemma \ref{lemm:homosub}}\\
		&\iff \phi(g_1)=\phi(g_2)\tag*{\qedhere}
	\end{align*}
\end{proof}

We'll extend this idea shortly; for the moment we use it to aid in finding homomorphisms.

\begin{thm}{}{}
	Let $\phi:G\to L$ be a homomorphism. If $G$ is finite, then $\image\phi$ is a finite group whose order divides that of $G$. The same holds for $L$. Otherwise said:
	\[
		\nm G<\infty
		\implies\nm{\image\phi}\Bigm| \nm G
		\quad\text{and}\quad 
		\nm L<\infty\implies\nm{\image\phi}\Bigm| \nm L
	\]
	If both groups are finite, then $\smash[t]{\nm{\image\phi}\Bigm|\gcd\bigl(\nm G,\nm L)}$.
	% \begin{enumerate}
	%   \item If $G$ is finite, then $\image\phi$ is a finite subgroup of $L$ whose order divides that of $G$. More succinctly:
	% 	\[\nm G<\infty\implies\nm{\image\phi}\Bigm| \nm G\]
	%   \item Similarly, $\nm L<\infty\implies\nm{\image\phi}\Bigm| \nm L$.
	% \end{enumerate}
\end{thm}

\begin{proof}
	If $G$ is a finite group, then $\ker\phi\le G$ is finite. Apply Lemma \ref{lemm:homobij} to see that
	\[
		\nm{\image\phi}=(G:\ker\phi) =\smash[b]{\frac{\nm G}{\ker\phi}}
	\]
	is a divisor of $\nm G$. The second situation $\nm{\image\phi}\Bigm|\nm{L}$ is Lagrange's Theorem (\ref{thm:lagrange}).
\end{proof}


\goodbreak


\begin{examples}{}{}
	\exstart If $\phi:\Z_{17}\to\Z_{13}$ is a homomorphism, then $\gcd(17,13)=1\Longrightarrow\nm{\image\phi}=1$. Thus $\image\phi=\{0\}$ is the trivial subgroup of $\Z_{13}$, and $\phi:x\mapsto 0\ (\forall x\in\Z_{17})$ the trivial homomorphism.\vspace{-5pt}
	\begin{enumerate}\setcounter{enumi}{1}
	  \item[] More generally, if $\gcd\bigl(\nm G,\nm L\bigr)=1$, then the trivial homomorphism $\phi:g\mapsto e_L$ is the unique homomorphism $\phi:G\to L$.

	  \item We describe all homomorphisms $\phi:\Z_4\to S_3$.\par
	  Since $\Z_4$ is cyclic, we need only describe what $\phi$ does to a generator (1) to obtain the entire homomorphism: $\phi(x)=\bigl(\phi(1)\bigr)^x$. There are \emph{six} choices for $\phi(1)\in S_3$, though not all will work.\par
	  The Theorem says that $\nm{\image\phi}=1$ or 2; the only common divisors of $4=\nm{\Z_4}$ and $6=\nm{S_3}$.\par
	  If $\image\phi$ has one element, we obtain the trivial homomorphism $\phi_0(x)=e,\ \forall x\in\Z_4$.\par
	  If $\nm{\image\phi}=2$, then $\image\phi$ is a subgroup of order 2, of which $S_3$ has exactly three: $\{e,(2\,3)\}$, $\{e,(1\,3)\}$, $\{e,(1\,2)\}$. This results in three further homomorphisms
	  \[
	  	\phi_1(x)=(2\,3)^x,\quad 
	  	\phi_2(x)=(1\,3)^x,\quad
	  	\phi_3(x)=(1\,2)^x
	  \]
	  %for a grand total of four distinct homomorphisms $\phi:\Z_4\to S_3$.
	\end{enumerate}
\end{examples}

We now turn to the general question of homomorphisms between finite cyclic groups $\phi:\Z_m\to\Z_n$. Two facts make this relatively simple:
\begin{enumerate}
  \item As above, it is enough to define $\phi(1)$, for then $\phi(x)=\phi(1)+\cdots+\phi(1)=\phi(1)\cdot x$.
  \item $\nm{\image\phi}$ must divide $d=\gcd(m,n)$. Since $\Z_n$ has exactly one subgroup of each order dividing $n$ (Corollary \ref{cor:subscyclic}), $\image\phi$ must be a subgroup of the \emph{unique} subgroup of $\Z_n$ of order $d$:
	\[
		\image\phi\le\ip{\frac nd}
		=\left\{0,\frac nd,\frac{2n}d,\ldots,\frac{(d-1)n}d\right\}
	\]
	It is therefore enough to let $\phi(1)$ be each element of this group in turn\ldots
\end{enumerate} 

\begin{cor}{}{homoszmn}
	There are $d=\gcd(m,n)$ distinct homomorphisms $\phi:\Z_m\to\Z_n$, namely
	\[
		\phi_k(x)=\frac{kn}{d}x
		\quad\text{where}\quad 
		k=0,\ldots,d-1
	\]
\end{cor}

\begin{proof}
	Following the above, it remains only to check that each $\phi_k$ is a well-defined function. For this, note first that $x=y\in\Z_m\Longleftrightarrow y=x+\lambda m$ for some $m\in\Z$, from which
	\[
	\phi_k(y)=\phi_k(x+\lambda m)
	=\frac{kn}d(x+\lambda m) 
	=\frac{kn}dx+\lambda k\textcolor{blue}{\frac md} n
	=\frac{kn}dx =\phi_k(x)
	\tag{in $\Z_n$}
	\]
	where we used the fact that $\textcolor{blue}{\frac md}$ is an \emph{integer.}
\end{proof}


\goodbreak


\begin{example}{}{homoz12-20}
	We describe all homomorphisms $\phi:\Z_{12}\to\Z_{20}$.\smallbreak
	Since $\gcd(12,20)=4$, we see that $\image\phi\le\ip{5}=\{0,5,10,15\}\le\Z_{20}$. There are four choices:
	\[
		\phi_0(x)=0,\qquad 
		\phi_1(x)=5x,\qquad 
		\phi_2(x)=10x,\qquad 
		\phi_3(x)=15x\pmod{20}
	\]
	We similarly see that there are four distinct homomorphisms $\psi:\Z_{20}\to\Z_{12}$:
	\[
		\psi_0(x)=0,\qquad 
		\psi_1(x)=3x,\qquad 
		\psi_2(x)=6x,\qquad 
		\psi_3(x)=9x\pmod{12}
	\]
\end{example} 

\goodbreak

\begin{exercises}{}{}
	Key concepts:
	\begin{quote}
		\emph{Image\qquad kernels are normal subgroups\qquad 
		$(G:\ker\phi)=\nm{\image\phi}$\qquad
		$\nm{\image\phi}\Bigm|\gcd\bigl(\nm G,\nm L\bigr)$}
	\end{quote}
	
	
	\begin{enumerate}
	  \item Check that you have a homomorphism (use Corollary \ref{cor:homoszmn}) and compute its kernel and image.
	  \begin{enumerate}
	    \item $\phi:\Z_{8}\to\Z_{14}$ defined by $\phi(x)=7x\pmod{14}$.
	    
	    \item $\phi:\Z_{36}\to\Z_{20}$ defined by $\phi(x)=5x\pmod{20}$.
		\end{enumerate}
	
		
		\item Describe all homomorphisms between the groups:
		\begin{enumerate}
		  \item \makebox[160pt][l]{$\phi:\Z_{15}\to\Z_{80}$\hfill (b)}\lstsp $\phi:\Z\to\Z_3$ 
		  \item[(c)] \makebox[160pt][l]{$\phi:\Z_6\to D_4$\hfill (d)}\lstsp $\phi:\Z_{15}\to A_4$
		\end{enumerate}
		
			
		\item Find the kernel and image of each homomorphism:
		\begin{enumerate}
		  \item The \emph{trace} of a matrix $\tr:M_2(\R)\to\R:
		  \smash{
			  \begin{smatrix}
					a&b\\c&d
			  \end{smatrix}
		  }
		  \mapsto a+d$.
			
			\item $\rT:\R^3\to\R^4:\vx\mapsto 
			\begin{smatrix}
				1&1&-1\\
				0&3&-1\\
				1&4&-2\\
				2&5&-3
			\end{smatrix}$
		\end{enumerate}
		
		
		\item Explain why the map $\phi$ is a homomorphism and find $\ker\phi$:
		\[
			\phi:S_n\to \bigl(\{1,-1\},\cdot\bigr):\sigma \mapsto 
			\begin{cases}
	  		1&\text{ if $\sigma$ even}\\
	  		-1&\text{ if $\sigma$ odd}
			\end{cases}
		\]
	
	
		\item\begin{enumerate}
		  \item Prove Part 4 of Lemma \ref{lemm:homosub}: if $\phi:G\to L$ is a homomorphism, then $\image\phi\le L$.
		 	\item If $H\le G$ and $\phi:G\to L$ a homomorphism, prove that $\phi(H):=\{\phi(h):h\in H\}\le\image\phi$.
		  \item Give an example to show that $\image\phi$ need not be a normal subgroup of $L$.
		\end{enumerate}
	
		
		\item\label{exs:totient} Prove that the number of distinct \emph{isomorphisms} $\phi:\Z_n\to\Z_n$ equals the cardinality of the group of units in $\Z_n$ (see Exercise \ref*{sec:cyclicclass}.\ref{exs:znmult}))
		\[
			\nm{\Z_n^\times}=\nm{\{x\in\Z_n:\gcd(x,n)=1\}}
		\]
			
			  
	  \item Prove that $\phi:\Z_m\times\Z_n\to\Z_m\times\Z_n$ is a well-defined homomorphism if and only if there exist integers $a,b,c,d$ for which
	  \[
	  	\phi(x,y)=\bigl(ax+by,cx+dy\bigr),\quad 
	  	m\mid bn\quad
	  	\text{and}\quad n\mid cm
	  \]
	  (\emph{Hint: let $(a,c)=\phi(1,0)$, etc.})
	  
	
		\item Find all homomorphisms $\phi:\Z_2\times\Z_7\to\Z_2\times\Z_5$. How do you know that there are no more?
		
		  
	  \item Consider $\phi:D_4\to D_4:\sigma\mapsto \sigma^2$. Show that $\phi$ is \emph{not} a homomorphism.
		
	\end{enumerate}
\end{exercises}

\clearpage



\subsection{The First Isomorphism Theorem}\label{sec:1stiso}

We've seen that all kernels of group homomorphisms are normal subgroups. In fact \emph{all} normal subgroups are the kernel of some homomorphism.

\begin{thm}{Canonical Homomorphism}{}
	Let $G$ be a group and $H\triangleleft G$. Then the function
	\[
		\gamma:G\to \quotient GH
		\quad\text{defined by}\quad 
		\gamma(g)=gH
	\]
	is a homomorphism with $\ker\gamma=H$.
\end{thm}

\begin{proof}
	Since $H$ is normal, $\quotient GH$ is a group. By the definition of multiplication in $\quotient GH$,
	\[
		\gamma(g_1)\gamma(g_2)
		=g_1H\cdot g_2H
		=(g_1g_2)H
		=\gamma(g_1g_2)
	\]
	whence $\gamma$ is a group homomorphism. Moreover, the identity in the factor group is $H$, whence
	\[
		\ker\gamma=\{g\in G:\gamma(g)=H\} 
		=\{g\in G:gH=H\}=H
		\tag*{\qedhere}
	\]
\end{proof}

This might feel a little sneaky and unsatisfying; we'd perhaps have preferred a homomorphism that doesn't have a factor group as its image! However, a companion result says that, among all homomorphisms with the same kernel, $\gamma$'s appearance is unavoidable.

\begin{thm}{1\st{} Isomorphism Thm}{}
	Let $\phi:G\to L$ be a homomorphism with kernel $H$. Then
	\[
		\mu:\quotient GH\to\image\phi
		\quad \text{defined by}\quad 
		\mu(gH)=\phi(g)
	\]
	is an isomorphism. Otherwise said, $\quotient G{\ker\phi}\cong\image\phi$.
\end{thm}


\begin{minipage}[t]{0.69\linewidth}\vspace{-5pt}
	These results may be summarized in a \emph{commutative diagram:} a homomorphism $\phi:G\to L$ \emph{factors} as $\phi=\mu\circ\gamma$, where $\gamma$ is the canonical homomorphism with kernel $\ker\phi$. This theorem is very important, and has analogues in several other parts of mathematics: e.g., the rank--nullity theorem from linear algebra is of close kin.
\end{minipage}
\hfill
\begin{minipage}[t]{0.3\linewidth}\vspace{-5pt}
	\flushright$\xymatrix{%
		G \ar[rr]^{\phi} \ar[dr]_\gamma && \image\phi\\
		& \quotient G{\ker\phi} \ar[ur]_{\mu} &
	}$
\end{minipage}



\begin{proof}
	The factor group exists since $\ker\phi\triangleleft G$ (Lemma \ref{lemm:homosub}). We check the isomorphism properties:
  \begin{description}
		\item[\normalfont\emph{Well-definition and Bijectivity}:] These are immediate from Lemma \ref{lemm:homobij} after writing $H=\ker\phi$:
		\[
			g_1H=g_2H
			\iff \phi(g_1)=\phi(g_2)
			\iff \mu(g_1H)=\mu(g_2H)
		\]
		\item[\normalfont\emph{Homomorphism}:] For all $g_1H,g_2H\in\quotient GH$,
		\begin{align*}
			\mu(g_1H\cdot g_2H)
			&=\mu(g_1g_2H)=\phi(g_1g_2)
				\tag{definition of $\mu$}\\ 
				&=\phi(g_1)\phi(g_2)
				\tag{$\phi$ is a homomorphism}\\
			&=\mu(g_1H)\mu(g_2H)
		\end{align*}
  \end{description}
	We conclude that $\mu$ is an isomorphism.
\end{proof}


\begin{examples}{}{}
	\exstart Let $\phi:\Z_{12}\to\Z_{20}$ be the homomorphism $\phi(x)=5x\!\!\pmod{20}$ (Example \ref{ex:homoz12-20}). Its kernel and image are
	\begin{gather*}
		\ker\phi
		=\bigl\{x\in\Z_{12}:5x\equiv 0\negthickspace\pmod{20}\bigr\}
		=\{0,4,8\}
		=\ip 4\le \Z_{12}\\
		\image\phi=\{5x\in\Z_{20}:x\in\Z_{12}\}
		=\{0,5,10,15\}
		=\ip{5}\le\Z_{20}
	\end{gather*}
	The relevant factor group is
	\begin{align*}
		\quotient{\Z_{12}}{\ker\phi}
		&=\Bigl\{\{0,4,8\},\{1,5,9\},\{2,6,10\},\{3,7,11\}\Bigr\}
		=\bigl\{\ip 4,1+\ip 4,2+\ip 4,3+\ip 4\bigr\}
	\end{align*}
	The canonical homomorphism $\gamma$ and the isomorphism $\mu$ are\vspace{-5pt}
	\begin{enumerate}\setcounter{enumi}{1}
	  \item[]\begin{quote}
			$\SelectTips{cm}{}\xymatrix @C40pt @R0pt{%
				\rlap{$\gamma(x)=x+\ip 4$} &&& \Z_{12} \ar@{->}[r]_-{\textstyle\gamma} \ar@{->}@/^15pt/[rr]^{\textstyle\phi} & \quotient{\Z_{12}}{\ip 4} \ar@{->}[r]_-{\textstyle\mu} & \image\phi\\
				\rlap{$\mu\bigl(x+\ip 4\bigr)=5x$} &&& x \ar@{|->}[r] & x+\ip 4 \ar@{|->}[r] & 5x
			}$
		\end{quote}
	  
	\item (Example \ref{ex:normalmoreex}.1)\lstsp $\phi:\R\to(\C^\times,\cdot):x\mapsto e^{ix}$ is a homomorphism with
  \[
  	\ker\phi=\{x\in\R:e^{ix}=1\}=\ip{2\pi}=2\pi\Z\le\R
  	\quad\text{and}\quad 
  	\image\phi=S^1
  	\tag{$S^1$ is the circle group}
  \]
  The canonical homomorphism $\gamma$ and the isomorphism $\mu$ from the theorem are
  \[
  	\gamma:\R\to\quotient{\R}{\ip{2\pi}}:x\mapsto x+\ip{2\pi}
  	\quad\text{and}\quad
  	\mu:\quotient\R{\ip{2\pi}}\to S^1:x+\ip{2\pi}\mapsto e^{ix}
  \]
  Note that $\mu$ is precisely the isomorphism we saw previously!
  
  \item $\phi:\Z\times\Z\to\Z:(x,y)\mapsto 3x-2y$ is a homomorphism. Moreover
  \[
  	\phi(x,y)=(0,0)\iff 3x=2y\iff (x,y)=(2n,3n)\text{ for some $n\in\Z$}
  \]
  We conclude that $\ker\phi=\ip{(2,3)}$. The canonical homomorphism is
  \[
  	\gamma:\Z\times\Z\to
  	\quotient{\Z\times\Z}{\ip{(2,3)}}:
  	(x,y)\mapsto (x,y)+\ip{(2,3)}
  \]
  Since $\phi$ is surjective (e.g., $n=\phi(n,n)$), the 1\st{} isomorphism theorem tells us that
  \[
  	\quotient{\Z\times\Z}{\ip{(2,3)}}\cong\Z
  	\quad\text{via}\quad 
  	\mu\bigl((x,y)+\ip{(2,3)}\bigr)=3x-2y
  \]
  
  \item $\phi(x,y)=(x,y-x)$ is a well-defined homomorphism $\phi:\Z\times\Z_6\to\Z_3\times\Z_6$. Moreover,
  \begin{gather*}
  	\phi(x,y)=(0,0)\iff
  	\begin{cases}
  		x=3k,\text{ and}\\
  		y=x=3k\in\Z_6
  	\end{cases}\\
  	\text{whence}\quad
  	\ker\phi=\ip{(3,3)}=\bigl\{\ldots,(-6,0),(-3,3),(0,0),(3,3),(6,0),\ldots\bigr\}
  \end{gather*}
  Plainly $\phi$ is surjective ($(m,n)=\phi(m,n+m)$); we conclude that $\quotient{\Z\times\Z_6}{\ip{(3,3)}}\cong\Z_3\times\Z_6$ via the isomorphism $\mu\bigl((x,y)+\ip{(3,3)})=(x,y-x)$.
	\end{enumerate}
\end{examples}

\vfil

\goodbreak


With a little creativity, the 1\st{} isomorphism theorem may be applied to the identification of factor groups: given $H\triangleleft G$, cook up a homomorphism $\phi:G\to L$ with $\ker\phi=H$, then $\quotient GH\cong\image\phi$. We revisit some examples from the previous section in this context.

\begin{examples*}{\ref{ex:factorfiniteabelian},\,mk.II}{}
	Let $G=\Z_4\times\Z_8$. For each subgroup $H$, we describe a homomorphism $\phi:G\to L$ with $\ker\phi =H$. While there are often several possible choices for $\phi$, ours will line up with what we saw in the original incarnations. Hopefully you'll feel that the reasons for our choices are independent of earlier discussions. %Note that if $\phi(1,0)=(a,c)$ and $\phi(0,1)=(b,d)$, then
	%\[\phi(x,y)=(ax+by,cx+dy)\]
	\begin{enumerate}
	  \item Given $H=\ip{(0,1)}$, we need a homomorphism where $\phi(0,1)$ is the identity. A simple way to do this is to ignore $y$ and define
	  \[
	  	\phi:\Z_4\times\Z_8\to\Z_4:(x,y)\mapsto x
	  \]
	  This indeed has kernel $\ker\phi=\{(0,y):y\in\Z_8\} =H$, whence
	  \[
	  	\quotient GH\cong\image\phi=\Z_4
	  \]
	  via the isomorphism $\mu:(x,y)+H\mapsto x$.\par
	  To connect back to the original problem, note that $(x,y)+H=(x,0)+H$ for all $x,y$, whence $\mu$ is the \emph{inverse} of the isomorphism $\psi:x\mapsto (x,0)+H$ stated previously!
	
	  \item Given $H=\ip{(0,2)}$ we require $\phi(0,2)$ to be the identity. This may be achieved by taking $y$ modulo 2 and defining
	  \[
	  	\phi:\Z_4\times\Z_8\to\Z_4\times\Z_2:(x,y)\mapsto (x,y)
	  \]
	  This is a well-defined ($\textcolor{red}{\phi(x+4m,y+8n)=\phi(x,y)}$, since $2\mid 8$) homomorphism with the required kernel $=H$. Moreover, $\phi$ is surjective, whence
	  \[
	  	\quotient GH\cong \image\phi= \Z_4\times\Z_2
	  \]
	  via the isomorphism $\mu\bigl((x,y)+H\bigr) =(x,y)$. Once again $\mu$ is the inverse of $\psi(x,y)=(x,y)+H$ in the original example.
	  
	  \item For the last version, finding a homomorphism with kernel $H=\ip{(2,4)}=\{(0,0),(2,4)\}$, is somewhat trickier. One approach is to observe that
	  \[
	  	(x,y)\in H\iff x\equiv 0\negthickspace\pmod 2\ \text{ and }\ y-2x\equiv 0\negthickspace\pmod 8
	  \]
	  We therefore choose
	  \[
	 		\phi:\Z_4\times\Z_8\to\Z_2\times\Z_8:(x,y)\mapsto\bigl(x,y-2x\bigr)
	 	\]
	  It is worth checking that this is \textcolor{red}{well-defined}: the $\textcolor{red}{2}x$ in the second factor is crucial! Certainly $\phi$ has the correct kernel. It is moreover surjective, e.g.\ $(m,n)=\phi(m,n+2m)$, whence
	  \[
	  	\quotient GH\cong\image\phi=\Z_2\times\Z_8
	  \]
	  via the isomorphism $\mu\bigl((x,y)+H\bigr)=(x,y-2x)$.
	\end{enumerate}
\end{examples*}


\goodbreak

\begin{exercises}{}{}
	Key concepts:
	\begin{quote}
		\emph{Canonical homomorphism $\gamma:G\to\quotient GH$\qquad 1\st{} isomorphism theorem $\mu:\quotient G{\ker\phi}\cong\image\phi$}
	\end{quote}
	
	\begin{enumerate}
	  \item Let $\phi:\Z_{18}\to\Z_{12}$ be the homomorphism $\phi(x)=10x$. 
	  \begin{enumerate}
	    \item Find the kernel of and image of $\phi$.
	    \item List the elements of the factor group $\smash[b]{\quotient{\Z_{18}}{\ker\phi}}$.
	    \item State an explicit isomorphism $\mu:\quotient{\Z_{18}}{\ker\phi}\to\image\phi$.
	    \item To what basic group $\Z_n$ is the factor group isomorphic?
	  \end{enumerate}
	  
	 	
	 	\item Repeat the previous question for the homomorphism $\phi:\Z\to\Z_{20}:x\mapsto 8x$.
	  
	  
	  \item For each function $\phi:\Z\times\Z\to\Z$, find the kernel and identify the factor group $\smash{\quotient{\Z\times\Z}{\ker\phi}}$.
	  \begin{enumerate}
	    \item $\phi(x,y)=3x+y$\qquad\qquad (b)\lstsp $\phi(x,y)=2x-4y$
	  \end{enumerate}  
	  
	    
	  \item\begin{enumerate}
	  	\item If a subgroup $H$ of $G=\Z_{15}\times\Z_3$ has order 5, find its elements.
			\item Show that $\phi(x,y)=(x,y)$ is a homomorphism $\phi:G\to\Z_3\times\Z_3$ with $\ker\phi=H$.
			\item What does the 1\st{} isomorphism theorem tell us about the factor group $\quotient GH$?
		\end{enumerate}
	  
	  
	  \item Suppose $G$ is a finite group with normal subgroup $H$ and that $\phi:G\to L$ is a homomorphism with $\ker\phi=H$. Prove that $(G:H)\le \nm L$ with equality if and only if $\phi$ is surjective.
	  
	  
	  \item Consider the map $\phi:\Z\times\Z_{12}\to \Z_3\times\Z_6$ defined by
	  \[
	  	\phi(x,y)=\bigl(2x+y,y\bigr)
	  \]
	  \begin{enumerate}
	    \item Verify that $\phi$ is a well-defined homomorphism.
	    \item Compute $\ker\phi$ and identify the factor group $\quotient{\Z\times\Z_{12}}{\ker\phi}$
	  \end{enumerate}
	  
	  
	  \item Let $H=\ip{(3,1)}\le G=\Z_9\times\Z_3$. Find an explicit homomorphism $\phi:G\to\Z_9$ whose kernel is $H$, and thus identify the factor group $\quotient GH$.\par
	  (\emph{Hint: $(x,y)\in H=\{(0,0),(3,1),(6,2)\}\iff\ldots$})
	  
	  \item Consider $H=\ip{(3,3)}\le G=\Z_9\times\Z_9$. Find a surjective homomorphism $\phi:G\to\Z_3\times\Z_9$ whose kernel is $H$ and hence prove that $\quotient GH\cong\Z_3\times\Z_9$.
	  
	  
	  \item Let $\phi:S^1\to S^1:z\mapsto z^2$.
	  \begin{enumerate}
	    \item Find the kernel of $\phi$ and describe the canonical homomorphism $\gamma:S_1\to\smash{\quotient{S^1}{\ker\phi}}$.
	    \item What does the first isomorphism theorem say about the factor group $\smash{\quotient{S^1}{\ker\phi}}$.
	    \item For each $n$, identify the factor group $\smash{\quotient{S^1}{U_n}}$, where $U_n$ is the group of $n\th$ roots of unity.
	  \end{enumerate}
	 
	  %\item Recall Example \ref*{ex:normalmoreex}.\ref{ex:nastyiso}. If $H=\ip{(1,2,2)}\le G=\Z_5\times\Z_6\times\Z$, try to find a surjective homomorphism $\phi:G\to \Z_5\times\Z_6\times\Z_2$ with $\ker\phi=H$.
	\end{enumerate}
\end{exercises}

\clearpage




\subsection{Conjugation, Cycle Types, Centers and Automorphisms}\label{sec:conj}

In this section we consider an important type of homomorphism and some its consequences.

\begin{defn}{}{}
	Let $G$ be a group and $x,y\in G$. We say that $y$ is \emph{conjugate to} $x$ if
	\[
		\exists g\in G\quad\text{such that}\quad y=gx g^{-1}
	\]
	If $g\in G$ is fixed, then \emph{conjugation by $g$} is the map $c_g:G\to G:x\mapsto gx g^{-1}$.
\end{defn}

We've met this notion before: recall that a subgroup $H$ is normal if and only if $c_g(h)\in H$ for all $g\in G$ (Corollary \ref{cor:normalconj}). It should also be familiar from linear algebra, in the context of \emph{similarity.} Recall that square matrices $A,B$ are similar if $B=MAM^{-1}$ for some invertible $M$. Such matrices have the same eigenvalues and, essentially, `do the same thing' with respect to different bases. An explicit group theory analogue of this is Theorem \ref{thm:cycletype} below.


\begin{lemm}{}{conjiso}
	Conjugation by $g$ is a isomorphism $c_g:G\cong G$.
\end{lemm}

\begin{proof}
	Conjugation by $g^{-1}$ is the inverse function of $c_g^{-1}$:
	\[
		c_{g^{-1}}\bigl(c_g(x)\bigr)=g^{-1}gx g^{-1}(g^{-1})^{-1}=x,\ \text{ etc.}
	\]
	We moreover have a homomorphism:
	\[
		c_g(xy)=g(xy)g^{-1}
		=\bigl(gx g^{-1}\bigr)\bigl(gy g^{-1}\bigr)
		=c_g(x)c_g(y)
		\tag*{\qedhere}
	\]
\end{proof}


\begin{lemm}{}{conjequiv}
	Conjugacy is an equivalence relation \ ($x\sim y\iff \exists g\in G$ such that $y=gx g^{-1}$).
\end{lemm}

The proof is an exercise. The equivalence classes under conjugacy are termed \emph{conjugacy classes.}


\begin{examples}{}{}
	\exstart If $G$ is abelian then every conjugacy class contains only one element:
	\[
		x\sim y\iff \exists g\in G
		\quad\text{such that}\quad 
		y=gx g^{-1}=xg g^{-1}=x
	\]
	\begin{enumerate}\setcounter{enumi}{1}
		\item The smallest non-abelian group is $S_3$ has conjugacy classes
		\[
			\{e\},\quad \{(1\,2),(1\,3),(2\,3)\},\quad \{(1\,2\,3),(1\,3\,2)\}
		\]
		This can be computed directly, but it follows immediately from\ldots
	\end{enumerate}
\end{examples}

\begin{thm}{}{cycletype}
	The conjugacy classes of $S_n$ are the \emph{cycle types}: elements are conjugate if and only if they have the same cycle type.
\end{thm}

If an element $\sigma\in S_n$ is written as a product of disjoint cycles, then its cycle type is clear. For instance:
\begin{itemize}
  \item $(1\,2\,3)(4\,5)$ has the same cycle type as $(1\,5\,6)(2\,3)$: we might call these 3,2-cycles.
  \item $(1\,2)(3\,4)$ has a different cycle type; 2,2.
\end{itemize}

\goodbreak

Before seeing the proof it is beneficial to try an example.

\begin{example}{}{}
If $\rho=(2\,4\,3)$ and $\sigma=(1\,2)(3\,4)$ in $S_4$, then
  \[
  	\rho\sigma\rho^{-1}=(2\,4\,3)(1\,2)(3\,4)(2\,3\,4)=(1\,4)(2\,3)
  \]
  Not only does this have the same cycle type as $\sigma$, but if may be obtained simply by applying $\rho$ to the entries of $\sigma$!
  \[
  	\rho\sigma\rho^{-1}=(1\,4)(2\,3) 
  	=\bigl(\rho(1)\,\rho(2)\bigr)\bigl(\rho(3)\,\rho(4)\bigr)
  \]
  \begin{minipage}[t]{0.75\linewidth}\vspace{-5pt}
	  This also tells us how to reverse the process: given 2,2-cycles $\sigma=(1\,2)(3\,4)$ and $\tau=(1\,4)(2\,3)$ simply place $\sigma$ on top of $\tau$ in a table to define a suitable $\rho=(2\,4\,3)$ for which $\rho\sigma\rho^{-1}=\tau$.
	  \end{minipage}
	  \hfill
	  \begin{minipage}[t]{0.2\linewidth}\vspace{-5pt}
	    $\begin{array}{l|cccc}
	  		x&1&2&3&4\\\hline
	  		\rho(x)&1&4&2&3
	  	\end{array}$
  \end{minipage}
\end{example}

The proof is nothing more than the example done abstractly!

\begin{proof}
	\begin{description}
		\item[\normalfont($\Rightarrow$)] We consider conjugation by $\rho\in S_n$. First let $\sigma=(a_1\cdots a_k)$ be a $k$-cycle and write
		\[
			A=\{a_1,\ldots,a_k\},\qquad R=\{\rho(a_1),\ldots,\rho(a_k)\}
		\]
		Since $\rho$ is a bijection, $\nm R=k$ are distinct and $x\in R\iff \rho^{-1}(x)\in A$. There are two cases:
		\begin{description}
			\item[\normalfont\emph{If $x\in R$}:] \ \ Let $x=\rho(a_j)$, then
			\[
				\rho\sigma\rho^{-1}\bigl(\rho(a_j)\bigr)
				=\rho\sigma(a_j)=\rho(a_{j+1})
			\]
			where $a_{k+1}$ is understood to be $a_1$.
			\item[\normalfont\emph{If $x\not\in R$}:] \ \ Since $\rho^{-1}(x)\not\in A$ it is unmoved by $\sigma$, whence
			\[
				\rho\sigma\rho^{-1}(x) 
				=\rho\sigma(\rho^{-1}(x))
				=\rho\rho^{-1}(x)=x
			\]
		\end{description}
		We conclude that $\rho\sigma\rho^{-1}=\bigl(\rho(a_1)\cdots\rho(a_k)\bigr)$	is also a $k$-cycle!\par
		More generally, if $\sigma=\sigma_1\cdots\sigma_l$ is a product of disjoint cycles, then
		\[
			\rho\sigma\rho^{-1}=(\rho\sigma_1\rho^{-1})(\rho\sigma_2\rho^{-1})\cdots (\rho\sigma_l\rho^{-1})
		\]
		has the same cycle type as $\sigma$.
		\item[\normalfont($\Leftarrow$)] Suppose $\sigma=\sigma_1\cdots\sigma_l$ and $\tau=\tau_1\cdots\tau_l\in S_n$ have the same cycle type, written so that the corresponding orbits have the same length. Moreover, assume we've included all necessary 1-cycles so that $\bigcup\sigma_i=\{1,\ldots,n\}=\bigcup\tau_i$. Define a permutation $\rho$ by writing the orbits of $\sigma$ and $\tau$ on top each other
		\[
			\begin{array}{l|cccccccc}
	      x&\sigma_1 &\sigma_2 & \cdots & \sigma_l\\\hline
	      \rho(x)&\tau_1 & \tau_2 & \cdots & \tau_l
	    \end{array}
	  \]
	If $s_{i,j}$ and $t_{i,j}$ are the $j\th$ elements of the orbits $\sigma_i$ and $\tau_i$, then
		\[
			\rho\sigma\rho^{-1}(t_{i,j})
			=\rho\sigma(s_{i,j})=\rho\bigl(s_{i,j+1}\bigr)
			=t_{i,j+1}=\tau\bigl(t_{i,j}\bigr)
		\]
		We conclude that $\rho\sigma\rho^{-1}=\tau$, as required.\qedhere
	\end{description}
\end{proof}

\goodbreak

\begin{examples}{}{conjexample}
	\exstart The permutations $\sigma=(1\,4\,5)(2\,7\,6)$ and $\tau=(1\,6\,5)(3\,4\,7)$ in $S_7$ are conjugate: the table defines a suitable $\rho$.
	\[
		\begin{array}{l|ccccccc}
      x&1&4&5&2&7&6&3\\\hline
      \rho(x)&1&6&5&3&4&7&2
    \end{array}
    \implies \rho=(2\,3)(4\,6\,7)
   \]
	Indeed
	\[
		\rho\sigma\rho^{-1}
		=(2\,3)(4\,6\,7)(1\,4\,5)(2\,7\,6)(2\,3)(4\,7\,6)
		=(1\,6\,5)(3\,4\,7) =\tau
	\]
	There are other possible choices of $\rho$; just write the orbits of $\sigma,\tau$ in different orders.
	\begin{enumerate}\setcounter{enumi}{1}
		\item (Example \ref*{sec:factor}.\ref{ex:a4normal}) \ We've checked previously that $V=\{e,(1\,2)(3\,4),(1\,3)(2\,4),(1\,4)(2\,3)\}$ is a normal subgroup of $A_4$. It is moreover a normal subgroup of $S_4$: since $V$ contains the identity and all 2,2-cycles it is closed under conjugacy and thus a normal subgroup of both $A_4$ and $S_4$.
	% 	
	% 	 $\quotient{A_4}V$ is a factor group of order $\frac{12}4=3$ which must therefore be isomorphic to $C_3$. Indeed the cosets of $V$ may be written\footnote{In fact $(123)V=(134)V=(243)V=(142)V$ and $(132)V=(143)V=(234)V=(124)V$.}
	% 	\[\quotient{A_4}V=\{V,(123)V,(132)V\}\]
	% 	An example isomorphism $\mu:\quotient{A_4}V\to C_3$ from the first isomorphism theorem is then
	% 	\[\mu:\threevec{V}{(123)V}{(132)V}=\threevec{e}{(123)}{(132)}\]
	% 	$\mu^{-1}(1\,2\,3)^k=(1\,2\,3)^kV$
	% 	where we view $C_3$ in a natural way as a subgroup of $S_4$.
	\end{enumerate}
\end{examples}



\boldsubsubsection{Automorphisms}

We've already seen that conjugation $c_g:G\to G$ by a fixed element is an isomorphism. We now consider all such maps.

\begin{defn}{}{}
	An \emph{automorphisms} of a group $G$ is an isomorphism of $G$ with itself. The set of such is denoted $\Aut G$. The \emph{inner automorphisms} are the conjugations
	\[
		\Inn G=\{c_g:G\to G\text{ where }c_g(x)=gx g^{-1}\}
	\]
\end{defn}

\begin{example}{}{}
	There are four homomorphisms $\phi_k:\Z_4\to\Z_4$ (Corollary \ref{cor:homoszmn});
	\[
		\phi_0(x)=0,\quad 
		\phi_1(x)=x,\quad 
		\phi_2(x)=2x,\quad 
		\phi_3(x)=3x
	\]
	of which two are automorphisms: $\Aut\Z_4=\{\phi_1,\phi_3\}$.
	Observe that $\phi_1$ is the identity function and that $\phi_3\circ\phi_3=\phi_1$. The automorphisms therefore comprise a \emph{group} (necessarily isomorphic to $\Z_2$) under composition of functions.\smallbreak
	As for conjugations, observe that for any $g\in\Z_4$,
	\[
		c_g(x)=g+x+(-g)=x
	\]
	since $\Z_4$ is abelian. There is only one inner automorphism of $\Z_4$, the identity function $\phi_1$.
\end{example}

Hunting for automorphisms can be difficult. Here is a helpful observation for narrowing things down; the proof is an exercise.

\begin{lemm}{}{autoorder}
	If $\phi\in\Aut G$ and $x\in G$, then the orders of $x$ and $\phi(x)$ are identical.
\end{lemm}

This helps to streamline the previous example: $\phi(1)$ must have the same order (four) as 1 and so our only possibilities are $\phi(1)=1$ or $\phi(1)=3$. These possibilities generate the two observed automorphisms.
% 
% \begin{proof}
% Since $\phi$ is a homomorphism, $\phi(e)=e$ and $\bigl(\phi(x)\bigr)^n=\phi(x^n)$. Since $\phi$ is a bijection we conclude
% \[\bigl(\phi(x)\bigr)^n=e\iff \phi(x^n)=\phi(e)\iff x^n=e\tag*{\qedhere}\]
% \end{proof}

\goodbreak


\begin{example}{}{}
	We describe all automorphisms $\phi$ of $S_3$. Consider $\sigma=(1\,2)$ and $\tau=(1\,2\,3)$. Since the order of an element is preserved by $\phi$, we conclude that
  \[
  	\phi(e)=e,\quad 
  	\textcolor{red}{\phi(\sigma)}\in\bigl\{(1\,2),(1\,3),(2\,3)\bigr\},\quad 
  	\textcolor{blue}{\phi(\tau)}\in \bigl\{(1\,2\,3),(1\,3\,2)\bigr\}
 	\]
  We therefore have a maximum of \emph{six} possible automorphism; it is tedious to check, but \emph{all} really do define automorphisms! Indeed all may be explicitly realized as conjugations whence $\Aut S_3=\Inn S_3$. Here is the data; verify some of it for yourself:\vspace{-2pt}
  \[
	  \begin{array}{c||cccccc}
		  \text{element $g$}&c_g(e)&\textcolor{red}{c_g(1\,2)}&c_g(1\,3)&c_g(2\,3)&\textcolor{blue}{c_g(1\,2\,3)}&c_g(1\,3\,2)\\\hline\hline
		  e&e&\textcolor{red}{(1\,2)}&(1\,3)&(2\,3)&\textcolor{blue}{(1\,2\,3)}&(1\,3\,2)\\\hline
		  (1\,2)&e&\textcolor{red}{(1\,2)}&(2\,3)&(1\,3)&\textcolor{blue}{(1\,3\,2)}&(1\,2\,3)\\\hline
		  (1\,3)&e&\textcolor{red}{(2\,3)}&(1\,3)&(1\,2)&\textcolor{blue}{(1\,3\,2)}&(1\,2\,3)\\\hline
		  (2\,3)&e&\textcolor{red}{(1\,3)}&(1\,2)&(2\,3)&\textcolor{blue}{(1\,3\,2)}&(1\,2\,3)\\\hline
		  (1\,2\,3)&e&\textcolor{red}{(2\,3)}&(1\,2)&(1\,3)&\textcolor{blue}{(1\,2\,3)}&(1\,3\,2)\\\hline
		  (1\,3\,2)&e&\textcolor{red}{(1\,3)}&(2\,3)&(1\,2)&\textcolor{blue}{(1\,2\,3)}&(1\,3\,2)
	  \end{array}
  \]
  As the next result shows, the automorphisms again form a group under composition, in this case a group of order 6 which is easily seen to be \emph{non-abelian}: for instance 
  \[
  	c_{(1\,2)}c_{(1\,3)}= c_{(1\,3\,2)}\neq c_{(1\,2\,3)}= c_{(1\,3)}c_{(1\,2)}
  \]
  By process of elimination, we conclude that $\Aut S_3\cong S_3$.
\end{example}


\begin{thm}{}{aut}
	$\Aut G$ and $\Inn G$ are groups under composition. Moreover $\Inn G\triangleleft\Aut G$.
\end{thm}

\begin{proof}
	That $\Aut G$ is a group is simply the fact that composition and inverses of isomorphisms are isomorphisms: you should already have made this argument when answering Exercise \ref*{sec:morph}.\ref{exs:isomorphiccomposition}. By Lemma \ref{lemm:conjiso}, every conjugation is an isomorphism, and it is simple to check that $c_g\circ c_h=c_{gh}$ and $c_g^{-1}=c_{g^{-1}}$: we conclude that $\Inn G\subseteq \Aut G$.\smallbreak
	For normality, we check that $\Inn G$ is closed under conjugation! Let $\tau\in\Aut G$ and $c_g\in\Inn G$. For any $x\in G$, we have\footnotemark
	\begin{align*}
		(\tau c_g\tau^{\textcolor{red}{-1}})(x)&=\tau\Bigl(c_g\bigl(\tau^{\textcolor{red}{-1}}(x)\bigr)\Bigr)\tag{definition of $c_g$}\\
		&=\tau\Bigl(g\bigl(\tau^{\textcolor{red}{-1}}(x)\bigr)g^{\textcolor{blue}{-1}}\Bigr)\\
		&=\Bigl(\tau(g)\Bigr)\Bigl(\tau\bigl(\tau^{\textcolor{red}{-1}}(x)\bigr)\Bigr)\Bigl(\tau(g^{\textcolor{blue}{-1}})\Bigr)\tag*{(since $\tau$ is a homomorphism)}\\
		&=\bigl(\tau(g)\bigr)x\bigl(\tau(g)\bigr)^{\textcolor{blue}{-1}}\tag*{(again since $\tau$ is an homomorphism)}\\
		&=c_{\tau(g)}(x)
	\end{align*}
	We conclude that $\tau c_g\tau^{-1}=c_{\tau(g)}\in \Inn G$, from which $\Inn G\triangleleft\Aut G$.
\end{proof}

\vspace{-2pt}

\footnotetext{%
	The challenge in reading the proof is simply to keep track of where everything lives. To help, the inverse symbol is colored: $\tau^{\textcolor{red}{-1}}$ means the inverse \emph{function,} whereas $g^{\textcolor{blue}{-1}}$ means the inverse of an \emph{element} in $G$.%
}

\goodbreak


\boldsubsubsection{Centers}

We say that an element $g$ in a group $G$ \emph{commutes} with another element $x\in G$ if the order of multiplication is irrelevant: i.e. if $gx=xg$. Otherwise said, if $c_g(x)=x$. It natural to ask whether there are any elements which commute with \emph{all others.} There are two very simple cases:
\begin{itemize}
  \item If $G$ is abelian, then every element commutes with every other element!
  \item The identity $e$ commutes with everything, regardless of $G$.
\end{itemize}
In general, the set of such elements will fall somewhere between these extremes. This subset will turn out to be another normal subgroup of $G$.

\begin{defn}{}{}
	The \emph{center} of a group $G$ is the subset of $G$ which commutes with everything in $G$:
	\[
		Z(G):=\{g\in G:\forall h\in G,\ gh=hg\}
	\]
\end{defn}

We will prove that $Z(G)\triangleleft G$ shortly. First we give a few examples; unless $G$ is abelian, the center is typically difficult to compute, so we omit more of the details.


\begin{examples}{}{centers}
	\exstart $Z(G)=G\iff G$ is abelian.
	\begin{enumerate}\setcounter{enumi}{1}
	  \item $Z(S_n)=\{e\}$ if $n\ge 3$. This is Exercise \ref{exs:centerSn}, though you should also consider Theorem \ref{thm:cycletype}.
	  \item $Z(D_{2n+1})=\{e\}$ and $Z(D_{2n})=\{e,\rho_{n/2}\}$, where $\rho_{n/2}$ is rotation by $180^\circ$. For instance, it is easy to see in $D_{2n+1}$ that any rotation and reflection fail to commute.
	  \item $Z\bigl(\rGL_n(\R)\bigr)=\{\lambda I_n:\lambda\in\R^\times\}$. If you've done enough linear algebra, an argument is reasonably straightforward (Exercise \ref{exs:zglnr})
	\end{enumerate}
\end{examples}

\begin{thm}{}{centernormal}
	For any group $G$:
	\begin{enumerate}
	  \item $Z(G)\triangleleft G$
	  \item $\quotient G{Z(G)}\cong \Inn G$
	\end{enumerate}
\end{thm}

\begin{proof}
	\begin{enumerate}
	  \item The function $\phi:G\to\Inn G$ defined by $\phi(g)=c_g$ is a homomorphism:
		\begin{align*}
			&c_{gh}(x)=(gh)x(gh)^{-1}=g(hx h^{-1})g^{-1}=c_g(c_h(x))\\
			\implies &\phi(gh)=\phi(g)\phi(h)
		\end{align*}
		Now observe that
		\[
			g\in\ker\phi\iff \forall x\in G,\ c_g(x)=gx g^{-1}=x\iff g\in Z(G)
		\]
		from which $\ker\phi=Z(G)$ is a normal subgroup of $G$.
		\item Since $\phi$ is surjective, the 1\st{} isomorphism theorem tells us that
		\[
			\quotient{G}{Z(G)}\cong\image\phi=\Inn G
			\tag*{\qedhere}
		\]
	\end{enumerate}
\end{proof}

\goodbreak

\begin{exercises}{}{}
	Key concepts:
	\begin{quote}
		\emph{Conjugation\qquad conjugacy classes\qquad cycle types are conjugacy classes in $S_n$\\
		(inner) automorphism\qquad center of a group}
	\end{quote}
	
	\begin{enumerate}    
	  \item Either find some $\rho\in G$ such that $\rho\sigma\rho^{-1}=\tau$, or explain why no such element exists:
		\begin{enumerate}
	  	\item $\sigma=(1\,2\,3)$, $\tau=(1\,3\,2)$ where $G=S_3$.
	  	\item $\sigma=(1\,4\,5\,6)(2\,3)(5\,6)$, $\tau=(1\,2\,3\,4)(5\,6)(2\,6)$ where $G=S_6$.
	  	\item $\sigma=(1\,4\,5\,6)(2\,3)(5\,6)$, $\tau=(1\,2)(3\,5\,6)$ where $G=S_6$.
		\end{enumerate}
		
		
	  \item Recall Example \ref{ex:conjexample}.1. Find another element $\nu\neq\rho$ for which $\nu\sigma\nu^{-1}=\tau$.
	  
	  
	  \item Prove Lemma \ref{lemm:conjequiv}. Prove that the relation
	  \[
	  	x\sim y\iff \text{$y$ is conjugate to $x$}
	  \]
	  is an equivalence relation on any group $G$. 
	  
	
		\item\begin{enumerate}
	  	\item Suppose $y$ is conjugate to $x$ in a group $G$. Prove that the orders of $x$ and $y$ are identical.
	  	\item Show that the converse to part (a) is \emph{false} by exhibiting two non-conjugate elements of the same order in some group.
		\end{enumerate}
	
	
		\item Let $H\le G$, fix $a\in G$ and define the \emph{conjugate subgroup} $K=c_a(H)=\{aha^{-1}:h\in H\}$.
	  \begin{enumerate}
	    \item Prove that $K$ is indeed a subgroup of $G$.
	    \item Prove that the function $\psi:H\to K:h\mapsto aha^{-1}$ is an isomorphism of groups.
	    \item If $H\triangleleft G$, what can you say about $c_a(H)$?
	    \item Let $H=\{e,(1\,2)\}\le S_3$ and $a=(1\,2\,3)$. Compute the conjugate subgroup $K=c_a(H)$.
	  \end{enumerate}
	  
	  
		\item We've already seen that $V=\{e,(1\,2)(3\,4),(1\,3)(2\,4),(1\,4)(2\,3)\}$ is a normal subgroup of $S_4$.
		\begin{enumerate}
	  	\item Show that \emph{normal subgroup} is not transitive by giving an example of a normal subgroup $K\triangleleft V$ which is \emph{not normal} in $S_4$.
	  	
	  	\item How many \emph{other} subgroups does $S_4$ have which are isomorphic to $V$? Why are none of them normal in $S_4$?
	  	
			\item Explain why $\quotient{S_4}V$ is a group of order six. Prove that
			\[
				(1\,2)V(1\,3)V\neq (1\,3)V(1\,2)V
			\]
			Hence conclude that $\quotient{S_4}V\cong S_3$.
			
			\item Why is it obvious that the following six left cosets are distinct.
			\[
				V,\ (1\,2)V,\ (1\,3)V,\ (2\,3)V,\ (1\,2\,3)V,\ (1\,3\,2)V
			\]
			(\emph{Hint: Think about how none of the representatives $a$ of the above cosets move the number 4 and consider $aV=bV\iff b^{-1}a\in V\ldots$})
			
			\item Define an isomorphism $\mu:\quotient{S_4}V\to S_3$ and prove that it is an isomorphism.
	\end{enumerate}
	
	
		\item Prove Lemma \ref{lemm:autoorder}: if $\phi\in\Aut G$ and $x\in G$, then $\phi(x)$ has the same order as $x$.
	  
	  
	  \item Describe all automorphisms of the Klein four-group $V$.\par
	  (\emph{Hint: use the previous question!})
	  
	  
	  \item Recall Exercise \ref*{sec:kerimage}.\ref{exs:totient}. Explain why $\Aut\Z_n\cong\Z_n^\times$.\par
	  (\emph{Hint: consider $\phi_k(x)=kx$ where $\gcd(k,n)=1$ and map $\psi:k\mapsto\phi_k$})
	  
	  
	  \item Let $G$ be a group. Prove directly that $Z(G)\triangleleft G$, \emph{without} using Theorem \ref{thm:centernormal}. That is:
		\begin{enumerate}
	  	\item Prove that $Z(G)$ is closed under the group operation and inverses.
	  	\item Prove that $gZ(G)=Z(G)g$ for all $g\in G$.
		\end{enumerate}
		
		
		\item Suppose $n\ge 3$ and that $\sigma\in Z(S_n)$.
		\begin{enumerate}
		  \item By considering $\sigma(1\,2)\sigma^{-1}$, prove that $\{\sigma(1),\sigma(2)\}=\{1,2\}$.
		  \item If $\sigma(1)=2$, repeat the calculation with $\sigma(1\,3)\sigma^{-1}$ to obtain a contradiction.
		  \item Hence, or otherwise, deduce that $Z(S_n)=\{e\}$.
		\end{enumerate}
		
		
		\item\label{exs:zglnr} We identify the center of the general linear group.\par
		The $n\times n$ matrix $A=\scalebox{0.5}{%
		$\begin{pmatrix}
		  0&1&0&\cdots&\cdots&0\\
		  0&0&1&&&0\\
		  0&0&0&&&0\\
		  \vdots&&&\ddots&\ddots&\\
		  0&0&0&&0&1\\
		  0&0&0&\cdots&&0
		\end{pmatrix}$}$ has a single one-dimensional eigenspace: $A\ve_1=\V0$.
		\begin{enumerate}
		  \item Let $B\in Z\bigl(\rGL_n(\R)\bigr)$. Use the fact that $AB=BA$ to prove that $B\ve_1=\lambda\ve_1$ for some $\lambda\neq 0$.
		  \item Let $\vx\in\R^n$ be non-zero and $X$ an invertible matrix for which $X\ve_1=\vx$ (e.g. put $\vx$ in the 1\st{} column of $X$). Prove that $B\vx=\lambda\vx$.
		  \item Since the observation in part (b) holds for \emph{any} $\vx\in\R^n$, what can we conclude about $B$? What is the group $Z\bigl(\rGL_n(\R)\bigr)$?
		\end{enumerate}
		
		
		\item\begin{enumerate}
		  \item Prove that $D_4$ has center $Z(D_4)=\{e,\rho_2\}$, where $\rho_2$ is rotation by $180^\circ$.
		  \item State the cosets of $Z(D_4)$. What is the order of each? Determine whether $\quotient{D_4}{Z(D_4)}$ is isomorphic to $\Z_4$ or to the Klein four-group $V$.
		  \item (Hard) \ Can you find a homomorphism $\phi:D_4\to D_4$ whose kernel is $Z(D_4)$?\par
		  (\emph{Hint: draw a picture and think about doubling angles of rotation and reflection!})
		\end{enumerate}
	
	
	% 	\item Prove that the centers of the following groups are as claimed:
	% 	\begin{enumerate}
	%   	\item 
	%   	\item $Z(A_4)=\{e\}$.
	%   	\item $Z(\mathrm{SL}_2(\R))=\left\{\left(\begin{smallmatrix}
	% \lambda&0\\0&\lambda
	% \end{smallmatrix}\right):\lambda=\pm 1\right\}$.
	% 	\end{enumerate}
	% 	What do the cosets of the center $Z(\mathrm{SL}_2(\R))$ look like in part (c)? That is, if $A\in B\cdot Z(\rSL_2(\R))$, what can you say about $A$?
	
	
	
	% \item The group of \emph{outer automorphisms} of $G$ is the factor group $\operatorname{Out} G=\quotient{\Aut G}{\Inn G}$.
	% \begin{enumerate}
	%   \item Prove that if $G$ is abelian then $\operatorname{Out}G\cong\Z_1$ is trivial.
	%   \item 
	% \end{enumerate}
	
		
	  %\item $Z(\rO_n(\R))=\{\pm I_n\}$.
	
	\end{enumerate}
\end{exercises}





% \item[HWfinal]
% \item Consider the map $\phi:\rO_2(\R)\to\rO_2(\R)$ defined by
% \[\phi(Rot(\theta))=Rot(2\theta),\qquad \phi(Ref(\psi))=Ref(2\psi),\]
% where $Rot(\theta)$ is rotation counter-clockwise through $\theta$, and $Ref(\psi)$ is reflection across the line making angle $\psi/2$ with the $x$-axis.
% \begin{enumerate}
% \item Prove that $\phi$ is a homomorphism and compute its kernel. (\emph{You may use the results from earlier homework sheets and the notes instead of computing $Rot(\theta)Ref(\psi)$, etc., directly.})
% \item Prove that $\phi$ restricts to a homomorphism $\phi:D_{2n}\to D_n$ for any positive integer $n$.
% \item What does the first isomorphism theorem say in the case of part (b)? That is, identify the factor group $\quotient{D_{2n}}{\ker\phi}$.
% \item What about in the case of part (a)? Prove your assertions.
% \end{enumerate}
% 
% 
% 
%   \item Let $X\subseteq G$ be a fixed subset of a group $G$. Define the \emph{normalizer} $N(X)$ of $X$ and \emph{centralizer} $C(X)$ of $X$ as follows
%   \begin{gather*}
% N(X)=\{g\in G:gX=Xg\},\qquad C(X)=\{g\in G:\forall x\in X,\ gx=xg\}.
%   \end{gather*}
% \begin{enumerate}
%   \item Show that $N(X)$ is a subgroup of $G$.
%   \item Write down $C(X)$ and $N(X)$ for the subset $X=\{\mu_1,\mu_2,\mu_3\}\subseteq D_3=\{e,\rho_1,\rho_2,\mu_1,\mu_2,\mu_3\}$.
%   \item Let $n\in N(X)$. Given that $C(X)$ is a subgroup of $N(X)$, show that $n$ has the same left and right cosets with $C(X)$ if and only if
%   \[\forall x\in X,\ \forall c\in C(X),\ ncn^{-1}x=xncn^{-1}.\]
%   \item Using (c), or otherwise, prove that the centralizer of any set $X$ is a normal subgroup of the normalizer of $X$.
%   \item If $X=G$, to what theorem of the course does part (d) reduce? \emph{Part (b) is a hint.}
% \end{enumerate} 


% 
% 
% \item\begin{enumerate}
%   \item $(1,3)\in\Z_2\times\Z_4$ has order $\lcm(2,4)=4$.
%   \item $(4,2,1)\in\Z_6\times\Z_4\times\Z_3$ has order $\lcm(3,2,3)=6$.
%   \item $((123),(15)(234))\in S_3\times S_5$ has order $\lcm(3,6)=6$.
%   \item $(\mu_1,\mu_1)\in D_3\times D_4$ has order $\lcm(2,2)=2$.
% \end{enumerate}
% 
% 
% 
% 
%   
%   \item\begin{enumerate}
%     \item Strictly this requires induction on $x$, the induction step being
%     \[\psi(x+1)=\psi(x)+\psi(1)=kx+k=(k+1)x\pmod n\]
%     \item $\psi_3(37)=3\cdot 37=111=24\cdot 4+15= 15\pmod{24}$ while $\psi_3(1)=3$. Clearly these are unequal. In $\Z_{36}$ we have $1=37$ so, if $\psi_3$ is to be a function we must have $\psi_3(1)=\psi_3(37)$. It follows that $\psi_3$ is not a function (not well-defined).
%     \item $\psi_k$ is well-defined if and only if
%     \begin{align*}
%     \forall x,y\in\Z,\quad\psi_k(x+36y)=\psi_k(x) &\iff \forall x,y,\quad kx+36ky\equiv  kx\pmod{24}\\
%     &\iff \forall y,\quad 36ky\equiv 0\pmod{24}\\
%     &\iff 36k\equiv 0\pmod{24}\\
%     &\iff 3k\equiv 0\pmod{2}\\
%     &\iff k\equiv 0\pmod{2}
%     \end{align*}
%     $\psi_k$ is a homomorphism iff $k\in\{0,2,4,6,8,\ldots,22\}$. There are 12 distinct homomorphisms $\psi:\Z_{36}\to\Z_{24}$.
%     \item None of the homomorphisms are injective: in particular $\psi_k(0)=0=\psi_k(12)$ for all of the homomorphisms. More generally, $\nm{\Z_{36}}>\nm{\Z_{24}}$ so there are no injective functions $\Z_{36}\to\Z_{24}$, let alone homomorphisms.\\
%     None of the homomorphisms are surjective either. In particular it is impossible for $\psi_k(x)$ to ever equal 1.
%     \item We must have $\psi(x)=kx$ for some $k$. This is well-defined iff
%     \begin{align*}
%     \forall x,y\in\Z,\quad\psi(x+12y)=\psi(x) &\iff \forall x,y,\quad kx+12ky\equiv  kx\pmod{24}\\
%     &\iff \forall y,\quad 12ky\equiv 0\pmod{24}\\
%     &\iff 12k\equiv 0\pmod{24}\\
%     &\iff k\equiv 0\pmod{2}
%     \end{align*}
%     Some of these are injective, in particular $\psi_2(x)=2x$. Indeed it is straightforward to see that $\psi_k(x)=kx$ is injective iff $\gcd(k,24)=2$. None are surjective.
%     \item Let $d=\gcd(m,n)$. $\psi_k:\Z_m\to\Z_n$ is a well-defined homomorphism iff
%     \begin{align*}
%     \forall x,y\in\Z,\quad\psi(x+my)=\psi(x) &\iff \forall x,y,\quad kx+mky\equiv  kx\pmod{n}\\
%     &\iff \forall y,\quad kmy\equiv 0\pmod{n}\\
%     &\iff km\equiv 0\pmod{n}\\
%     &\iff k\frac md\equiv 0\pmod{\tfrac nd}\\
%     &\iff k\equiv 0\pmod{\tfrac nd}
%     \end{align*}
%     (\emph{Recall that $\gcd(\frac md,\frac nd)=1\implies \exists \lambda,\mu\in\Z$ such that $\lambda\frac md+\mu\frac nd=1\implies \lambda$ is the multiplicative inverse of $\frac md$ modulo $\frac nd$.})\\
%     There are $d=\gcd(m,n)$ distinct homomorphisms, namely $\psi_0,\psi_{\frac nd},\psi_{\frac{2n}d},\ldots,\psi_{\frac{(d-1)n}d}$. 
%     \item\begin{enumerate}
%       \item Suppose that $\psi:\Z_m\to\Z_n$ is an injective homomorphism. Then $\nm{\image\psi}=m$. However, $\image\psi\le\Z_n$. Since all subgroups of $\Z_n$ have order a divisor of $n$, it follows that $m\mid n$.\\
%       Conversely, suppose that $m\mid n$. Then the homomorphism $\psi_{\frac nm}(x)=\frac nmx$ is easily seen to be injective.
%       \item Suppose that $\psi:\Z_m\to\Z_n$ is a surjective homomorphism. Then $\image\psi=\Z_n$. In particular, there must exist some $x\in\Z_m$ for which $\psi(x)=1$. But then $mx=0$ in $\Z_m$, from which
%       \[0=\psi(0)=\psi(mx)=m\quad\text{in }\Z_n \]
%       Therefore $m\equiv 0\pmod n$, from which we see that $n\mid m$.\\
%       Conversely, if $n\mid m$ then $\gcd(m,n)=n$ which says that \emph{every} $k\in\Z_n$ defines a homomorphism $\psi_k:\Z_m\to\Z_n$. Certainly $\psi_1(x)=x$ is surjective.
% 		\end{enumerate}
% 		\item $\psi:\Z_n\to\Z_n$ is a bijective homomorphism if and only if $\psi(1)$ is a generator of $\Z_n$. The generators of $\Z_n$ are precisely those elements which are relatively prime to $n$: that is, the units.
% 		\item Let $x,y\in\Z_n^\times$. Then $\gcd(x,n)=1=\gcd(y,n)$ which says that
% 		\[\exists \kappa,\lambda,\mu,\nu\in\Z\quad\text{such that}\quad\begin{cases}
% 		\kappa x+\lambda n=1\\
% 		\mu y+\nu n=1
% 		\end{cases}\]
% 		But this says that $\kappa\mu xy=1+n(\lambda\nu-\lambda-\nu)$, from which $\gcd(xy,n)=1$. Thus $\Z_n^\times$ is closed under multiplication.\\
% 		Multiplication is associative, and the identity element $1\in\Z_n$ is certainly a unit.\\
% 		Finally, if $x\in\Z_n^\times$, then $\kappa x+\lambda n=1\implies \kappa x\equiv 1\pmod n$, whence $x^{-1}=\kappa$. 
% 		\item Let $\phi:\Z_n^\times\to\Aut(\Z_n)$ be defined by $\phi(k)=\psi_k$. We have already seen that $\psi_k$ is an isomorphism if and only if $k$ is a unit, so this is a bijective function. Now compute what $\phi(kl)$ does to an element $x\in\Z_n$:
% 		\[\phi(kl)(x)=\psi_{kl}(x)=(kl)x=k(lx)=\left(\phi(k)\circ\phi(l)\right)(x)\]
% 		It follows that $\phi(kl)=\phi(k)\circ\phi(l)$, and so $\phi$ is a homomorphism, and thus an isomorphism.
%   \end{enumerate}

